Thank you for the careful reading of our manuscript and the referee
report and we deeply appreciate your comments and critiques on the
manuscript. The specific comments and suggestions for review have been
followed, and the result is an improved manuscript. In particular,
the discussions surrounding the theoretical results in Section IV and
the experiments in Section V had been expanded, most importantly the
addition of Figure 5, illustrating, through Q-Q plots, 
the convergence of the statistics to their limiting distribution. Below we
give specific responses to the comments of the reviewers.  Again, we
thank the editor and reviewers for their valuable comments.

=================
Referee 1

Although the idea of linearly fusing attributes seems appealing and
the authors present their ideas in a technically sound way, nicely
structured and well supported by theoretical analysis and derivations,
I have some critical reservations:

1. The practicality of the approach: Both examples consider the case
of graphs with k=2 attributes and a convenient single parameter
representation ($\theta$). However for larger k, sets of different
optimal values would be harder to argue about. Also given a set of
real graph snapshots the determination (or approximation) of the
corresponding Markov processes - assuming all properties assumed hold
- is not commented on.

*** Response to point 1:
    The cases where $ k \geq 3$ was not considered in the experiments,
    as we deemed them to be more complex but not more
    difficult. Theoretically, the optimal $\lambda$ for $k \geq 3$ can
    be found for all of the graph invariants by numerical techniques,
    i.e., find the $\lambda$ that maximizes non-central parameters
    $\mu_{\lambda}$ in the cases for size and number of triangles, or
    the ratio $\rho_{\lambda}/\zeta_{\lambda}$ in the case of max
    degree or scan. As for empirical evaluation, one can parametrize 
    the set $\{\lambda \colo \| \lambda \| = 1\}$ for $k \geq 3$
    using hypershperical coordinates and the problem of finding the
    optimal $\lambda$ can be done by exhaustive search. This is 
    inefficient or even intractable for $k \geq 5$, but this is a
    common issue for most global optimization problems. 
    
    The issue of determining or estimating the parameters of the
    underlying Markov chain given a set of snapshots of the graphs is an
    important issue that we failed to comment on explicitly in the
    previous submission. We have added in a paragraph to the revision that
    allude to this issue. In general, it is intractable to estimate the
    parameters of the underlying Markov chains given the snapshots. It is
    possible, however, to estimate the latent vectors associated with the
    first and second-order approximation via a spectral embedding
    technique [1]. We also note that more refined models, where the
    messaging events between each pairs of vertices is modeled as point
    process, e.g. [2,3] can also be formulated. The snapshot of the graphs
    is then a binning of these point processes and the issue of parameters
    estimation is fundamental but feasible in these point processes
    approaches.

    [1] Sussman et. al. ``A consistent dot product embedding for
    stochastic blockmodel graphs'', Online at http://arxiv.org/abs/1108.2228

    [2] Heard et. al. ``Bayesian anomaly detection methods for social
    networks'', Annals of Applied Statistics, (4):645--662, 2010.

    [3] P. Perry and P. J. Wolfe. ``Point process modeling for directed
    interaction networks'', Online at http://arxiv.org/abs/1011.1703v2

***

2. Journal relevance: This is very important since there does not seem
to exist some clear identification of a concrete application from
signal processing. The content would be much more relevant to a
statistics journal or to network analysis audience.

*** Response to point 2:
    
We have added a new section to motivates the consideration of
time-series of attributed graphs. We believe that the notion of a
time-series is a fundamental concept in signal processing and
furthermore the representation of data as (attributed) graphs is
ubiquitous in many applications domain. Also, anomaly/change-point
detection is one of the main problems in time-series analysis. As
such, anomaly detection on time-series of graphs is a natural and
challenging problem. There have been recent work on signal processing
on graphs, with many of them on the construction of wavelets on
graphs. Our paper is similar in spirit to these wavelets paper in that
they all consider signal processing on non-Euclidean data (graphs in
particular), but we have taken a more time-series and statistical 
point of view, as evidenced by our setup of the anomaly detection
problem via hypothesis testings along with the derivations of the
limiting behaviour of the test statistics.  
***

=======
Referee 2

*** Comments
The contributions of this paper are timely, relevant, and novel. The
development of signal processing methods for graphs is an important
topic to the signal processing community, as noted by the special
sessions at recent ICASSP and SSP conferences. The literature on
analysis methods for dynamic graphs is limited, and even more so for
attributed graphs. Furthermore, theoretical results that are also
practically useful are extremely rare.

The main weakness of this paper is that it is not organized well and
is not self-contained.

As such, it is difficult to understand the latent process model
without referring to the previous work (reference [1]). There really
is no introduction to the paper, no context is provided for the
problem--aside from the first sentence of the abstract, the reader is
left with no idea why change point detection in dynamic graphs is
useful, and specifically why one may want to use attributed graphs
rather than graphs without attributes. Finally, the latent process
model is a new model, and it would be beneficial to specify early in
the introduction the relationship between this model and other models
that the reader may be familiar with, namely the mixed-membership
stochastic blockmodel and the latent space models of Hoff et al. To
address these weakness, I would suggest to add a few paragraphs
preceding the current section I.A providing context and relationship
to existing models. The authors should find it useful to refer to
other papers published in the Trans on SP to get an idea on the type
of format, organization and style of writing that the TSP reader
expects.

Furthermore, there are terms and quantities that are introduced
without definition, e.g., the bracket notation "[]", e.g., "[K+1]",
the "cadlag process", and the "rdpm" in Sec 1.A. These are not
notations and quantities that are familiar to a signal processing
audience and should be carefully defined before they are introduced.

Specific comments:
- Page 3, figure 1: Please provide a more descriptive caption with
meanings of the regions and trajectories.
- Page 11, line 38: From the definitions of E and F, they appear to be
scalars, but E+F is defined as the convolution of E and F, which does
not make sense.
- Page 13, lines 10-11: What is the change point t* in this example?
- Page 14, line 49: Change "yields" to "yield"
- Page 15, line 4: A plot of each of the normalized test statistics
T_lambda^l (with the optimal lambda for each statistic) against time
should also be included so the reader can get a sense for how
anomalous this particular time step happens to be.
- Page 17, figure 6: Why are there multiple lines for the second order
approximation and exact model? Are they for different values of r? If
so, the values of r should be indicated somewhere in the plots
themselves (perhaps by reducing the number of values of r shown) or by
listing them in the caption.
- Appendix, page 1, line 31: Change "U-statistics" to "U-statistic"

*** Response

We have reorganized the paper to incorporate the referees comments
about the lack of a proper introduction (Section 2 in the new
manuscript was orignally Section 1 and the new Section 1 now discusses
the motivation behind time-series of attributed graphs). As we allude
to in the manuscript, representations of data as
graphs are ubiquitous in many application
domains and as the graphs representations do change over time,
the notion of a time-series of graphs is natural. Furthermore, in many
application domains, the edges are attributed, e.g., in neuroscience,
the edges can represent either inhibitory or excitatory synapses, in
social networks, the edges can represent the medium of communication
(twitter, email, sms). It is thus desirable for 
any approach that analyzes time-series of
graphs to also extends to attributed graphs
in a transparent manner. 

We have also added several paragraphs to Section 2 to clarify/motivate the
latent process model and to compare it with other models such as the
latent vector model or the dot product model. We have also went
through the paper to carefully note/clarify notations for
terms/quantities such as [K+1] or rdpm. 

For the specific comments, we addressed them in the paper. In
particular, 
1) the caption for Figure 1 is now updated to contains
description of the regions and tracjectories 
2) It is noted that E and F are meant to be random variables. The
statement that defined E and F had been adjusted accordingly. 
3) t* for the synthetic example is noted to be 11.
4) Figure 7 in the paper now contains  a plot of the
normalized test statistics T_{\lambda}^{l} (with the optimal lambda
for each statistic) against time.
5) The multiple lines for the exact model and the second-order
approximation are for different values of r. The set of vertex process
parameters r considered is also noted in the caption of Figure 5.
