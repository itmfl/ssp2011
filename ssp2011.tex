\documentclass[final]{IEEEtran}
\usepackage[utopia]{mathdesign}
\usepackage{amsmath}
%\usepackage{amssymb}
\usepackage{amsthm}
\usepackage{graphicx}
\usepackage{subfigure}
%\usepackage{mathrsfs}
\usepackage{bm}
\newtheorem{theorem}{Theorem}
\newtheorem{lemma}[theorem]{Lemma}
\newtheorem{corollary}[theorem]{Corollary}
\newtheorem{proposition}[theorem]{Proposition}
\theoremstyle{definition}
\newtheorem{definition}{Definition}
\usepackage[colorlinks=true,pagebackref,linkcolor=magenta]{hyperref}
\usepackage[noadjust]{cite}
%\usepackage[colon,sort&compress]{natbib}
%\numberwithin{equation}{section}
\renewcommand\arraystretch{1.2}
\let\underbrace\LaTeXunderbrace
\let\overbrace\LaTeXoverbrace
\newcommand{\argmax}{\operatornamewithlimits{argmax}}
\newcommand{\argmin}{\operatornamewithlimits{argmin}}
\def\mathllap{\mathpalette\mathllapinternal}
\def\mathllapinternal#1#2{%
\llap{$\mathsurround=0pt#1{#2}$}% $
}
\def\clap#1{\hbox to 0pt{\hss#1\hss}}
\def\mathclap{\mathpalette\mathclapinternal}
\def\mathclapinternal#1#2{%
\clap{$\mathsurround=0pt#1{#2}$}%
}
\def\mathrlap{\mathpalette\mathrlapinternal}
\def\mathrlapinternal#1#2{%
\rlap{$\mathsurround=0pt#1{#2}$}% $
}
\bibliographystyle{IEEEtran}
\begin{document}
\title{Attribute fusion in a latent process model for time series of
  graphs}
\author{Carey~E.~Priebe, Nam~H.~Lee, Youngser~Park, and Minh~Tang}%
%\thanks{Johns Hopkins University \\ Department of Applied Mathematics
%  and Statistics \\ Baltimore, Maryland 21218-2682 USA}%
\maketitle
\begin{abstract}
 We consider the problem of anomaly/change point detection given a
 time series of graphs with categorical attributes on the
 edges. Various attributed graph invariants are considered, and their
 power for detection as a function of a linear fusion parameter is
 presented.  
\end{abstract}
\begin{IEEEkeywords}
  Anomaly detection, Attributed Random Graphs, Fusion, Random Dot
  Product Graphs
\end{IEEEkeywords}
\section{Introduction}

\subsection{Latent Process Model}
The latent process model for time series of attributed graphs was
presented in
\cite{lee:_laten_proces_model_time_attrib_random_graph}. We summarized
here the ideas that are relevant to our discussion. We begin by
introducing some terminology. Let $\mathscr{S}$ be the unit simplex in
$\mathbb{R}^{K}$, i.e.,
\begin{equation}
  \mathscr{S} = \{ \xi \in [0,1]^{K}
  \colon \sum_{k = 1}^{K} \xi_k \leq 1 \}.
\end{equation}
A {\em random dot product space} for attributed graphs with vertices
in $[n]$ and edge attributes $\mathscr{K} = [K]
$ is then a pair $(\mathbf{X},G)$ of random elements such that
\begin{enumerate}
\item $\mathbf{X} = \{X_i\}_{i = 1}^{n}$ is a collection of
  $\mathscr{S}$-valued random vectors.
\item $G$ is a random graph with vertices set $[n]$ such that
  \begin{equation}
    \label{eq:1}
    \mathbb{P}(i \sim j \,|\, \mathbf{X} = (x_1, x_2, \dots,
    x_n)) = \langle x_i, x_j \rangle.
  \end{equation}
  and that $\mathbf{P}(i \sim j \,|\, \mathbf{X})$ and $\mathbf{P}(i' \sim
  j' \,|\, \mathbf{X})$ are independent whenever $(i,j) \not = (i',j')$. If
  $i \sim j$ in $G$, then the attribute of the edge $\{i,j\}$ is an
  element of $\mathscr{K}$. In particular, $\{i,j\}$ has attribute $k$
  with probability $x_{i,k} x_{j,k}$. 
\end{enumerate}
Let $\mathscr{K}_{+} = \{1,\dots,K+1\}$. We say that a c\'{a}dl\'{a}g
process $W \colon [0,\infty) \mapsto \mathscr{K}_{+}^{n}$ induces the
sequence of random dot product spaces $\mathscr{V} = \{X(t), G(t)\}_{t
  = 1}^{\infty}$ if
\begin{enumerate}
\item Each $(X(t), G(t))$ is a random dot product space with vertices
  $[n]$ and attributes $\mathscr{K}$. Furthermore, for each
  $i \in [n]$, $k \in \mathscr{K}$ and $t \in \mathbb{N}$, we have
  $X_{i,k}(t)  = \int_{t - 1}^{t}{ \mathbf{1}\{W_i(u) = k\}\, du}$.
\item  For each $t \in \mathbb{N}$, we have
  \begin{equation}
    \label{eq:2}
    \mathbb{P}(G(t) = g \,|\, \mathscr{F}_{\leq t}) = \mathbb{P}(G(t) = g \,|\, X(t))
  \end{equation}
where $\mathscr{F}_{\leq t}$ are the sigma fields generated by $\{W(s)
  \colon s \leq t\}$.
\end{enumerate}
We will call any pair $(\mathscr{V}, W)$ that satisfies the above
properties a random dot process model.  In particular, we are
interested in the pairs $(\mathscr{V}, W)$ possessing the following
properties:
\begin{enumerate}
\item For each $t \in \mathbb{N}$ and vertex $i \in [n]$, there exists
  a matrix $\mathbf{Q}^{(i)}(t)$ such that $W_i$, when restricted to
  the interval $[t, t+1)$, is a stationary, continuous-time Markov
  chain with state space $\mathscr{K}_+$, intensity matrix
  ${\mathbf{Q}^{(i)}(t)}$, and stationary distribution
  $\pi^{(i)}(t)$. $W$ is thus a stationary, continuous-time Markov chain with state
  space $\mathscr{K}_{+}^{n}$ and intentisy matrix
  $\otimes_{i=1}^{n}\mathbf{Q}^{(i)}(t)$.
\item There exists a $t^{*} \in \mathbb{N}$ and a $m < n$ such that 
  \begin{enumerate}
  \item for  $t < t^{*}$
    \begin{gather*}
      \pi^{(i)}(t) \equiv \pi_0 \\
      \mathbf{Q}^{(i)}(t) \equiv \mathbf{Q}_0
    \end{gather*}
  \item  for $t \geq t^{*}$
    \begin{gather*}
      \pi^{(1)}(t) = \dots = \pi^{(m)}(t) = \pi_1 \\
      \pi^{(m+1)}(t)  = \dots = \pi^{(n)}(t) = \pi_0 \\
      \mathbf{Q}^{(1)}(t)  = \dots = \mathbf{Q}^{(m)}(t) = \mathbf{Q}_1 \\
      \mathbf{Q}^{(m+1)}(t) = \dots = \mathbf{Q}^{(n)}(t) = \mathbf{Q}_0 
    \end{gather*}
  \end{enumerate}
\end{enumerate}
The above properties characterize a random dot process model with a
change-point phenomena. We will refer to $(t^{*}, m, \pi_0, \pi_1,
\mathbf{Q}_0, \mathbf{Q}_1)$ as the change parameters. Because $W$ is completely
determined by the change parameters, we often choose to omit $W$ and
only mention $\mathscr{V}$ when referring to random dot process models
with change-point phenomena. For a rdpm $\mathscr{V}$ with change
parameters $(t^{*}, m, \pi_0, \pi_1, \mathbf{Q}_0, \mathbf{Q}_1)$, we can construct
several approximations to $\mathscr{V}$. Of particular interests are the
following two approximations.
\begin{definition}
  \label{def:1}
  Let $\mathscr{V}$ be a rdpm with change parameters $(t^{*}, m,
  \pi_0, \pi_1, \mathbf{Q}_0, \mathbf{Q}_1)$. The first order approximation
  $\bar{\mathscr{V}}$ of $\mathscr{V}$ is the sequence $\{(\bar{X}(t),
  \bar{G}(t)\}_{t = 1}^{\infty}$ of independent random dot product
  spaces such that
 \begin{enumerate}
 \item For $t < t^{*}$,
   \begin{equation}
     \label{eq:5}
     \bar{X}_{i}(t) \equiv \hat{\pi}_0.
   \end{equation}
 \item For $t \geq t^{*}$
   \begin{gather*}
     \bar{X}_{i}(t) \equiv \hat{\pi}_1 \quad \text{for $i \leq m$} \\
     \bar{X}_{i}(t) \equiv \hat{\pi}_0 \quad \text{for $i > m$} 
   \end{gather*}
 \end{enumerate}
 where $\hat{\pi}_0$ and $\hat{\pi}_1$ are sub-probability vectors
 obtained by removing the last coordinate of $\pi_0$ and $\pi_1$. 
\end{definition}
\begin{definition}
  \label{def:2}
  Let $\mathscr{V}$ be a rdpm with change parameters $(t^{*}, m,
  \pi_0, \pi_1, \mathbf{Q}_0, \mathbf{Q}_1)$. Define
  $\mathbf{Z}_0$ and $\mathbf{Z}_1$ by
  \begin{gather}
    \mathbf{Z}_0 = (\mathbf{1}\mathbf{\pi}_0^{T} -
     \mathbf{Q}_0)^{-1}(\mathbf{I} - \mathbf{1}\pi_0^{T}) \\
    \mathbf{Z}_1 = (\mathbf{1}\mathbf{\pi}_1^{T} -
     \mathbf{Q}_1)^{-1}(\mathbf{I} - \mathbf{1}\pi_1^{T})
  \end{gather}
  $\mathbf{Z}_0$ and $\mathbf{Z}_1$ are the fundamental matrices
  for the continuous-time Markov chain on $\mathscr{K}_{+}$ with intensity
  matrix $\mathbf{Q}_0$ and $\mathbf{Q}_1$ (see
  e.g., \cite[p. 55]{asmussen03:_applied_probab_queues}). Let
  $\Sigma_0$ and $\Sigma_1$ be given by
  \begin{gather*}
    \Sigma_0 = \mathrm{diag}(\pi_0) \mathbf{Z}_0 +
    \mathbf{Z}_0^{T}
    \mathrm{diag}(\pi_0) \\
    \Sigma_1 = \mathrm{diag}(\pi_1) \mathbf{Z}_1 +
    \mathbf{Z}_1^{T}
    \mathrm{diag}(\pi_1)
  \end{gather*}
  The second order approximation $\widehat{\mathscr{V}}$ of
  $\mathscr{V}$ is the sequence $\{\widehat{X}(t),
  \widehat{G}(t)\}_{t=1}^{\infty}$ where
  \begin{enumerate}
  \item For each $t$ and each $i \in [n]$, $\widehat{X}_i(t)$ is a
    random vector obtained by truncating a multivariate normal random
    vector $Z_{i}(t)$ to $\mathscr{S}$ with mean and covariance
    matrices given below.
  \item For $t < t^{*}$,
    \begin{gather}
      \mathbb{E}[Z_i(t)] \equiv \hat{\pi}_0 \\
      \mathrm{Var}[Z_i(t)] \equiv \widehat{\Sigma}_0
    \end{gather}
  \item For $t \geq t^{*}$,  
    \begin{gather*}
      \mathbb{E}[Z_i(t)] \equiv \hat{\pi}_0, \quad \text{for $i \leq
        m$}  \\
      \mathbb{E}[Z_i(t)] \equiv \hat{\pi}_1, \quad \text{for $i > 
        m$}  \\
      \mathrm{Var}[Z_i(t)] \equiv \widehat{\Sigma}_{0} \quad \text{for $i
        \leq m$} \\
      \mathrm{Var}[Z_i(t)] \equiv \widehat{\Sigma}_{1} \quad \text{for $i
        > m$}
    \end{gather*}
    where $\widehat{\Sigma}_0$ and $\widehat{\Sigma}_1$ are the $K
    \times K$ matrices
    obtained by removing the last row and column of $\Sigma_0^{(r)}$ and
    $\Sigma_1^{(r)}$, respectively. 
  \end{enumerate}
\end{definition}
Suppose that we have a sequence $\{\mathscr{V}^{r} \colon r > 0 \}$ of
rdpm with vertices $[n]$ and attributes $\mathscr{K}$ where for each
$r > 0$, $\mathscr{V}^{r}$ has change parameters $(t^{*}, m, \pi_0,
\pi_1, r \mathbf{Q}_0, r \mathbf{Q}_1)$. The parameter $r$ can be
thought of as the vertex process rate, i.e., the waiting time
decreases exponentially as $r$ increases. Let us now consider the
sequence of first approximation $\bar{\mathscr{V}}^{r}$ and second
approximations $\widehat{\mathscr{V}}^{r}$ of $\mathscr{V}$. We note
that $\bar{\mathscr{V}}^{r_1} = \bar{\mathscr{V}}^{r_2}$ for any $r_1,
r_2 > 0$. Let us then denote by $\bar{\mathscr{V}}$ the (a.e.) unique
first order approximation of $\mathscr{V}^{r}$ for $r > 0$. If we
denote by $\widehat{\Sigma}_{0}^{(r)}$ and
$\widehat{\Sigma}_{1}^{(r)}$ the matrices $\widehat{\Sigma}_0$ and
$\widehat{\Sigma}_1$ for $\widehat{\mathscr{V}}^{r}$, then
$\widehat{\Sigma}_0^{(r)} = \widehat{\Sigma}_0^{(1)}/r$ and
$\widehat{\Sigma}_{1}^{(r)} = \widehat{\Sigma}_1^{(1)}/r$. Therefore,
as $r \rightarrow \infty$, $\widehat{\Sigma}_0^{(r)} \rightarrow 0$
and $\widehat{\Sigma}_1^{(r)} \rightarrow 0$ and so
$\widehat{\mathscr{V}}^{r} \rightsquigarrow
\bar{\mathscr{V}}$. This indicates that for sufficiently large $r$,
there is little statistical difference amongst the $\bar{\mathscr{V}}$,
$\mathscr{V}^{r}$ and $\widehat{\mathscr{V}}$ (\cite[Theorem
2]{lee:_laten_proces_model_time_attrib_random_graph}). $\bar{\mathscr{V}}$
will thus serve as the basis for our subsequent analysis on graph
invariants for attributed random graphs. 
\section{Change-point detection}
\begin{figure}[htbp]
  \centering
  \includegraphics[width=8cm]{graphics/Fig1-SSP2011.pdf}
  \caption{Notional depiction of the problem of change-point detection
    in a time-series of graphs}
  \label{fig:notional_change_point}
\end{figure}
Let $\mathscr{V}$ be a random dot process model with change parameters
$(t^{*}, m, \pi_0, \pi_1, \mathbf{Q}_0, \mathbf{Q}_1)$. The
change-parameters encapsulate a notion of chatter anomalies, i.e., a
subset of vertices of $\mathscr{V}$ with altered communication
behaviour in an otherwise stationary setting as depicted in
Fig.~\ref{fig:notional_change_point}. We are interested in the problem
of testing, for a $t \in \mathbb{N}$, the hypotheses that $t$ is the
change-point of $\mathscr{V}$, namely
\begin{gather*}
  \mathscr{H}_0 \colon t^{*} > t \\
  \mathscr{H}_A \colon t^{*} = t
\end{gather*}
%\subsection{Graphs Invariants}
%In this paper we consider the problem of detecting chatter anomalies
This will be done using the notion of fusion of attributed graph
invariants.  The particular invariants of interests are the size
$\mathcal{E}$, number of triangles $\tau$, scan $\Psi$, and
max degree $\Delta$. Let $G = (V,E,\phi)$ be an attributed graph with 
$\phi \colon \tbinom{V}{2} \mapsto \mathscr{K}_{+}$. For a given
$\{u,v\} \in \tbinom{V}{2}$, we define a $\Gamma_{uv} \in
\mathbb{R}^{K+1}$ by $\Gamma_{uv}(k) = \mathbb{I}\{\phi(\{u,v\}) =
k\}$. Let $\widehat{\Gamma}_{uv}$ be the first $K$ coordinates of
$\Gamma_{uv}$. Thus $\widehat{\Gamma}_{uv} = \bm{0}$ unless $\{u,v\}
\in E$. Under the independent edges assumption, the
$\Gamma_{uv}$ are independent, and hence the $\widehat{\Gamma}_{uv}$
are also independent. We can consider linear attribute fusion of graph invariants with
parameter $\lambda \in \mathbb{R}^{K}$ via
%\begin{gather}
%  \label{eq:6}
%  \mathcal{E}_{\lambda}(G) = \sum_{k=1}^{K} \lambda_k \sum_{u,v}
%  \mathbb{I}\{ \phi(uv) = k \} \\
%  \tau_{\lambda}(G) = \sum_{(i,j,k) \in K^{3}\vphantom{\tbinom{V}{3}}}
%  \, \, \sum_{(u,v,w)
%      \in \tbinom{V}{3}} \lambda_{i}\lambda_j \lambda_k h_{ijk}(u,v,w) \\
%  \Delta_{\lambda}(G) = \max_{v \in V} \, \sum_{k = 1}^{K} \lambda_k
%  \sum_{u \in N(v)}{\mathbb{I}\{\phi(\{u,v\}) = k\}} \\
%  \label{eq:3}
%  \Psi_{\lambda}(G) = \max_{v \in V} \sum_{k = 1}^{K}
%  \lambda_k \!\!\sum_{u,w \in N(v)}\!\!\! \mathbb{I}\{\phi(\{u,w\}) = k\} 
%  \end{gather}

\begin{align}
  \label{eq:6}
  \mathcal{E}_{\lambda}(G) &= \sum_{\mathclap{\{u,v\} \in
      \tbinom{V}{2}}} \langle
  \lambda, \widehat{\Gamma}_{uv} \rangle \\
  \tau_{\lambda}(G) &= \sum_{\mathclap{\{u,v,w\} \in \tbinom{V}{3}}} \langle
  \lambda, \widehat{\Gamma}_{uv} \rangle \langle \lambda,
  \widehat{\Gamma}_{uw} \rangle \langle \lambda, \widehat{\Gamma}_{vw}
  \rangle \\ 
\Delta_{\lambda}(G) &= \max_{v \in V} \, \sum_{w \in N(v)} \langle
\lambda, \widehat{\Gamma}_{vw} \rangle \\
\label{eq:3}
  \Psi_{\lambda}(G) &= \max_{v \in V} \sum_{u,w \in N(v)} \langle
  \lambda, \widehat{\Gamma}_{uw} \rangle
  \end{align}
  Let $\{G(t)\}$ be a time series of graphs. Let $J_{\lambda}(t)$ be a
  statistic for $G(t)$ of the form as in Eq.~\eqref{eq:6} through
  Eq.~\eqref{eq:3}. We define the normalization
  $\bar{J}^{l}_\lambda(t)$ of $J_{\lambda}$ based on recent past as
\begin{equation}
  \label{eq:4}
 \bar{J}^{l}_{\lambda}(t) = \frac{1}{l}\sum_{s = 1}^{l} J_{\lambda}(t - s) 
\end{equation}
where $l \in \mathbb{N}$ specified the width of the running-average
window. Our main interests is in the normalized fusion
statistic $T_{\lambda}^{l}(t)$ as depicted in Fig.~\ref{fig:temporal}, namely
\begin{equation}
  \label{eq:7}
 T_{\lambda}^{l}(t) = % \begin{cases}
   % J_{\lambda}(t) - J_{\lambda}(t - 1) & \text{if $m = 1$} \\
   \frac{J_{\lambda}(t) -
     \bar{J}_{\lambda}^{l}(t)}{\sqrt{\tfrac{1}{l-1}
       \sum_{s=1}^{l}(J_{\lambda}(t - s) - \bar{J}_{\lambda}^{l}(t))^2}}
%   & \text{if $m \geq 2$}
 %  \end{cases}
\end{equation}
for $l \geq 2$, i.e., we want to find a parameter $\lambda$ such that
the power of the test using $T_{\lambda}^{l}$ is maximum. We note that
$T_{\lambda}^{l}$ is scale invariant in $\lambda$ for all of our graph
invariants. It was noted in
\cite{lee:_laten_proces_model_time_attrib_random_graph} that the
maximum of the power for scale invariant tests is
attainable in $\{ \| \lambda \| \leq 1 \}$. \\ \\
\noindent
As we have mentioned earlier, we will assume that the
number of vertices $n$ is large and that the use of the first order approximation
 $\bar{\mathscr{V}}$ of $\mathscr{V}$ is appropriate in our study of the 
asymptotic theory for these graphs invariants. We strive to
obtain estimates for the power in testing $t^{*} > t$ against $t^{*} =
t$ using $T_{\lambda}(t)$ and from these estimates, find the $\lambda$
that maximize the power. We now introduce a few
additional notations that will be used later on in the paper. First,
let $\pi_{00}, \pi_{01}$, and $\pi_{11}$ be sub-probability vectors
whose components are given by
\begin{gather*}
  \pi_{00}(k) = \pi_{0}(k) \pi_{0}(k) \\
  \pi_{01}(k) = \pi_{0}(k) \pi_{1}(k) \\
  \pi_{11}(k) = \pi_{1}(k) \pi_{1}(k)
\end{gather*}
where $k \in \mathscr{K}$. We also let $\eta_{00}, \eta_{01}$, and
$\eta_{11}$ be matrices of size $K \times K$ defined by 
\begin{gather*}
  \eta_{00} = \mathrm{diag}(\pi_{00}) - \pi_{00} \pi_{00}^{T} \\
  \eta_{01} = \mathrm{diag}(\pi_{01}) - \pi_{01} \pi_{01}^{T} \\
  \eta_{11} = \mathrm{diag}(\pi_{11}) - \pi_{11} \pi_{11}^{T}
\end{gather*}
% The structure of the remaining sections of the paper is as follows. We
% discussed the power estimates for $\mathcal{E}_{\lambda},
% \tau_{\lambda}, \Delta_{\lambda}$ and $\Psi_{\lambda}$ in
% \S~\ref{sec:power-estim-mathc}
%  through \S~\ref{sec:power-estim-psi_l}. We illustrate our analysis with some
% simulation experiments in \S~\ref{sec:experimental-results}  
\begin{figure}[htbp]
  \centering
  \includegraphics[width=8cm]{graphics/Fig4-SSP2011.pdf}
  \caption{Temporal standardization: when testing for change at time
    $t$, the recent past (graphs $G(t - l), \dots, G(t-1))$ is used to
    standardize the invariants}
  \label{fig:temporal}
\end{figure}
\subsection{Power estimates for $\mathcal{E}_\lambda$}
\label{sec:power-estim-mathc}
It was shown in
\cite{lee:_laten_proces_model_time_attrib_random_graph} that if
$J_\lambda(t) = \mathcal{E}_{\lambda}(G(t))$, then $T_{\lambda}^{l}(t)$ 
follows a $t$-distribution with $l - 1$
degrees of freedom in the limit. Specifically, we have the following
results
\begin{theorem}
  \label{thm:9}
  Let $t \in \{l+1, \dots, t^{*}\}$ and $\lambda \in
  \mathbb{R}^{K}$. Define the vector $\zeta$ and the matrix $\xi$ by
  \begin{gather*}
    \zeta = \tbinom{m}{2}(\pi_{11} - \pi_{00}) + (n-m)m(\pi_{01} -
    \pi_{00}) \\
    \xi = \tfrac{l+1}{l}\tbinom{n}{2} \eta_{00} +
          \tbinom{m}{2}(\eta_{11} - \eta_{00}) + (n-m)m(\eta_{01} -
          \eta_{00})
  \end{gather*}
  Define the random variable $\psi^{l}_{\lambda}(t)$ by
  \begin{equation}
    \label{eq:10}
    \psi^{l}_{\lambda}(t) = \begin{cases}
      \sqrt{\frac{\langle \lambda, \tbinom{n}{2} \eta_{00}
            \lambda \rangle}{\langle \lambda, \xi
            \lambda \rangle}} T_{\lambda}^{l}(t) & \text{if $t = t^{*}$} \\
      \sqrt{\frac{l}{l + 1}} T_{\lambda}^{l}(t)& \text{if $t < t^{*}$}
      \end{cases}
  \end{equation}
As $n \rightarrow \infty$,
  $\psi^{l}_{\lambda}(t)$ converges weakly to the Student
  $t$-distribution with $l-1$ degrees of freedom and non-centrality
  parameter $\mu_{\lambda}$, where
  \begin{equation}
    \label{eq:15}
    \mu_{\lambda} = \begin{cases}
      \frac{\langle \lambda, \zeta \rangle}{\sqrt{\langle \lambda, \xi \lambda \rangle}} & \text{if $t = t^{*}$} \\
      0 & \text{if $t < t^{*}$}
    \end{cases}
  \end{equation}
\end{theorem}
We are interested in finding the $\lambda$ that will maximize the
power of the test. The power approximation is determined by various
factors but the dominating factor is likely to be
$\mu_{\lambda}$. The following corollary extends a result in
\cite{lee:_laten_proces_model_time_attrib_random_graph} for $K = 2$ to the case
where $K \geq 2$. 
\begin{corollary}
  \label{cor:1}
  Let $\zeta$ and $\xi$ be as defined in Theorem~\ref{thm:9}. Suppose
  that $\xi$ is also positive definite. Let $\nu^{*}$ be the
  normalized eigenvector corresponding to the largest eigenvalue of $ \xi^{-1/2}
  \zeta \zeta^{T} \xi^{-1/2}$, i.e.,
 \begin{equation}
   \label{eq:9}
  \nu^{*} = \argmax_{\nu \colon \| \nu \| = 1}
  \nu^{T} \xi^{-1/2} \zeta \zeta^{T} \xi^{-1/2}
  \nu.
 \end{equation}
 Then $\lambda^{*} = \tfrac{\xi^{-1/2} \nu^{*}}{\|\xi^{-1/2} \nu^{*} \|}$ satisfies
 \begin{gather}
   \label{eq:8}
 \argmax_{ \|\lambda\| = 1}
 \mu_{\lambda} \, \cap \, \{\lambda^{*}, - \lambda^{*}\} \not = \emptyset \\
 \argmin_{ \| \lambda \| = 1} \mu_{\lambda} \, \cap \, \{\lambda^{*}, -
 \lambda^{*}\} \not = \emptyset.
 \end{gather}
\end{corollary}
\subsection{Power estimates for $\tau_{\lambda}$}
\label{sec:power-estim-tau_l}
The limiting distribution for the number of triangles in unattributed
random graphs was considered in
\cite{nowicki88:_subgr_u_statis_method} for the Erd\"{o}s-Renyi 
and in \cite{rukhin09:_asymp_analy_various_statis_random_graph_infer}
for the kidney-egg model. We note here the small changes that allow us
to extend the results in
\cite{rukhin09:_asymp_analy_various_statis_random_graph_infer,%
nowicki88:_subgr_u_statis_method}
to our attributed graphs model. Let $Y_{uv} = \langle \lambda,
\widehat{\Gamma}_{uv} \rangle$ for $\{u,v\} \in \tbinom{V(t)}{2}$. We
can now write $\tau_{\lambda}(G(t))$ as
\begin{equation}
  \label{eq:45}
  \frac{\tau_{\lambda}(G(t))}{\tbinom{n}{3}} = \tbinom{n}{3}^{-1}
  \sum_{\mathclap{\{u,v,w\} \in \tbinom{V}{3}}} Y_{uv} Y_{uw} Y_{vw}
\end{equation}
$\tau_{\lambda}(G(t))/\tbinom{n}{3}$ is then an
$U$-statistic on $\{Y_{e}\}$, with kernel function $h(Y_{1}, Y_{2},
Y_{3}) = Y_{1} Y_{2} Y_{3}$. By using Hajek's projection method, we
can show that $\tau_{\lambda}(G(t))$ converges to a normal
distribution as $n \rightarrow \infty$. 
\begin{lemma}
  \label{lem:3}
  Let $\tau_{\lambda}^{*}$ be the Hajek's projection of $\tau_{\lambda}$, i.e.,
\begin{equation}
  \label{eq:46}
  \tau_{\lambda}^{*}(G) - \mathbb{E}[\tau_{\lambda}(G)] =
  \sum_{\mathclap{\{u,v\} \in \tbinom{V(t)}{2}}} \, \Bigl(\mathbb{E}[
  \tau_{\lambda}(G) \, \lvert \, Y_{uv}] -
  \mathbb{E}[\tau_{\lambda}(G)]\Bigr)
\end{equation}
For $t < t^{*}$, we have
\begin{align*}
  \mu_0 = \mathbb{E}[\tau_{\lambda}^{*}(G(t))] &= \tbinom{n}{3} \langle 
  \lambda, \pi_{00} \rangle^{3} \\
  \sigma_0 = \sqrt{\mathrm{Var}[\tau_{\lambda}^{*}(G(t))]} &= (n-2) \langle \lambda,
  \pi_{00} \rangle^{2} \sqrt{\tbinom{n}{2} \langle \lambda, \eta_{00}
  \lambda \rangle}
\end{align*}
For $t = t^{*}$, we have
  \begin{gather*}
    \label{eq:42}
    \begin{split}
    \mu_A = \mathbb{E}[\tau_{\lambda}^{*}(G(t))] = \tbinom{m}{3}
    \langle \lambda, \pi_{11} \rangle^{3} &+
    \tbinom{m}{2}(n-m) \langle \lambda, \pi_{11} \rangle \langle
    \lambda, \pi_{01} \rangle^{2} \\ &+ m \tbinom{n-m}{2} \langle \lambda,
    \pi_{01} \rangle^{2} \langle \lambda, \pi_{00} \rangle \\ &+
    \tbinom{n-m}{3} \langle \lambda, \pi_{00} \rangle^{3}
    \end{split} \\
    \begin{split}
    \sigma_A^{2} = \mathrm{Var}[\tau_{\lambda}^{*}(G(t))] = \tbinom{m}{2} \langle \lambda, \eta_{11} \lambda
    \rangle S_1  
    &+ m(n-m) \langle \lambda, \eta_{01} \lambda \rangle S_2  \\
    &+ \tbinom{n-m}{2} \langle \lambda, \eta_{00} \lambda \rangle S_3
    \end{split}
  \end{gather*}
  where $S_1$, $S_2$, and $S_3$ are given by
\begin{align*}
    S_1 &= \Bigl[(m-2) \langle \lambda, \pi_{11} \rangle^{2} + (n-m)
    \langle \lambda, \pi_{01} \rangle^{2} \Bigr]^{2} \\ 
    S_2 &= \Bigl[(m-1)
    \langle \lambda, \pi_{11} \rangle \langle \lambda, \pi_{01}
    \rangle + (n-m-1) \langle \lambda, \pi_{00} \rangle \langle
    \lambda, \pi_{01} \rangle \Bigr]^{2} \\
    S_3 &= \Bigl[m \langle \lambda, \pi_{01} \rangle^{2} + (n-m-2) \langle
    \lambda, \pi_{00} \rangle^{2} \Bigr]^{2} 
    \end{align*}
  As $n \rightarrow \infty$ and $m = \Omega(\sqrt{n})$, we have
  \begin{equation}
    \label{eq:49}
    \frac{\tau_{\lambda}(t) -
      \mathbb{E}[\tau_{\lambda}^{*}(G(t))]}{\sqrt{\mathrm{Var}[\tau_{\lambda}^{*}(G(t))]}}
      \rightsquigarrow \mathcal{N}(0,1)
  \end{equation}
\end{lemma}
From Lemma~\ref{lem:3}, we can show that the limiting distribution of
$T_{\lambda}^{l}(t)$ is once again a Student t-distribution with
$l-1$ degrees of freedom.
\begin{theorem}
  \label{thm:5}
  Let $t \in \{l+1, \dots, t^{*}\}$ and $\lambda \in
  \mathbb{R}^{K}$. Define the random varialbel $\psi_{\lambda}^{l}(t)$
  by
  \begin{equation}
    \label{eq:50}
    \psi_{\lambda}^{l}(t) = \begin{cases}
      \Bigl(\sqrt{\tfrac{l}{l+1}}\,\Bigr) T_{\lambda}^{l}(t) & \text{if $t <
        t^{*}$} \\
      \Bigl(\sqrt{\tfrac{l \sigma_0^{2}}{l \sigma_A^{2} + \sigma_0^{2}} }\,\Bigr)
      T_{\lambda}^{l}(t) & \text{if $t = t^{*}$}
        \end{cases}
  \end{equation}
  As $n \rightarrow \infty$, $\psi_{\lambda}^{l}(t)$ converges to a Student $t$-distribution with $l - 1$ degrees of freedom and
  non-centrality parameter $\mu_{\lambda}$ where
  \begin{equation}
    \label{eq:51}
    \mu_{\lambda} = \begin{cases} 0 & \text{if $t < t^{*}$} \\
      \frac{\mu_A - \mu_0}{\sqrt{l \sigma_A^{2} + \sigma_0^{2}}} &
       \text{if $t = t^{*}$}
      \end{cases}
  \end{equation}
\end{theorem}
The power approximation for $\tau_{\lambda}$ is determined by various
factors, but similar to the power approximation for
$\mathcal{E}_{\lambda}$, the dominating factor is likely to be
  $\mu_\lambda$. Thus, we are also interested in finding the $\lambda$
  that will maximize $\mu_\lambda$. However, because of the high power
  of $\lambda$ in the expression for $\mu_{\lambda}$, an analytic
  solution to $\argmax_{\lambda}{\mu_{\lambda}}$ might not be possible.
\subsection{Power estimates for $\Delta_{\lambda}(t)$}
\label{sec:power-estim-delt}
The analysis of the power estimates for $\Delta_{\lambda}(t)$ depends
on some preliminary results which we now state. We denote by
$\mathcal{G}(\alpha, \beta)$ the Gumbel distribution with location
parameter $\alpha$ and scale parameter $\beta$.  
\begin{proposition}
  \label{prop:3}
  Let $a_n$ and $b_n$ be functions of $n$ given by
  \begin{align*}
    a_n &= (2 \log{n})^{1/2}\Bigl(1 - \frac{\log{\log{n}} + \log{4\pi}}{4 \log{n}} \Bigr) \\ 
    b_n &= (2 \log{n})^{-1/2}
  \end{align*}
  Then as $n \rightarrow \infty$ and $m = \Omega( \sqrt{n \log{n}})$,
  we have
  \begin{equation}
    \label{eq:14}
    \Delta_{\lambda}(t) \rightsquigarrow \begin{cases}
      \mathcal{G}(a_n \sigma_0 + \mu_0, b_n \sigma_0 ) & \text{for $t < t^{*}$.} \\
     \mathcal{G}(a_m \sigma_A + \mu_A, b_m \sigma_A) &
     \text{for $t = t^{*}$.}
     \end{cases}
  \end{equation}
  where
  \begin{align*}
    \mu_0 &= (n-1)\langle \lambda, \pi_{00} \rangle \\
    \sigma_A &= \sqrt{(n-1)\langle \lambda, \eta_{00} \lambda \rangle} \\
    \mu_0 &= \,\, (m - 1) \langle x, \pi_{11} \rangle + (n-
    m)\langle \lambda, \pi_{01} \rangle
    \\ \sigma_A &= \sqrt{ (m - 1) \langle \lambda, \eta_{11} \lambda \rangle + (n -
      m) \langle \lambda, \eta_{01} \lambda \rangle}
    \end{align*}
\end{proposition}
\begin{theorem}
  \label{thm:8}
  Let $t \in \{l+1, \dots, t^{*}\}$ and $\lambda \in
  \mathbb{R}^{K}$. For sufficiently large $n$ and sufficiently large
  $l$, the variable $T_{\lambda}^{l}(t)$ has approximately a
  $\mathcal{G}(\rho_{\lambda}, \varsigma_{\lambda})$ distribution with
\begin{align}
  \label{eq:13}
  \rho_{\lambda} &= \begin{cases}
    \frac{\sqrt{\pi}}{6} \frac{\mu_A - \mu_0 + a_m\sigma_A - (a_n + b_n
      \gamma)\sigma_0}{b_n \sigma_0} & \text{if $t =
        t^{*}$} \\
      - \frac{ \sqrt{\pi}}{6} \gamma & \text{if $t < t^{*}$}
  \end{cases}\\
  \varsigma_{\lambda} &= \begin{cases}
    \frac{\sqrt{\pi} b_m \sigma_A}{6 b_n \sigma_0} & \text{if $t =
      t^{*}$} \\
    \frac{\sqrt{\pi}}{6} & \text{if $t < t^{*}$}
    \end{cases}
\end{align}
$\gamma \approx 0.57721$ is the Euler-Mascheroni constant.
\end{theorem}
We are interested in finding the $\lambda$ that will maximize the
power of the test using $\Delta_{\lambda}$. The power approximation is
determined by various factors, but the dominating factor is
be $\tfrac{\rho_\lambda}{\varsigma_\lambda}$. For sufficiently large $n$
and $m$, we have for $t = t^{*}$,
\begin{equation*}
  \begin{split}
  \frac{\rho_\lambda}{\varsigma_\lambda} &= (1 + o(1)) \frac{\mu_A - \mu_0}{b_m
    \sigma_A} \\ &= (1 + o(1)) \frac{\langle \lambda, (m-1)\pi_{11} + (n-m)\pi_{01} -
    (n-1) \pi_{00} \rangle}{b_m \sqrt{\langle \lambda, ((m-1) \eta_{11} +
    (n-m)\eta_{01}) \lambda \rangle}}
  \end{split}
\end{equation*}
We can thus find the maximum and minimum of
$\tfrac{\rho_{\lambda}}{\varsigma_{\lambda}}$ by solving an 
eigenvalue problem as in Corollary~\ref{cor:1}.
\subsection{Power estimates for $\Psi_{\lambda}(t)$}
\label{sec:power-estim-psi_l}
The limiting distribution for the scan statistics for unattributed
random graphs was considered in
\cite{rukhin:_limit_distr_graph_scan_statis}. This subsection is
concerned with adapting the results in
\cite{rukhin:_limit_distr_graph_scan_statis} to attributed random
graphs. We state here only the relevant results. The proofs are
relegated to the appendices. \\ \\
\noindent
The limiting distribution for $\Psi_{\lambda}(t)$ is Gumbel for both
the cases when $t < t^{*}$ and $t = t^{*}$. To aid the exposition,
 we first introduce some notations. Let $z_n$ be defined, for $n \in \mathbb{N}$, by
\begin{equation}
  \label{eq:20}
   z_n = \sqrt{2
      \log{n}}\Bigl(1 - \frac{\log{\log{n}} + \log{4 \pi}}{4
      \log{n}}\Bigr)
\end{equation}
Let us also define $E$ and $F$ by
\begin{gather*}
 %  B \sim \mathrm{Bin}(m, \langle 1, \pi_{00} \rangle); \quad C \sim \mathrm{Bin}(n-m-1,
 % \langle 1, \pi_{01} \rangle) \\
  E \sim \mathrm{Bin}(m-1, \langle 1, \pi_{11} \rangle) \\ F \sim \mathrm{Bin}(n-m,
  \langle 1, \pi_{01} \rangle)
\end{gather*}%
Let $E + F$ be the convolution of $E$ and $F$. Denote by $\mu_{E+F}$ and
$\sigma_{E+F}^{2}$ the mean and variance of $E + F$. Finally, let
$N_{\kappa} = \mu_{E + F} + z_m \sigma_{E + F}$. We then have the
following results.
%Let $\lambda^{(2)}$ be the
% element-wise square of $\lambda$. Define $C_{00}$, $C_{01}$, $C_{11}$
% and $p_{00}$, $p_{01}$, $p_{11}$ to be
% \begin{gather}
%   \label{eq:19}
%   C_{00} = \tfrac{\langle \lambda^{(2)}, \pi_{00} \rangle}{\langle \lambda,
%     \pi_{00}\rangle}; \quad p_{00} = \tfrac{(\langle \lambda, \pi_{00}
%     \rangle)^{2}}{\langle \lambda^{(2)},\, \pi_{00} \rangle} \\
% C_{11} = \tfrac{\langle \lambda^{(2)}, \pi_{11} \rangle}{\langle \lambda,
%     \pi_{11}\rangle}; \quad p_{11} = \tfrac{(\langle \lambda, \pi_{11}
%     \rangle)^{2}}{\langle \lambda^{(2)},\, \pi_{11} \rangle} \\
% C_{10} = \tfrac{\langle \lambda^{(2)}, \pi_{11} \rangle}{\langle \lambda,
%     \pi_{10}\rangle}; \quad p_{10} = \tfrac{(\langle \lambda, \pi_{10}
%     \rangle)^{2}}{\langle \lambda^{(2)},\, \pi_{10} \rangle} 
% \end{gather}
\begin{lemma}
  \label{lem:5}
  Let $\Psi_{\lambda}(t)$ be the scan statistic for $t < t^{*}$. Let
  $N_{0}$, $a_n$, and $b_n$ be defined by
\begin{align*}
   \mu_0 &= (n-1) \langle 1, \pi_{00} \rangle \\
   \sigma_0 &= \sqrt{(n-1) (\langle 1, \pi_{00} \rangle - \langle
     1, \pi_{00} \rangle^2)} \\ 
    N_{0} &= \mu_0 + z_n \sigma_0 \\
  %  a_n &= N_{0} \frac{\langle \lambda, \pi_{00} \rangle}{\langle
  %    1, \pi_{00} \rangle} + C_{00}p_{00}\tbinom{N_{0}}{2} \\ 
    a_n &=  \langle \lambda, \pi_{00} \rangle \tbinom{N_{0}}{2} \\ 
   b_n &= \langle \lambda, \pi_{00} \rangle N_{0} \frac{\sigma_0}{\sqrt{2 \log
     n}}
%   b_n &= (1 - \tfrac{C_{00}p_{00}}{2} + C_{00}p_{00} N_{0}) \frac{\sigma_0}{\sqrt{2 \log
%     n}}
\end{align*}
Then we have
\begin{equation}
  \frac{\Psi_{\lambda}(t) - a_{n}}{b_n} \rightsquigarrow
  \mathcal{G}(0, 1)
\end{equation}
\end{lemma}%
\begin{lemma}
  \label{lem:6}
  Let $a_{n,m}$ and $b_{n,m}$ be given by
  \begin{gather}
    \label{eq:28}
    \begin{split}
    a_{n,m} =%  N_{\kappa} \frac{\langle x, \pi_{01} \rangle}{\langle
 %      1, \pi_{01} \rangle} \\
 % & + 
\langle \lambda, \pi_{00} \rangle \tbinom{N_\kappa}{2} &+
\langle \lambda, \pi_{11} - \pi_{00} \rangle
\tbinom{\mu_E}{2} \\ &+ \langle \lambda, \pi_{01} - \pi_{00} \rangle \mu_E \mu_F 
    \end{split} \\
    b_{n,m} = \langle \lambda, \pi_{00} \rangle N_\kappa \frac{\sigma_{E + F}}{\sqrt{2
        \log{m}}}
    %b_{n,m} = (1 - \tfrac{C_{00} p_{00}}{2} + C_{00} p_{00}
    %N_\kappa)\frac{\sigma_{E + F}}{\sqrt{2 \log{n_1}}}
  \end{gather}
  If $m = \Omega(\sqrt{n \log n})$ and $m = O(n^{k/(k+1)})$ for some
  $k \in \mathbb{N}$ then
  \begin{equation}
    \label{eq:29}
    \frac{\Psi_{\lambda}(t^{*}) - a_{n,m}}{b_{n,m}}  \rightsquigarrow
    \mathcal{G}(0,1)  \end{equation}
\end{lemma}
\begin{theorem}
  \label{thm:6}
  Let $t \in \{l+1, \dots, t^{*}$ and $\lambda \in
  \mathbb{R}^{K}$. For sufficiently large $n$ and sufficiently large
  $l$, $T_{\lambda}^{l}(t)$ has approximately a
  $\mathcal{G}(\rho_{\lambda}, \varsigma_{l})$ distribution with
  \begin{align}
    \label{eq:52}
    \rho_{\lambda} &= \begin{cases}
      \tfrac{\sqrt{\pi}}{6} \tfrac{a_{n,m} - a_n - b_n \gamma}{b_n} & \text{if
        $t = t^{*}$} \\
      - \tfrac{\sqrt{\pi}}{6} \gamma & \text{if $t < t^{*}$} 
    \end{cases} \\
      \varsigma_{\lambda} &= \begin{cases}
        \tfrac{\sqrt{\pi} b_{n,m}}{6 b_n} & \text{if $t = t^{*}$} \\
        \tfrac{\sqrt{\pi}}{6} & \text{if $t < t^{*}$}
      \end{cases}
  \end{align}
\end{theorem}
The dominating factor in the power approximation for $\Psi_{\lambda}$
is $\tfrac{\rho_{\lambda}}{\varsigma_{\lambda}}$. For $t = t^{*}$ and
sufficiently large $n$ and $l$, we have
\begin{equation}
  \label{eq:53}
  \begin{split}
    \frac{\rho_{\lambda}}{\varsigma_{l}} &= \frac{a_{n,m} - a_n - b_n
      \gamma}{b_{n,m}} \\
    &= (1 + o(1)) \frac{ \langle \lambda, \xi \rangle}{\langle
      \lambda, N_{\kappa} \tfrac{\sigma_{E+F}}{\sqrt{2 \log{n}}}
      \pi_{00} \rangle}
  \end{split}
\end{equation}
where $\xi$ is given by
\begin{equation*}
 \xi =  \Bigl(\tbinom{N_{\kappa}}{2} -
    \tbinom{N_0}{2} - \tbinom{\mu_E}{2} - \mu_E \mu_F\Bigr)\pi_{00} +
    \tbinom{\mu_{E}}{2} \pi_{11} + \mu_E \mu_F \pi_{10}  
\end{equation*}
From Eq.~\eqref{eq:53}, we see that the $\lambda$ that maximimizes
$\tfrac{\rho_\lambda}{\varsigma_{\lambda}}$ would be on the boundary,
i.e., $\lambda(k) \not = 0$ for exactly one $k$. Therefore the asymptotic theory
for scan statistics, in contrast with the other graph invariants that
were considered, indicates that there are no benefits in fusing attributes.  
%\section{Experimental Results}
%\label{sec:experimental-results}
\appendices
\section{Proofs of some stated results}
\label{sec:proofs-some-stated}
\begin{IEEEproof}[Corollary\ref{cor:1}]
  The maximizer of $\mu_\lambda$ also maximizes
  \begin{equation*}
    \mu_{\lambda}^{2} = \frac{\lambda^{T} \zeta \zeta^{T}
      \lambda}{\lambda^{T} \xi \lambda}
  \end{equation*}
  Because $\xi$ is positive definite, there exist a positive definite matrix
  $\xi^{1/2}$ such that $\xi^{1/2} \xi^{1/2} = \xi$. Letting $\nu = \xi^{1/2}
  \lambda$, the above expression can be rewritten as
  \begin{equation*}
    \mu_{\lambda}^{2} = \frac{\nu^{T} \xi^{-1/2} \zeta \zeta^{T}
      \xi^{-1/2} \nu}{ \nu^{T} \nu}
  \end{equation*}
  The claim then follows directly from the Rayleigh-Ritz theorem for
  Hermitean matrices.
\end{IEEEproof}
\begin{IEEEproof}[Lemma~\ref{lem:3}]
  $\tau_{\lambda}(G)$ is a U-statistics with kernel function
  $h(Y_1, Y_2, Y_3) = Y_1 Y_2 Y_3$. By the theory of U-statistics, we
  know that
  \begin{equation}
    \label{eq:48}
    \frac{\tau_{\lambda}(G) -
      \mathbb{E}[\tau_{\lambda}^{*}(G)]}{\sqrt{\mathrm{Var}[\tau_{\lambda}^{*}(G)]}}
    \rightsquigarrow N(0,1)
  \end{equation}
  provided that $\mathrm{Var}[\tau_{\lambda}(G) -
  \tau_{\lambda}^{*}(G)] = o(\mathrm{Var}[\tau_{\lambda}^{*}(G)]$. We
  will not show this here. The deta 

By the independent edge
  assumption, we have 
  \begin{align}
    \mathbb{E}[h(Y_i, Y_j, Y_k) &= \mathbb{E}[Y_i]
  \mathbb{E}[Y_j] \mathbb{E}[Y_k] \\ 
  \mathbb{E}[h(Y_i, Y_j, Y_k) |
  Y_i] &= Y_i E[Y_j] \mathbb{E}[Y_k] 
  \end{align}
 Thus, for $t < t^{*}$, we have $\mathbb{E}[\tau_{\lambda}^{*}(G(t))] = \tbinom{n}{3} \langle
  \lambda, \pi_{00} \rangle^{3}$ and
  \begin{equation}
    \begin{split}
      \mathrm{Var}[\tau_{\lambda}^{*}(G(t))] &=
      \mathrm{Var}\Bigl[\sum_{\{u,v,w\}} Y_{uv} \mathbb{E}[Y_{uw}]
      \mathbb{E}[Y_{vw}]\Bigr] \\
      &= (n-2)^{2} \langle \lambda, \pi_{00} \rangle^{4}
      \mathrm{Var}\Bigl[\sum_{\{u,v\}} Y_{uv} \Bigr] \\
      &= (n-2)^{2} \langle \lambda, \pi_{00} \rangle^{4} \tbinom{n}{2}
      \langle \lambda, \eta_{00} \lambda \rangle
    \end{split}
  \end{equation}
  The case when $t = t^{*}$ can be argued in a similar manner. We
  sketch here the derivation of
  $\mathrm{Var}[\tau_{\lambda}^{*}(G(t))]$. We can partition the set
  $\{u,v\} \in \tbinom{V}{2}$ into the sets 
\begin{align*}
\mathcal{S}_1 &= \{ u,v \in [m]
  \}, \\ \mathcal{S}_2 &= \{ u \in [m], v \in [n] \setminus [m]\}, \\
  \mathcal{S}_3 &= \{ u, v \in [n] \setminus[m]\}
\end{align*} Now, for $\{u,v\} \in \mathcal{S}_1$, we have
\begin{equation}
  \label{eq:58}
  \begin{split}
    \sqrt{S_1} &= \sum_{w \not \in \{u,v\}} \mathbb{E}[ h(Y_{uv},
    Y_{uw}, Y_{vw}) | Y_{uv}] \\ & = ((m-2) \langle \lambda, \pi_{11}
    \rangle^{2} + (n-m) \langle \lambda, \pi_{10} \rangle^{2})Y_{uv}
  \end{split}
\end{equation}
The above expression is reasoned as follows. If $w \in [m]$, then
$\mathbb{E}[Y_{uw}] = \mathbb{E}[Y_{vw}] = \langle \lambda,
\pi_{11}$ and there are $m-2$ possible choices for $w \in [m]$
different from $u$ and $v$. If $w \in [n] \setminus [m]$, then
$\mathbb{E}[Y_{vw}] = \mathbb{E}[Y_{uw}] = \langle \lambda, \pi_{10}
\rangle$ and there are $n - m$ possible choices for $w$. Analogous
reasoning gives the expressions for $S_2$ and $S_3$ in the statement
of the lemma. We therefore have
\begin{equation}
  \label{eq:60}
  \begin{split}
  \mathrm{Var}[\tau_{\lambda}^{*}(G(t))] &=  \mathrm{Var}\Bigl[\sum_{\{u,v,w\}} Y_{uv} \mathbb{E}[Y_{uw}]
      \mathbb{E}[Y_{vw}]\Bigr] \\ &=
      S_1 \mathrm{Var}[\sum_{\{u,v\} \in \mathcal{S}_1} Y_{uv}] + S_2
      \mathrm{Var}[\sum_{\{u,v\} \in \mathcal{S}_1} Y_{uv}] \\ &+ S_3
      \mathrm{Var}[\sum_{\{u,v\} \in \mathcal{S}_1} Y_{uv}] \\ &=
      \tbinom{m}{2} \langle \lambda, \eta_{00} \lambda \rangle S_1 +
      m(n-m) \langle \lambda, \eta_{01} \lambda \rangle S_2 \\ &+
      \tbinom{n-m}{2} \langle \lambda, \eta_{00} \lambda \rangle S_3
  \end{split}
\end{equation}
as desired. To complete the proof one must show that
$\mathrm{Var}[\tau_{\lambda}(G) - \tau_{\lambda}^{*}(G)] =
o(\mathrm{Var}[\tau_{\lambda}^{*}(G)]$ and this follows directly from
the corresponding argument in \cite{nowicki88:_subgr_u_statis_method}
or \cite{rukhin09:_asymp_analy_various_statis_random_graph_infer}.
\end{IEEEproof}
\begin{IEEEproof}[Proposition~\ref{prop:3}]
  Let $v \in V(t)$ and denote by $d_{\lambda}(v;t)$ the (fused) degree
  of vertex $v$, i.e.,
  \begin{equation*}
    d_{\lambda}(v;t) = \sum_{w \in N(v)} \langle \lambda,
    \widehat{\Gamma}_{vw} \rangle
  \end{equation*}
  For $t < t^{*}$, each of the $\widehat{\Gamma}_{vw}$ is a multinomial
  trial with probability vector $\pi_{00}$. The following statements are
  made as $n \rightarrow \infty$ for fixed
  $K$.  By the central limit theorem, we have
  \begin{equation}
    \label{eq:17}
    \frac{d_{\lambda}(v;t) - (n-1) \langle \lambda, \pi_{00}
      \rangle}{\sqrt{(n-1) \langle \lambda, \eta_{00} \lambda \rangle}}
      \rightsquigarrow \mathcal{N}(0, 1)
    \end{equation}
    We can thus consider the degree sequence of $G(t)$ for $t < t^{*}$
    as a sequence of {\em dependent} normally distributed random
    variables. By an argument analogous to the argument for
    Erd\"{o}s-Renyi random graphs in \cite[\S
    III.1]{bollobas85:_random_graph} we can show that the dependency
    among the $\{d_{\lambda}(v;t)\}_{v \in V(t)}$ can be
    ignored. $d_{\lambda}(v;t)$, can thus be considered as a sequence
    of independent random variables from a normal distribution. It is
    widely known that the sample maximum of standard normal random
    variables converges weakly to a Gumbel distribution \cite[\S
    2.3]{galambos87:_asymp_theor_extrem_order_statis}. It is, however,
    not clear whether the convergence of $\Delta_{\lambda}(t)$ to a
    Gumbel distribution continues to hold under the composition of
    weak convergence as outlined above. We avoid this problem by
    showing directly that
    \begin{equation}
      \label{eq:22}
  \mathbb{P}\Bigl(\tfrac{\Delta_{\lambda}(t) - (n-1)
        \langle \lambda, \pi_{00} \rangle}{\sqrt{(n-1) \langle
          \lambda, \eta_{00} \lambda \rangle}} \leq a_n + b_n x\Big)
      \rightarrow 
      e^{-e^{-x}} 
    \end{equation}
    Let $\zeta_v = \tfrac{d_{\lambda}(v;t) - (n-1) \langle \lambda,
      \pi_{00} \rangle}{\sqrt{(n-1) \langle \lambda, \eta_{00} \lambda
        \rangle}}$ and $F_n(u) = \mathbb{P}(\zeta_v \leq u)$. If $n
    \rightarrow \infty$ and $u = O(\sqrt{\log{n}})$, we have the
    following moderate deviations result \cite[Theorem~2, \S
    XVI.7]{feller71:_introd_probab_theor_its_applic,rubin65:_probab}.
    \begin{equation}
      \label{eq:23}
      \frac{1 - F_{n}(u)}{1 - \Phi(u)} = \Bigl[1 + (C\tfrac{u^{3}}{\sqrt{n}}) + O(\tfrac{u^{6}}{n}) \Bigr]
    \end{equation}
    for some constant $C$. Letting $u_n = a_n + b_n x$ in
    Eq.~\eqref{eq:23}, we have
    \begin{equation*}
      \begin{split}
     F_n(u_n) &= 1 - (1 - \Phi(u_n))(1 + C \tfrac{u_n^{3}}{\sqrt{n}} +
     O(\tfrac{u_n^{6}}{n})) \\
     &= \Phi(u_n) + (1 - \Phi(u_n))(C \tfrac{u_n^{3}}{\sqrt{n}} +
     O(\tfrac{u_n^6}{n})) \\
     &= \Phi(u_n) + O(\tfrac{1}{u_n n^{1- \delta}})(C \tfrac{u_n^{3}}{\sqrt{n}} +
     O(\tfrac{u_n^6}{n})) \\
     &= \Phi(u_n) + O(\tfrac{u_n^{5}}{n^{3/2 - \delta}})
     \end{split}
    \end{equation*}
    for some sufficiently small $\delta > 0$. We therefore have
    \begin{equation}
      \label{eq:25}
      \begin{split}
      \mathbb{P}(\max_{v \in [n]} \zeta(v) \leq u_n) &= (F_n(u_n))^{n} \\
      &= \Bigl[\Phi(u_n) + O(\tfrac{u_n^{5}}{n^{3/2 - \delta}})\Bigr]^{n} \\
      &= (\Phi(u_n))^{n} + O(\tfrac{u_n^{5}}{n^{1/2 - \delta}}) \\
      & \rightarrow e^{-e^{-x}} 
      \end{split}
    \end{equation}
    Eq.~\eqref{eq:22} is established and we obtain the limiting Gumbel distribution for
    $\Delta_{\lambda}(t)$ for $t < t^{*}$ in Eq.~\eqref{eq:14}.
    % Now $n(1 - \Phi(u_n)) = \exp(-x)(1 + o(u_n^{-1 + \theta}))$ for
    % some small $\theta > 0$. We thus have
    % \begin{equation}
    %   \label{eq:26}
    %   \begin{split}
    %   \mathbb{P}(\max_{v \in [n]} \zeta(v) \leq u_n) &= \Bigl[ \Phi(u_n) - C
    %   u_n^{3}n^{-3/2} e^{-x}(1 + o(u_n^{-1 + \theta})) + O(u_n^{4}/n)
    %   n^{-1} e^{-x} \Bigr]^{n} \\
    %   &= (\Phi(u_n))^{n}\Bigl[1 - C u_n^{3}n^{-1/2} e^{-x}(n \Phi(u_n))^{-1} +
    %   o(u_n^{-4 + \theta})n^{-2} e^{-x}\Bigr]^{n} \\
    %   &= (\Phi(u_n))^{n} - C u_n^{3}n^{-1/2} e^{-x} (\Phi(u_n))^{n-1} +
    %   o(u_n^{-4 + \theta}n^{-1/2}) e^{-x} \\
    %   &= (1 + o(1)) \Phi(u_n)^{n}
    %   \end{split}
    % \end{equation}
    % where the term $o(1)$ can be made to be independent of
    % $x$. Furthermore, we can verify that $\Phi(u_n)^{n} \rightarrow
    % e^{e^{-x}}$ and thus Eq.~\eqref{eq:22} is established. 
 % Let us define
 %  $\xi_s$ for $s \in \mathbb{R}$ as the number of vertices $v$ such
 %  that $\tfrac{d_\lambda(v;t) - (n-1) \langle \lambda, \pi_{00}
 %    \rangle}{\sqrt{(n-1) \langle \lambda, \eta_{00} \lambda \rangle}}
 %  \geq s$. Note that $\xi_s \leq k$ only if there are at most $k$
 %  vertices with $\tfrac{d_{\lambda}(v;t) - (n-1) \langle \lambda,
 %    \pi_{00} \rangle}{\sqrt{(n-1) \langle \lambda, \eta_{00} \lambda
 %      \rangle}} \geq s$. Thus, $\tfrac{\Delta_{\lambda}(v;t) - (n-1)
 %    \langle \lambda, \pi_{00} \rangle}{\sqrt{(n-1) \langle \lambda,
 %      \eta_{00} \lambda \rangle}} < s$ only if $\xi_s = 0$. Because
 %  the sample maximum of Gaussian random variables converges weakly to
 %  a Gumbel \cite[\S 2.3]{galambos87:_asymp_theor_extrem_order_statis},
 %  by taking $n \rightarrow \infty$ followed by $s \rightarrow \infty$,
 %  we can show that $\tfrac{\Delta_{\lambda}(v;t) - (n-1) \langle \lambda,
 %  \pi_{00} \rangle}{\sqrt{(n-1) \langle \lambda, \eta_{00} \lambda
 %    \rangle}} \rightsquigarrow \mathcal{G}(a_n, b_n)$. 
The case when $t = t^{*}$ can be
derive in a similar manner. We first show that if $m =
\Omega(\sqrt{n \log{n}})$, then $\Delta_{\lambda}(v;t^{*}) \rightsquigarrow
\max_{v \in [m]}{d_{\lambda}(v;t^{*})}$ \cite[Lemma
3.1]{rukhin:_limit_distr_graph_scan_statis}.  
We then show, again by the 
central limit theorem that for $v \in [m]$, $\tfrac{d_{\lambda}(v;t^{*}) -
  \mu_2}{\sigma_2} \rightsquigarrow N(0,1)$. It then follows, similar
to our previous reasoning for the case where $t < t^{*}$, that
$\max_{v \in [m]} \tfrac{d_{\lambda}(v;t^{*}) -
  \mu_2}{\sigma_2} \rightsquigarrow \mathcal{G}(a_m,b_m)$ and we obtain
Eq.~\eqref{eq:14} for $t = t^{*}$.  
\end{IEEEproof}
\begin{IEEEproof}[Theorem~\ref{thm:8}]
  Let $ X \sim \mathcal{G}(\alpha, \beta)$. We consider the
  normalization $\tfrac{X - \mu}{\sigma}$. We have
  \begin{equation*}
    \begin{split}
    \mathbb{P}\Bigl[ \tfrac{X - \mu}{\sigma} \leq z\Bigr]  &= \mathbb{P}[X \leq z
    \sigma + \mu] 
    = e^{-e^{-(z \sigma + \mu - \alpha)/\beta}} \\
      &= e^{- e^{-(z - (\alpha - \mu)/\sigma)/(\beta/\sigma)}}
    \end{split}
  \end{equation*}
  Thus, $\tfrac{X - \mu}{\sigma} \sim \mathcal{G}(\tfrac{\alpha -
    \mu}{\sigma}, \tfrac{\beta}{\sigma})$. Because the sample
  mean and the sample variance are consistent estimators, the claim
  follows after an application of Slutsky's theorem.
\end{IEEEproof}
\begin{IEEEproof}[Lemma~\ref{lem:5}]
Let $\phi_{\lambda}(v;t) = \psi_{\lambda}(v;t) - d_{\lambda}(v;t)$ be
the (fused) locality statistics for
vertex $v$ at time $t$ not including the (fused) degree of $v$, i.e.,
\begin{equation}
  \label{eq:11}
  \phi_{\lambda}(v;t) = \sum_{\substack{uw
      \in N(v) \\ u,w \not = v}} \langle \lambda,
  \widehat{\Gamma}_{uw} \rangle
\end{equation}
The following statements are conditional on $|N(v)| = l$. First
of all, we have
\begin{equation*}
  \phi_{\lambda}(v;t) = \sum_{k=1}^{K}{\lambda_k z_k}
\end{equation*}
where the $(z_1, \dots, z_K)$ are distributed as
\begin{gather*}
  (z_1,z_2,\dots,z_K) \sim \textrm{multinomial}\Bigl(
  \tbinom{l}{2}, \pi_{00}\Bigr) 
\end{gather*}
By the central limit theorem, we have
\begin{equation*}
  \frac{\phi_\lambda(v;t) - \tbinom{l}{2} \langle \lambda, \pi_{00}
    \rangle}{\sqrt{\tbinom{l}{2} \langle \lambda, \eta_{00} \lambda \rangle}}
 \rightsquigarrow \mathcal{N}(0,1) 
\end{equation*}
Let $Y_{l} = C_{00} \mathrm{Bin}(\tbinom{l}{2}, p_{00})$. Then $\mathbb{E}[Y_l] = \tbinom{l}{2} \langle \lambda, \pi_{00} \rangle$
and $\mathrm{Var}[Y_l] = \tbinom{l}{2} \langle \lambda,
\eta_{00} \lambda \rangle$ and again by the central limit theorem, we have
\begin{equation}
  \label{eq:12}
  \frac{\psi_{\lambda}(v;t) - \tbinom{l}{2} \langle \lambda, \pi_{00}
    \rangle}{\sqrt{\tbinom{l}{2} \langle \lambda, \eta_{00} \lambda
      \rangle}} \rightsquigarrow \frac{Y_l - \tbinom{l}{2} \langle \lambda, \pi_{00} \rangle}{\sqrt{\tbinom{l}{2}
    \langle \lambda, \eta_{00} \lambda \rangle}}
\end{equation}
Eq.~\eqref{eq:12} states that the locality statistics for our
attributed random graphs model with $t < t^{*}$
can be approximated by the locality statistics for an Erd\"{o}s-Renyi
graph with edge probability $p_{00}$. The lemma then follows
from Theorem~1.1 in \cite{rukhin:_limit_distr_graph_scan_statis}.
\end{IEEEproof}
\begin{IEEEproof}[Lemma~\ref{lem:6}]
  For ease of exposition we drop the index $t^{*}$ from our discussion. Let
  $\phi_{\lambda}(v) = \psi_{\lambda}(v) - d_{\lambda}(v)$. Let
  $M(v)$ be the number of neighbors of $v$ that lies in $[m]$ and
  $W(v)$ be the number of neighbors of $v$ that lies in $[n]
\setminus [m]$. The following statements are conditional on
$M(v) = l_{\zeta}$ and $W(v) = l_{\xi}$. We have
\begin{equation}
  \phi_{\lambda}(v) = \sum_{k=1}^{K} \lambda_k ( y^{(\zeta)}_k +
  y^{(\xi)}_k + y^{(\omega)}_k)
\end{equation}
where $(y^{(\zeta)}_1, \dots, y^{(\zeta)}_K)$, $(y^{(\xi)}_1,\dots,
 y^{(\xi)}_K)$,  $(y^{(\omega)}_1, \dots, y^{(\omega)}_K)$ are
 distributed as
\begin{gather*}
(y^{(\zeta)}_1,\dots,y^{(\zeta)}_K) \sim \textrm{multinomial}\Bigl(
\tbinom{l_\zeta}{2}, \pi_{11}\Bigr) \\ 
(y^{(\xi)}_1,\dots,y^{(\xi)}_K) \sim \textrm{multinomial}\Bigl(
\tbinom{l_\xi}{2}, \pi_{00}\Bigr) \\
(y^{(\omega)}_1,\dots,y^{(\omega)}_m) \sim \textrm{multinomial}\Bigl(
l_\zeta l_\xi, \pi_{10}\Bigr)
\end{gather*}
Let $\rho$ and $\varsigma$ be defined as
\begin{gather*}
  \rho = \langle \lambda, \tbinom{l_{\zeta}}{2} \pi_{11} +
  \tbinom{l_{\xi}}{2} \pi_{00} + l_{\zeta} l_{\xi} \pi_{10} \rangle \\
  \varsigma = \langle \lambda, \Bigl(\tbinom{l_{\zeta}}{2} \eta_{11} +
  \tbinom{l_{\xi}}{2} \eta_{00} + l_{\zeta} l_{\xi} \eta_{10}\Bigr) \lambda
  \rangle
\end{gather*}
By the central limit theorem, as $l_{\zeta} \rightarrow
\infty$ and $l_{\xi} \rightarrow \infty$
\begin{equation}
  \label{eq:16}
  \frac{\phi_{\lambda}(v) - \rho}{\varsigma} \rightsquigarrow
  \mathcal{N}(0,1)
\end{equation}
Let $Y_{\zeta} \sim C_{11} \mathrm{Bin}\Bigl(\tbinom{l_{\zeta}}{2},
p_{11}\Bigr)$, $Y_{\xi} \sim C_{00} \mathrm{Bin}\Bigl(\tbinom{l_\xi}{2},
p_{00}\Bigr)$ and $Y_{\omega} \sim C_{10} \mathrm{Bin}\Bigl( l_{\zeta}
l_{\xi}, p_{10} \Bigr)$. We also set $Y = Y_{\zeta} + Y_{\xi} +
Y_{\omega}$. By the central limit theorem, we have
\begin{equation}
  \label{eq:21}
 \frac{\phi_{\lambda}(v) - \rho}{\varsigma} \rightsquigarrow
 \frac{Y - \rho}{\varsigma}
\end{equation}
Eq.~\eqref{eq:21} states that the locality statistics
$\phi_\lambda(v)$ for our attributed random graphs model at time $t =
t^{*}$ can be approximated by the locality statistics $Y(v)$ for an
unattributed kidney and egg model. The limiting distribution for the
scan statistics in unattributed kidney-egg graphs had previously been
considered in \cite{rukhin:_limit_distr_graph_scan_statis}. We
provided a sketch of the arguments from
\cite{rukhin:_limit_distr_graph_scan_statis} below, along with some
minor changes to handle the case where the probability of
kidney-kidney and
kidney-egg connections are different. \\ \\
\noindent
Let $G$ be an instance of
$\kappa(n,m,p_{11}, p_{10}, p_{00})$, an unattributed kidney-egg graph
with the probability of egg-egg, egg-kidney, and kidney-kidney
connections being $p_{11}$, $p_{10}$, and $p_{00}$,
respectively. $D(v) = M(v) + W(v)$ is then the degree of $v$ in
$G$. We now show two inequalities relating the tail distribution of
$\Delta(G)$ and $\Upsilon(G) = \max_{v \in V(G)} Y(v)$.
\begin{gather}
  \label{eq:27}
    \limsup\,\, \mathbb{P}( \Upsilon(G) \geq a_{n,m} ) \leq \lim
   \mathbb{P}( \Delta(G) \geq N_\kappa) \\
   \label{eq:30}
  \liminf\,\, \mathbb{P}( \Upsilon(G) \geq a_{n,m} ) \geq \lim \mathbb{P}(
  \Delta(G) \geq N_{\kappa})
\end{gather}
\begin{IEEEproof}[Eq.~\eqref{eq:27}]
 Let $C^{*} = \max\{C_{11},
C_{10}, C_{00}\}$ and $d_{n,m} = \sqrt{2 a_{n,m}/C^{*}}$. We first
note that
\begin{equation*}
\Upsilon(G) \geq a_{n,m} \Rightarrow C^{*} \tbinom{D(v)}{2} \geq
a_{n,m} \Rightarrow D(v) \geq d_{n,m} 
\end{equation*}
Let us define $h(v) = \mathbb{E}[Y(v)]$, i.e., 
\begin{equation*}
  \begin{split}
  h(v) = C_{00}p_{00} \tbinom{D(v)}{2} &+ (C_{11} p_{11} - C_{00}
  p_{00}) \tbinom{M(v)}{2} \\ &+ (C_{10} p_{10} - C_{00} p_{00}) M(v) W(v)
  \end{split}
\end{equation*}
We then have
  \begin{equation*}
    \label{eq:31}
    \begin{split}
    \mathbb{P}(\Upsilon(G) \geq a_{n,m}) &= \mathbb{P}\Bigl( \bigcup_{v
      \in V(G)} Y(v) \geq a_{n,m}\Bigr) \\
    &= \mathbb{P}\Bigl( \bigcup_{v
      \in V(G)} Y(v) \geq a_{n,m}, \, D(v) \geq d_{n,m} \Bigr) \\
    &\leq P_1 + P_2
    \end{split}
  \end{equation*}
  where 
  \begin{align*}
    \vartheta_n &= C_{00} \Bigl[ \tbinom{n}{2} p_{00} (1 -
    p_{00})\Bigr]^{1/2} \log{n} \\
    P_1 &= \mathbb{P}(\bigcup_{v \in V(G)} D(v) \geq d_{n,m}, h(v) \geq
    a_{n,m} - \vartheta_n) \\
    P_2 &= \mathbb{P}(\bigcup_{v \in V(G)} D(v) \geq d_{n,m},
    Y(v) - h(v) \geq \vartheta_n)
  \end{align*}
  We now show that $P_2$ is negligible as $n \rightarrow \infty$. To
  proceed, let $p_{e,f} = \mathbb{P}(M(v) = e, W(v) =
f)$. $P_2$ can then be bounded as follows
\begin{equation*}
  \begin{split}
    \frac{P_2}{n} & \leq \sum_{e + f \geq
        d_{n,m}}{\mathbb{P}( Y(v) - h(v) \geq
      \vartheta_{n}, M(v) = e, W(v) = f) p_{e,f}}
  \\
  &=  \sum_{e + f \geq d_{n,m}}{\mathbb{P}(
    \tfrac{Y(v) - h(v)}{\mathrm{Var}[Y(v)]^{1/2}} \geq
    \tfrac{\vartheta_n}{\mathrm{Var}[Y(v)]^{1/2}}, M(v) = e, W(v) = f) p_{e,f}} \\
    &\leq \sum_{e + f \geq d_{n,m}} (1 + o(1))
    \mathbb{P}(Z \geq \Theta(\log{n})) p_{e,f} = o(n^{-1})
  \end{split}
\end{equation*}
We now consider $P_1$. We note that $P_1 \leq R_1 + R_2$
where
\begin{align*}
  R_1 &= \mathbb{P}\Bigl(\bigcup_{v \in [m]} D(v) \geq
  d_{n,m}, h(v) \geq a_{n,m} - \vartheta_n \Bigr) \\  
  R_2 &= \mathbb{P}\Bigl(\bigcup_{v \in [n] \setminus [m]} D(v) \geq
  d_{n,m}, h(v) \geq a_{n,m} - \vartheta_n \Bigr)
\end{align*}
Let us define $g(v) = h(v) - C_{00}p_{00} \tbinom{D(v)}{2}$.
$R_1$ is then bounded as follows
\begin{equation}
  \begin{split}
    \label{eq:36}
    R_1 &\leq \mathbb{P}\Bigl( \bigcup_{v \in [m]} h(v) \geq
    a_{n,m} - \vartheta_{n} \Bigr) \\
    % &\leq \mathbb{P}\Bigl( \bigcup_{v \in [n_1]}  C_{00} p_{00}
    % \tbinom{D(v)}{2} \geq a_{n,m} - \vartheta_{n} - g(v) \Bigr) \\
    &\leq \mathbb{P}\Bigl(\bigcup_{v \in [m]} D(v) \geq
      \sqrt{\tfrac{2 (a_{n,m} - \vartheta_n -
        g(v))}{C_{00} p_{00}}} \Bigr)
  \end{split}
\end{equation}
We now consider the term $a_{n,m} - g(v)$. We have
\begin{equation*}
  \begin{split}
  a_{n,m} - g(v) &= C_{00} p_{00} \tbinom{N_\kappa}{2} \\ &+ (C_{11}p_{11} -
  C_{00} p_{00})(\tbinom{\mu_E}{2} - \tbinom{M(v)}{2})
  \\ &+ (C_{10} p_{10} - C_{00} p_{00})(\mu_E\mu_F - M(v)W(v))
  \end{split}
\end{equation*}
Let $\mathfrak{E}$ and $\mathfrak{F}$ be sets of vertices defined by
\begin{align}
\mathfrak{E} &=
\{v \colon |M(v) - \mu_E| \leq \sigma_E \log{m}\} \\ \mathfrak{F} &=
\{v \colon |W(v) - \mu_F| \leq \sigma_F \log{(n-m)}\}.
\end{align}
Then we have, for $v \in \mathfrak{E} \cap \mathfrak{F}$
\begin{equation}
  \label{eq:34}
  \begin{split}
  a_{n,m} - g(v) =  C_{00} p_{00} \tbinom{N_\kappa}{2} &+
  \Theta(m^{3/2} \log{m}) \\ &+ 
  \Theta(m \sqrt{n - m}) \\
  \end{split}
\end{equation}
When $m = \Omega(\sqrt{n \log n})$, Eq.~\eqref{eq:34} gives
\begin{equation}
  \label{eq:35}
  a_{n,m} - g(v) = N_{\kappa}^{2}\Bigl(\tfrac{C_{00}p_{00}}{2} + O(n^{-1/2 - a}
  \log{n})\Bigr)
\end{equation}
for some $a > 0$. The set $\{v \in [m]\}$ can be partitition into
$\{v \in [m] \cap (\mathfrak{E} \cap\mathfrak{F})\}$ and $\{v \in [m]
\setminus (\mathfrak{E} \cap \mathfrak{F})\}$. We can show that
$\mathbb{P}\{v \in [m] \setminus (\mathfrak{E} \cap \mathfrak{F})\} =
o(1)$ by using a concentration inequality, e.g., Hoeffding's
bound. We thus have
\begin{equation}
  \label{eq:37}
  \begin{split}
    R_1 &\leq \mathbb{P}\biggl( \bigcup_{\substack{v \in [m] \\ v
        \in \mathfrak{E} \cap \mathfrak{F}}} D(v) \geq
    N_{\kappa} \sqrt{1 + O(\tfrac{\log{n}}{n^{1/2+a}})} \Bigr) + o(n^{-1}) \\
    & = \mathbb{P}\Bigl(\bigcup_{v \in [m]} D(v) \geq N_\kappa +
    O(n^{1/2 - a} \log{n})\Bigr) + o(n^{-1}) \\
    &= \mathbb{P}\Bigl(\Delta \geq \mu_{E+F} +
    \sigma_{E+F}(z_m + O(\tfrac{\log{n}}{n^{a}}))\Bigr) + o(n^{-1}) \\ 
    & \rightarrow \mathbb{P}(\Delta \geq N_{\kappa})
    \end{split}
\end{equation}
The same argument can be applied to $R_2$ to show that
\begin{equation}
  \label{eq:38}
  R_2 \leq \mathbb{P}\Bigl(\bigcup_{v \in [n] \setminus [m]} D(v) \geq
  N_{\kappa}(1 + o(1))\Bigr) = o(1)
\end{equation}
Eq.~\eqref{eq:27} is therefore established.
\end{IEEEproof}
\begin{IEEEproof}[Eq.~\eqref{eq:30}]
  We start by noting that
  \begin{equation*}
    \begin{split}
      \mathbb{P}(\Upsilon(G) \geq a_{n,m}) &=
      \mathbb{P}\Bigl(\bigcup_{v \in [n]}Y(v) \geq
      a_{n,m}\Bigr) \\
      & \geq \mathbb{P}\Bigl(\bigcup_{v \in [m]}Y(v) \geq
      a_{n,m}, D(v) \geq N_\kappa \Bigr) \\
      & \geq \mathbb{P}\Bigl(\bigcup_{v \in [m]} D(v) \geq
      N_\kappa\Bigr) \\ &- \mathbb{P}\Bigl( \bigcup_{v
        \in [m]} Y(v) < a_{n,m}, D(v) \geq N_\kappa\Bigr)
    \end{split}
  \end{equation*}
  We now show that $\mathbb{P}( \cup_{v
        \in [m]} Y(v) < a_{n,m}, D(v) \geq N_\kappa) \rightarrow 0$ as
      $n \rightarrow \infty$. Let $v \in [m]$ be arbitrary. It is then
      sufficient to show that $m\mathbb{P}(Y(v) < a_{n,m}, D(v) \geq
      N_{\kappa}) = o(1)$. We note that $\mathbb{P}(Y(v) < a_{n,m}, D(v) \geq
      N_{\kappa})$ can be rewritten as
      \begin{equation}
        \label{eq:24}
        \sum_{e + f \geq N_{\kappa}}{\mathbb{P}(Y(v) \leq a_{n,m}, M(v) = e, W(v) =
          f)}p_{e,f}
      \end{equation}
We now split the indices set $e + f \geq N_{\kappa}$ in
Eq.~\eqref{eq:24} into three parts $S_1$, $S_2$ and $S_3$, namely
\begin{gather}
  \label{eq:32}
    S_1 = \{ \{e,f\}\colon e \geq
        \mu_E + \sigma_E \log{m}\} \\
    S_2 = \{ \{e,f\} \colon  e \leq
        \mu_E + \sigma_E \log{m},e + f \leq
        N_\kappa + \varphi(n)\} \\
    S_3 = \{ \{e,f\} \colon e \leq \mu_E + \sigma_E \log{m}, e + f \geq
        N_\kappa + \varphi(n)\}
\end{gather}
where $\varphi(n) = \Theta(n^{1/2 - a})$ for some $a > 0$. We can then
show that $m\mathbb{P}(M(v) = e,
W(v) =f, \{e,f\} \in S_1) = o(1)$ by applying a concentration
inequality. Similarly, $e + f \geq N_{\kappa}$ and $e \leq \mu_{E}
+ \sigma_{E} \log{m}$ implies that
\begin{equation}
  \label{eq:33}
  f \geq \mu_{F} + (z_{m} - o(1)) \sigma_F 
\end{equation}
and once again, by a concentration inequality, we can show that $m
\mathbb{P}(M(v) = e, W(v) = f, \{e,f\} \in S_2) = o(1)$. As for
$S_3$, from the fact that $e + f \geq N_{\kappa} + \varphi(n)$, we have the bound
  \begin{equation}
    \begin{split}
    a_{n,m} - h(v) &\leq (C_{11} p_{11} - C_{10} p_{10})[m \sigma_E
    \log{m} + \tbinom{\log{m}}{2}] \\ &- C_{00}p_{00} N_\kappa \varphi(n)
    \end{split}
  \end{equation}
  As $\mathrm{Var}[Y(v)] = \Theta(N_\kappa)$ for $\{M(v), W(v)\} \in
  S_3$, we have
  \begin{equation}
    \label{eq:39}
    \begin{split}
      p_{S_3} &= \sum_{\{e,f\} \in S_3} \mathbb{P}( Y(v) < a_{n,m}) p_{e,f} 
      \\ &\leq \sum_{\{e,f\} \in S_3} \mathbb{P}\Bigl(\tfrac{Y(v) - h(v))}{\mathrm{Var}[Y(v)]^{1/2}} \leq \tfrac{a_{n,m} -
      h(v)}{\mathrm{Var}[Y(v)]^{1/2}}\Bigr)p_{e,f} \\
    &\leq \sum_{\{e,f\} \in S_3} \mathbb{P}\Bigl[Z \leq
    O\bigl(\tfrac{m^{3/2} \log m}{N_{\kappa}} - \varphi(n)\bigr)\Bigr] p_{e,f}
    \end{split}
  \end{equation}
  We now set $a = \tfrac{1}{2(k+1)}$. Then for $m =
  O(n^{k/(k+1)})$ and $\varphi(n) = O(n^{1/2 - a})$ we have
  \begin{equation}
    \label{eq:41}
    \tfrac{m^{3/2} \log m}{N_\kappa} - \varphi(n) =
    -O(n^{k/2(k+1))})
  \end{equation}
  which then implies
  \begin{equation}
    \label{eq:44}
    m p_{S_3} \leq m \sum_{\{e,f\} \in S_3} \mathbb{P} \Bigl[Z \leq
    - O(n^{k/2(k+1)}\Bigr] p_{e,f} = o(1)
  \end{equation}
  Thus $\mathbb{P}(Y(v) < a_{n,m}, D(v) \geq
      N_{\kappa}) \rightarrow 0$ as desired.
\end{IEEEproof}
From Eq.~\eqref{eq:27} and Eq.~\eqref{eq:30}, we have
\begin{equation}
  \label{eq:40}
 \lim
\mathbb{P}(\Upsilon(G) \geq a_{n,m}) = \lim \mathbb{P}(\Delta(G) \geq
N_{\kappa})
 \end{equation}
Let $N_{\kappa,y} = N_\kappa + y \tfrac{\sigma_{E+F}}{\sqrt{2
    \log{m}}}$. We now define $a_{n,m,y}$ as 
\begin{equation}
  \label{eq:18}
  \begin{split}a_{n,m,y} &= \langle \lambda, \pi_{00} \rangle
    \tbinom{N_{\kappa,y}}{2} +
  \langle \lambda, \pi_{11} - \pi_{00} \rangle \tbinom{\mu_E}{2}  +
  \langle \lambda, \pi_{10} - \pi_{00} \rangle \mu_E \mu_F \\
  &= a_{n,m} + \langle \lambda, \pi_{00} \rangle y
  \frac{\sigma_{E+F}}{\sqrt{2 \log m}} \Bigl(N_{\kappa} + y
  \frac{\sigma_{E+F}^{2}}{2 \sqrt{2 \log{m}}} + O(1) \Bigr) \\
  &= a_{n,m} + (y + o(1)) b_{n,m} 
  \end{split}
\end{equation}
We therefore have
\begin{equation}
  \label{eq:43}
  \begin{split}
  \lim \mathbb{P}(\Upsilon(G) \geq a_{n,m,y}) &= 
  \lim \mathbb{P}\Bigl(\frac{\Upsilon(G) - a_{n,m}}{b_{n,m}} \geq y \Bigr) \\
  &= \lim \mathbb{P}(\Delta(G) \geq N_{\kappa,y}) \\
  &= \lim \mathbb{P}\Bigl(\frac{\Delta(G) -
    N_{\kappa}}{\sigma_{E+F}} \geq \frac{y}{\sqrt{2 \log{m}}} \Bigr) \\
  \end{split}
\end{equation}
Because $\Delta(G)$ converges weakly to a Gumbel
distribution in the limit
(\cite{bollobas85:_random_graph,rukhin:_limit_distr_graph_scan_statis}),
we have
\begin{equation}
  \label{eq:47}
  \mathbb{P}\Big(\frac{\Upsilon(G) - a_{n,m}}{b_{n,m}} \leq
    y\Bigr) \rightarrow e^{- e^{-y}} 
\end{equation}
and Eq.~\eqref{eq:29} follows.
\end{IEEEproof}
\bibliography{ssp2011}
\end{document}
%%% Local Variables: 
%%% mode: latex
%%% TeX-master: t
%%% End: 
