\documentclass[draftcls]{IEEEtran}
\usepackage{amsmath}
\usepackage{amssymb}
\usepackage{amsthm}
\usepackage{graphicx}
\usepackage{subfigure}
\usepackage{mathrsfs}
\usepackage{bm}
\newtheorem{theorem}{Theorem}
\newtheorem{lemma}[theorem]{Lemma}
\newtheorem{proposition}[theorem]{Proposition}
\usepackage[colorlinks=true,pagebackref,linkcolor=magenta]{hyperref}
\usepackage[colon,sort&compress]{natbib}
%\numberwithin{equation}{section}
\renewcommand\arraystretch{1.2}
\let\underbrace\LaTeXunderbrace
\let\overbrace\LaTeXoverbrace
\newcommand{\argmax}{\operatornamewithlimits{argmax}}
\newcommand{\argmin}{\operatornamewithlimits{argmin}}
\bibliographystyle{IEEEtran}
\begin{document}
\title{Attribute fusion in a latent process model for time series of
  graphs}
\author{Carey~E.~Priebe, Nam~H.~Lee, Youngser~Park, and Minh~Tang}%
%\thanks{Johns Hopkins University \\ Department of Applied Mathematics
%  and Statistics \\ Baltimore, Maryland 21218-2682 USA}%
\maketitle
\begin{abstract}
 We consider the problem of anomaly/change point detection given a
 time series of graphs with categorical attributes on the
 edges. Various attributed graph invariants are considered, and their
 power for detection as a function of a linear fusion parameter is
 presented.  
\end{abstract}
\begin{IEEEkeywords}
  Anomaly detection, Attributed Random Graphs, Fusion, Random Dot
  Product Graphs
\end{IEEEkeywords}
\section{Introduction}
\subsection{Latent Process Model}
The latent process model for time series of attributed graphs was
originally presented in
\cite{lee:_laten_proces_model_time_attrib_random_graph}. We summarized
here the ideas that are relevant to our discussion. We begin by
introducing some terminology. Let $\mathscr{S}$ be the unit simplex in
$\mathbb{R}^{K}$, i.e.,
\begin{equation}
  \mathscr{S} = \{ \xi \in [0,1]^{K}
  \colon \sum_{k = 1}^{K} \xi_k \leq 1 \}.
\end{equation}
A {\em random dot product space} for attributed graphs with vertices
in $[n]$ and edge attributes $\mathscr{K} = [K]
$ is then a pair $(\mathbf{X},G)$ of random elements such that
\begin{enumerate}
\item $\mathbf{X} = \{X_i\}_{i = 1}^{n}$ is a collection of
  $\mathscr{S}$-valued random vectors.
\item $G$ is a random graph with vertices set $[n]$ such that
  \begin{equation}
    \label{eq:1}
    \mathbb{P}(i \sim j \,|\, \mathbf{X} = (x_1, x_2, \dots,
    x_n)) = \langle x_i, x_j \rangle.
  \end{equation}
  and that $\mathbf{P}(i \sim j \,|\, \mathbf{X})$ and $\mathbf{P}(i' \sim
  j' \,|\, \mathbf{X})$ are independent whenever $(i,j) \not = (i',j')$. If
  $i \sim j$ in $G$, then the attribute of the edge $\{i,j\}$ is an
  element of $\mathscr{K}$. In particular, $\{i,j\}$ has attribute $k$
  with probability $x_{i,k} x_{j,k}$. 
\end{enumerate}
Let $\mathscr{K}_{+} = \{1,\dots,K+1\}$. We say that a c\'{a}dl\'{a}g
process $W \colon [0,\infty) \mapsto \mathscr{K}_{+}^{n}$ induces the
sequence of random dot product spaces $\mathscr{V} = \{X(t), G(t)\}_{t
  = 1}^{\infty}$ if
\begin{enumerate}
\item Each $(X(t), G(t))$ is a random dot product space with vertices
  $[n]$ and attributes $\mathscr{K}$. Furthermore, for each
  $i \in [n]$, $k \in \mathscr{K}$ and $t \in \mathbb{N}$, we have
  $X_{i,k}(t)  = \int_{t - 1}^{t}{ \mathbf{1}\{W_i(u) = k\}\, du}$.
\item  For each $t \in \mathbb{N}$, we have
  \begin{equation}
    \label{eq:2}
    \mathbb{P}(G(t) = g \,|\, \mathscr{F}_{\leq t}) = \mathbb{P}(G(t) = g \,|\, X(t))
  \end{equation}
where $\mathscr{F}_{\leq t}$ are the sigma fields generated by $\{W(s)
  \colon s \leq t\}$.
\end{enumerate}
We will call any pair $(\mathscr{V}, W)$ that satisfies the above
properties a random dot process model. In particular, we are
interested in the pairs $(\mathscr{V}, W)$ possessing the following
properties. 
\begin{enumerate}
\item For each time $t \in \mathbb{N}$ and vertex $i \in [n]$, there
  exists a matrix $\mathbf{Q}^{(i)}(t)$ such that for each $k,k' \in
  \mathscr{K}_{+}$
  \begin{equation}
    \label{eq:3}
    \mathbb{P}(W_{i}(t + \epsilon) = k' \, | \, W_{i}(t) = k)
    = \begin{cases}
      1 - \mathbf{Q}_{kk}^{(i)}(t)\epsilon + o(\epsilon) & \text{if $k
        = k'$} \\
      \mathbf{Q}_{kk}^{(i)}(t)\epsilon + o(\epsilon) &
      \text{otherwise}
    \end{cases}
  \end{equation}
\end{enumerate}

\bibliography{ssp2011}
\end{document}

%%% Local Variables: 
%%% mode: latex
%%% TeX-master: t
%%% End: 
