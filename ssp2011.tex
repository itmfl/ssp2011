\documentclass[10pt,journal,compsoc]{IEEEtran}
\usepackage{fontenc}
\pdfminorversion=4
%\usepackage{lmodern}
\usepackage[utopia]{mathdesign}
\usepackage{graphicx}
%\usepackage{xltxtra}
%\setmainfont[Mapping=tex-text]{Linux Libertine}
\usepackage[cmex10]{amsmath}
\interdisplaylinepenalty=2500
\usepackage{array}
%\usepackage{amssymb}
\usepackage{amsthm}
%\usepackage[tight,normalsize,sf,SF]{subfigure}
\ifCLASSOPTIONcompsoc
\usepackage[font=normalsize,labelfont=sf,textfont=sf]{subfig}
\else
\usepackage[font=footnotesize]{subfig}
\fi
\usepackage{subfig}
\usepackage{parskip}
\usepackage{mathrsfs}
\usepackage{bm}
\newtheorem{theorem}{Theorem}
\newtheorem{lemma}[theorem]{Lemma}
\newtheorem{corollary}[theorem]{Corollary}
\newtheorem{proposition}[theorem]{Proposition}
\theoremstyle{definition}
\newtheorem{definition}{Definition}
\usepackage[colorlinks=true,pagebackref,linkcolor=magenta]{hyperref}
\usepackage[noadjust]{cite}
\renewcommand\arraystretch{1.2}
\newcommand{\argmax}{\operatornamewithlimits{argmax}}
\newcommand{\argmin}{\operatornamewithlimits{argmin}}
\def\mathllap{\mathpalette\mathllapinternal}
\def\mathllapinternal#1#2{%
\llap{$\mathsurround=0pt#1{#2}$}% $
}
\def\clap#1{\hbox to 0pt{\hss#1\hss}}
\def\mathclap{\mathpalette\mathclapinternal}
\def\mathclapinternal#1#2{%
\clap{$\mathsurround=0pt#1{#2}$}%
}
\def\mathrlap{\mathpalette\mathrlapinternal}
\def\mathrlapinternal#1#2{%
\rlap{$\mathsurround=0pt#1{#2}$}% $
}
\bibliographystyle{IEEEtran}
\begin{document}
\title{Attribute fusion in a latent process model for time series of
  graphs}
\author{Minh~Tang, Youngser~Park, Nam~H.~Lee, and Carey~E.~Priebe%
\IEEEcompsocitemizethanks{\IEEEcompsocthanksitem Minh~Tang, Nam~H.~Lee
 and Carey~E.~Priebe are with the Department of Applied Mathematics
 and Statistics, Johns Hopkins University.\protect \\
  Email: \{mtang10, nhlee, cep\}@jhu.edu
\IEEEcompsocthanksitem Youngser Park is with the Center for Imaging
Science , Johns Hopkins University.\protect \\
  Email: youngser@jhu.edu}%
\thanks{}}
\markboth{IEEE Transactions on Pattern Analysis and Machine
  Intelligence (TPAMI)}{}
\IEEEcompsoctitleabstractindextext{%
\begin{abstract}
  Hypothesis testing on time series of attributed graphs has
  applications in diverse areas, e.g., social network analysis
  (wherein vertices represent individual actors or organizations),
  connectome inference (wherein vertices are neurons or brain regions)
  and text processing (wherein vertices represent authors or
  documents).  We consider the problem of anomaly/change point
  detection given the latent process model for time series of graphs
  with categorical attributes on the edges presented in \cite{lee11}.
  Various attributed graph invariants are considered, and their power
  for detection as a function of a linear fusion parameter is
  presented.  Our main result is that inferential performance in
  mathematically tractable first-order and second-order approximation
  models does provide guidance for methodological choices applicable
  to the exact (realistic but intractable) model. Furthermore, to the
  extent that the exact model is realistic, we may tentatively
  conclude that approximation model investigations have some bearing
  on real data applications.
\end{abstract}
\begin{IEEEkeywords}
  Anomaly detection, Attributed Random Graphs, Random Dot
  Product Graphs.
\end{IEEEkeywords}}
\maketitle
\section{Introduction}

\subsection{Latent Process Model}
The latent process model for time series of attributed graphs was
presented in \cite{lee11}. We summarize here the ideas that are
relevant to our discussion.  Let $G = (V,E)$ be an undirected graph,
equipped with an edge-attribution function $\phi \colon \tbinom{V}{2}
\mapsto \{1,2,\dots, K+1\}$ for some $K$ such that $\phi(e) \leq K$
for $e \in E$ and $\phi(e) \equiv K+1$ for $e \not \in E$. We refer to
$\phi(e)$ as the attribute of edge $e \in E$, and the pair $(G,\phi)$
as an attributed graph. Let $\mathscr{S}$ be the unit simplex in
$\mathbb{R}^{K}$, i.e.,
\begin{equation}
  \mathscr{S} = \{ \xi \in [0,1]^{K}
  \colon \sum_{k = 1}^{K} \xi_k \leq 1 \}.
\end{equation}
A {\em random dot product space} for attributed graphs with vertices
in $[n]$ and edge attributes $[K]$ is a pair $(\mathbf{X},G)$ of
random elements such that
\begin{enumerate}
\item $\mathbf{X} = \{X_v\}_{v = 1}^{n}$ is a collection of
  independent $\mathscr{S}$-valued random vectors.
\item $G$ is a random undirected graph with vertex set $[n]$ such that
  \begin{equation}
    \label{eq:1}
    \mathbb{P}(u \sim v \,|\, \mathbf{X}) = \mathbb{P}(u \sim v \, |
    \, X_u,X_v) = \langle x_u, x_v \rangle
  \end{equation}
  and $\bm{1}(u \sim v \,|\, \mathbf{X})$ and $\bm{1}(u' \sim
  v' \,|\, \mathbf{X})$ are independent whenever $(u,v) \not =
  (u',v')$. Furthermore, we also have 
  \begin{equation*}
    \mathbb{P}(\phi(\{u,v\} = k \, | \, u \sim
    v) = x_{u,k} x_{v,k}. 
  \end{equation*}
\end{enumerate}
We say that a c\'{a}dl\'{a}g process $W \colon [0,\infty) \mapsto
[K+1]^{n}$ induces the sequence of random dot product spaces
$\mathscr{V} = (\mathbf{X}(t), G(t))_{t = 1}^{\infty}$ if
\begin{enumerate}
\item Each $(\mathbf{X}(t), G(t))$ is a random dot product space with
  vertices $[n]$ and edge attributes $[K]$. Furthermore, for each $v
  \in [n]$, $k \in [K]$ and $t \in \mathbb{N}$, we have
  \begin{equation}
    \label{eq:54}
  X_{v,k}(t)  = \int_{t - 1}^{t}{ \mathbf{1}\{W_v(\omega) = k\}\, d\omega}
  \end{equation}
  where we denote $W_{v}(\omega)$ as the $v$-th component of $W(\omega)$ and
  $X_{v,k}$ as the $k$-th component of $X_v$.  
\item  For each $t \in \mathbb{N}$, we have
  \begin{equation}
    \label{eq:2}
    \mathbb{P}(G(t) = g \,|\, \mathscr{F}_{\leq t}) = \mathbb{P}(G(t)
    = g \,|\, \mathbf{X}(t))
  \end{equation}
where $\mathscr{F}_{\leq t}$ is the $\sigma$-field generated by $\{W(s)
  \colon s \leq t\}$.
\end{enumerate}
We will call any pair $(\mathscr{V}, W)$ that satisfies the above
properties a random dot process model. In particular, we are
interested in the pairs $(\mathscr{V}, W)$ possessing the following
properties:
\begin{enumerate}
\item For each $t \in \mathbb{N}$ and vertex $v \in [n]$, there exists
  a matrix $\mathbf{Q}^{(v)}(t)$ such that $W_v$, when restricted to
  the interval $[t, t+1)$, is a stationary, continuous-time Markov
  chain with state space $[K+1]$, intensity matrix
  ${\mathbf{Q}^{(v)}(t)}$, and stationary distribution
  $\pi^{(v)}(t)$. $W$ is thus a piecewise stationary, continuous-time
  Markov chain with state space $[K+1]^{n}$ and intensity matrix
  $\otimes_{v=1}^{n}\mathbf{Q}^{(v)}(t)$.
\item There exists a $t^{*} \in \mathbb{N}$ and a $m < n$ such that,
  for some $\pi_0, \pi_1, \mathbf{Q}_0$ and $\mathbf{Q}_1$ we have
  \begin{enumerate}
  \item for $t < t^{*} - 1$,
    \begin{gather*}
      \pi^{(1)}(t) = \dots = \pi^{(n)}(t) = \pi_0 \\
      \mathbf{Q}^{(1)}(t) = \dots = \mathbf{Q}^{(n)}(t) = \mathbf{Q}_0,
    \end{gather*}
  \item  for $t \geq t^{*} - 1$,
    \begin{gather*}
      \pi^{(1)}(t) = \dots = \pi^{(m)}(t) = \pi_1 \\
      \pi^{(m+1)}(t)  = \dots = \pi^{(n)}(t) = \pi_0 \\
      \mathbf{Q}^{(1)}(t)  = \dots = \mathbf{Q}^{(m)}(t) = \mathbf{Q}_1 \\
      \mathbf{Q}^{(m+1)}(t) = \dots = \mathbf{Q}^{(n)}(t) = \mathbf{Q}_0 
    \end{gather*}
  \end{enumerate}
\end{enumerate}
The above properties characterize a random dot process model with a
change-point phenomena for $(X(t),G(t))$ at time $t = t^{*}$. We chose $\{1,2,\dots,m\}$
as the set of vertices with a change in the stationary distribution
and intensity matrix at time $t = t^{*} - 1$ for ease of
discussion. Permutation of the vertex labels does not affect our
subsequent analysis. We will refer to $(t^{*}, m, \pi_0, \pi_1,
\mathbf{Q}_0, \mathbf{Q}_1)$ as the change parameters.  Because $W$ is
completely determined by the change parameters, we often choose to
omit $W$ and only mention $\mathscr{V}$ when referring to random dot
process models (rdpm) with change-point phenomena. For a rdpm
$\mathscr{V}$ with change parameters $(t^{*}, m, \pi_0, \pi_1,
\mathbf{Q}_0, \mathbf{Q}_1)$, we can construct several approximations
to $\mathscr{V}$. Of particular interests are the following two
approximations.
\begin{definition}
  \label{def:1}
  Let $\mathscr{V}$ be a rdpm with change parameters $(t^{*}, m,
  \pi_0, \pi_1, \mathbf{Q}_0, \mathbf{Q}_1)$. The first order approximation
  $\bar{\mathscr{V}}$ of $\mathscr{V}$ is the sequence $\{(\bar{X}(t),
  \bar{G}(t)\}_{t = 1}^{\infty}$ of {\em independent} random dot product
  spaces such that
 \begin{enumerate}
 \item for $t < t^{*} - 1$,
   \begin{equation}
     \label{eq:5}
     \bar{X}_{v}(t)  \equiv \hat{\pi}_0 \quad \text{for all $v \in [n]$}
   \end{equation}
 \item for $t \geq t^{*} - 1$
   \begin{gather*}
     \bar{X}_{v}(t) \equiv \hat{\pi}_1 \quad \text{for $v \leq m$} \\
     \bar{X}_{v}(t) \equiv \hat{\pi}_0 \quad \text{for $v > m$} 
   \end{gather*}
 \end{enumerate}
 where $\hat{\pi}_0$ and $\hat{\pi}_1$ are sub-probability vectors
 obtained by removing the last coordinate of $\pi_0$ and $\pi_1$. 
\end{definition}
The first approximation yields a sequence of independent random graphs
with independent edges. For $t \leq t^{*} - 1$, $G(t)$ is an
attributed instance of the Erd\"{o}s-Renyi random graphs. For $t \geq
t^{*}$, $G(t)$ is an attributed instance of the kidney-egg model
\cite{rukhin11} (see \figurename~\ref{fig:kidney-egg}), which in
itself is a special instance of the mixed membership stochastic
blockmodel \cite{airoldi08:_mixed}. The unattributed kidney-egg model
can be denoted as $\kappa(n,m,p,q,s)$ with the quantities $p$, $q$,
and $s$ representing
\begin{gather*}
p = \mathbb{P}(u \sim v \, | \, u \leq m, v \leq m) = \langle
\hat{\pi}_1, \hat{\pi}_1 \rangle \\ q =
\mathbb{P}(u \sim v \, | \, u \leq m, v > m) = \langle \hat{\pi}_1,
\hat{\pi}_0 \rangle \\ s = \mathbb{P}(u
\sim v \, | \, u > m, v > m) = \langle \hat{\pi}_0, \hat{\pi}_0
\rangle . 
\end{gather*}
\begin{figure}[!tp]
  \centering
  \includegraphics[width=5cm]{graphics/Fig2-SSP2011-hat1.pdf}
  \caption{An illustration of the kidney-egg model for $G(t^{*})$.}
  \label{fig:kidney-egg}
\end{figure}
 \begin{definition}
  \label{def:2}
  Let $\mathscr{V}$ be a rdpm with change parameters $(t^{*}, m,
  \pi_0, \pi_1, \mathbf{Q}_0, \mathbf{Q}_1)$. Define
  $\mathbf{Z}_0$ and $\mathbf{Z}_1$ by
  \begin{gather}
    \mathbf{Z}_0 = (\mathbf{1}\mathbf{\pi}_0^{T} -
     \mathbf{Q}_0)^{-1}(\mathbf{I} - \mathbf{1}\pi_0^{T}) \\
    \mathbf{Z}_1 = (\mathbf{1}\mathbf{\pi}_1^{T} -
     \mathbf{Q}_1)^{-1}(\mathbf{I} - \mathbf{1}\pi_1^{T}).
  \end{gather}
  $\mathbf{Z}_0$ and $\mathbf{Z}_1$ are the fundamental matrices
  for the continuous-time Markov chain on $[K+1]$ with intensity
  matrix $\mathbf{Q}_0$ and $\mathbf{Q}_1$ (see
  e.g. \cite[p. 55]{asmussen03:_applied_probab_queues}). Let
  $\Sigma_0$ and $\Sigma_1$ be given by
  \begin{gather*}
    \Sigma_0 = \mathrm{diag}(\pi_0) \mathbf{Z}_0 +
    \mathbf{Z}_0^{T}
    \mathrm{diag}(\pi_0) \\
    \Sigma_1 = \mathrm{diag}(\pi_1) \mathbf{Z}_1 +
    \mathbf{Z}_1^{T}
    \mathrm{diag}(\pi_1).
  \end{gather*}
  The second order approximation $\widehat{\mathscr{V}}$ of
  $\mathscr{V}$ is the sequence $\{\widehat{X}(t),
  \widehat{G}(t)\}_{t=1}^{\infty}$ where
  \begin{enumerate}
  \item For each $t$ and each $v \in [n]$, $\widehat{X}_v(t)$ is a
    random vector obtained by truncating a multivariate normal random
    vector $Z_{v}(t)$ to $\mathscr{S}$ with mean and covariance
    matrices given below.
  \item For $t < t^{*} - 1$,
    \begin{equation}
      \mathbb{E}[Z_v(t)] \equiv \hat{\pi}_0, \,\, 
  \mathrm{Var}[Z_v(t)] \equiv \widehat{\Sigma}_0 \quad \text{for all
    $v$}.
    \end{equation}
  \item For $t \geq t^{*} - 1$,  
    \begin{gather*}
      \mathbb{E}[Z_v(t)] \equiv \hat{\pi}_1, \,\, 
\mathrm{Var}[Z_v(t)] \equiv \widehat{\Sigma}_{1} \quad \text{for $v
  \leq m$} \\
  \mathbb{E}[Z_v(t)] \equiv \hat{\pi}_0, \,\, \mathrm{Var}[Z_v(t)] \equiv \widehat{\Sigma}_{0} \quad \text{for $v
        > m$}
    \end{gather*}
    where $\widehat{\Sigma}_0$ and $\widehat{\Sigma}_1$ are the $K
    \times K$ matrices
    obtained by removing the last row and column of $\Sigma_0$ and
    $\Sigma_1$, respectively. 
  \end{enumerate}
\end{definition}
A second-order approximation yields a sequence of independent latent
position graphs \cite{marchette08:_predic,scheinerman10:_model,hoff02:_laten}. 

Suppose that we have a sequence $\{\mathscr{V}^{r} \colon r > 0 \}$ of
rdpm with vertices $[n]$ and attributes $[K]$ where for each $r > 0$,
$\mathscr{V}^{r}$ has change parameters $(t^{*}, m, \pi_0, \pi_1, r
\mathbf{Q}_0, r \mathbf{Q}_1)$. The parameter $r$ can be thought of as
the vertex process rate, i.e., the waiting time decreases
exponentially as $r$ increases. Let us now consider the sequence of
first approximations $\bar{\mathscr{V}}^{r}$ and second approximations
$\widehat{\mathscr{V}}^{r}$ of $\mathscr{V}^{r}$. We note that
$\bar{\mathscr{V}}^{r_1} = \bar{\mathscr{V}}^{r_2}$ for any $r_1, r_2
> 0$. Let us then denote by $\bar{\mathscr{V}}$ the (a.e.) unique
first order approximation of $\mathscr{V}^{r}$ for $r > 0$. If we
denote by $\widehat{\Sigma}_{0}^{(r)}$ and
$\widehat{\Sigma}_{1}^{(r)}$ the matrices $\widehat{\Sigma}_0$ and
$\widehat{\Sigma}_1$ for $\widehat{\mathscr{V}}^{r}$, then
$\widehat{\Sigma}_0^{(r)} = \widehat{\Sigma}_0^{(1)}/r$ and
$\widehat{\Sigma}_{1}^{(r)} = \widehat{\Sigma}_1^{(1)}/r$. Therefore,
as $r \rightarrow \infty$, $\widehat{\Sigma}_0^{(r)} \rightarrow
\bm{0}$ and $\widehat{\Sigma}_1^{(r)} \rightarrow \bm{0}$ and so
$\widehat{\mathscr{V}}^{r} \overset{\mathrm{d}}{\longrightarrow}
\bar{\mathscr{V}}$. This indicates that for sufficiently large $r$,
there is little statistical difference amongst the
$\bar{\mathscr{V}}$, $\mathscr{V}^{r}$ and $\widehat{\mathscr{V}}^{r}$
(\cite[Theorem 2]{lee11}). The behavior of
$\widehat{\mathscr{V}}^{r}$ for ``small'' $r$ is less
clear. $\bar{\mathscr{V}}$ will thus serve as the basis for our
subsequent analysis on graph invariants for attributed random graphs.
\subsection{Change-point detection}
\begin{figure}[htbp]
  \centering
  \includegraphics[width=8cm]{graphics/Fig1-SSP2011.pdf}
  \caption{Notional depiction of the problem of change-point detection
    in a time-series of graphs}
  \label{fig:notional_change_point}
\end{figure}
Let $\mathscr{V}$ be a random dot process model with change parameters
$(t^{*}, m, \pi_0, \pi_1, \mathbf{Q}_0, \mathbf{Q}_1)$. The
change-parameters encapsulate a notion of chatter anomalies, i.e., a
subset of vertices of $\mathscr{V}$ with altered communication
behavior in an otherwise stationary setting as depicted in
Fig.~\ref{fig:notional_change_point}. We are interested in the problem
of testing, for a $t \in \mathbb{N}$, the hypotheses that $t$ is the
change-point of $\mathscr{V}$, namely
\begin{gather*}
  \mathscr{H}_0 \colon t^{*} > t \\
  \mathscr{H}_A \colon t^{*} = t
\end{gather*}
This will be done using the notion of fusion of attributed graph
invariants. The particular invariants of interests are the size
$\mathcal{E}$, number of triangles $\tau$, scan $\Psi$, and max degree
$\Delta$. The scan statistics $\Psi$ are introduced and applied to the
problem of detecting chatter anomalies in
\cite{priebe05:_scan_statis_enron_graph}. The use of various graph
invariants as test statistics was considered in
\cite{pao11:_statis_infer_random_graph} and it was shown there, via
Monte Carlo analysis, that no single invariant is uniformly most
powerful.

Let $(G,\phi)$ be an attributed graph. For a given $\{u,v\}
\in \tbinom{V}{2}$, we define a $\Gamma_{uv} \in \mathbb{R}^{K}$ by
$\Gamma_{uv}(k) = \mathbb{I}\{\phi(\{u,v\}) = k\}$. Thus $\Gamma_{uv}
= \bm{0}$ unless $\{u,v\} \in E$. Under the independent edges
assumption, the $\Gamma_{uv}$ are independent. We consider linear
attribute fusion of graph invariants with parameter $\lambda \in
\mathbb{R}^{K}$ via
%\begin{gather}
%  \label{eq:6}
%  \mathcal{E}_{\lambda}(G) = \sum_{k=1}^{K} \lambda_k \sum_{u,v}
%  \mathbb{I}\{ \phi(uv) = k \} \\
%  \tau_{\lambda}(G) = \sum_{(i,j,k) \in K^{3}\vphantom{\tbinom{V}{3}}}
%  \, \, \sum_{(u,v,w)
%      \in \tbinom{V}{3}} \lambda_{i}\lambda_j \lambda_k h_{ijk}(u,v,w) \\
%  \Delta_{\lambda}(G) = \max_{v \in V} \, \sum_{k = 1}^{K} \lambda_k
%  \sum_{u \in N(v)}{\mathbb{I}\{\phi(\{u,v\}) = k\}} \\
%  \label{eq:3}
%  \Psi_{\lambda}(G) = \max_{v \in V} \sum_{k = 1}^{K}
%  \lambda_k \!\!\sum_{u,w \in N(v)}\!\!\! \mathbb{I}\{\phi(\{u,w\}) = k\} 
%  \end{gather}

\begin{align}
  \label{eq:6}
  \mathcal{E}_{\lambda}(G) &= \sum_{\mathclap{\{u,v\} \in
      \tbinom{V}{2}}} \langle
  \lambda, \Gamma_{uv} \rangle \\
  \tau_{\lambda}(G) &= \sum_{\mathclap{\{u,v,w\} \in \tbinom{V}{3}}} \langle
  \lambda, \Gamma_{uv} \rangle \langle \lambda,
  \Gamma_{uw} \rangle \langle \lambda, \Gamma_{vw}
  \rangle \\ 
\Delta_{\lambda}(G) &= \max_{v \in V} \, \sum_{w \in N(v)} \langle
\lambda, \Gamma_{vw} \rangle \\
\label{eq:3}
  \Psi_{\lambda}(G) &= \max_{v \in V} \sum_{u,w \in N[v]} \langle
  \lambda, \Gamma_{uw} \rangle.
  \end{align}
  where $N(v) = \{u \colon u \sim v\}$ is the set of neighbors of
  $v$ and $N[v] = v \cup N(v)$.  

  Let $\{G(t)\}$ be a time series of graphs. Let $J_{\lambda}(t)$ be a
  statistic for $G(t)$ of the form as in Eq.~\eqref{eq:6} through
  Eq.~\eqref{eq:3}. We define the running average 
  $\bar{J}^{l}_\lambda(t)$ of $J_{\lambda}$ as
\begin{equation}
  \label{eq:4}
 \bar{J}^{l}_{\lambda}(t) = \frac{1}{l}\sum_{s = 1}^{l} J_{\lambda}(t - s) 
\end{equation}
where $l \in \mathbb{N}$ specified the width of the running-average
window. Our main interest is in the normalized fusion statistic
$T_{\lambda}^{l}(t)$ as depicted in Fig.~\ref{fig:temporal}, namely
\begin{equation}
  \label{eq:7}
 T_{\lambda}^{l}(t) = % \begin{cases}
   % J_{\lambda}(t) - J_{\lambda}(t - 1) & \text{if $m = 1$} \\
   \frac{J_{\lambda}(t) -
     \bar{J}_{\lambda}^{l}(t)}{\sqrt{\tfrac{1}{l-1}
       \sum_{s=1}^{l}(J_{\lambda}(t - s) - \bar{J}_{\lambda}^{l}(t))^2}}
%   & \text{if $m \geq 2$}
 %  \end{cases}
\end{equation}
for $l \geq 2$, i.e., we want to find a parameter $\lambda$ such that
the power of the test using $T_{\lambda}^{l}$ is maximized. We note that
$T_{\lambda}^{l}$ is scale invariant in $\lambda$ for all of our graph
invariants. It was noted in \cite{lee11} that the maximum of the power
for scale invariant test statistics is attainable in the set $\{ \lambda \colon \|
\lambda \| = 1 \}$. \\ \\

As we have mentioned earlier, we will assume that the number of
vertices $n$ is large and that the use of the first order
approximation $\bar{\mathscr{V}}$ of $\mathscr{V}$ is appropriate in
our study of the asymptotic theory for these graphs invariants. We
strive to obtain approximations for the power in testing $t^{*} > t$
against $t^{*} = t$ using $T_{\lambda}^{l}(t)$ and from these
approximations, find the $\lambda$ that maximize the power. We now
introduce additional notation that will be used later on in the
paper. First, let $\pi_{00}, \pi_{01}$, and $\pi_{11}$ be
sub-probability vectors whose components are given by
\begin{gather*}
  \pi_{00}(k) = \pi_{0}(k) \pi_{0}(k) \\
  \pi_{01}(k) = \pi_{0}(k) \pi_{1}(k) \\
  \pi_{11}(k) = \pi_{1}(k) \pi_{1}(k)
\end{gather*}
where $k \in [K]$. We also let $\eta_{00}, \eta_{01}$, and
$\eta_{11}$ be matrices of size $K \times K$ defined by 
\begin{gather*}
  \eta_{00} = \mathrm{diag}(\pi_{00}) - \pi_{00} \pi_{00}^{T} \\
  \eta_{01} = \mathrm{diag}(\pi_{01}) - \pi_{01} \pi_{01}^{T} \\
  \eta_{11} = \mathrm{diag}(\pi_{11}) - \pi_{11} \pi_{11}^{T}.
\end{gather*}
% The structure of the remaining sections of the paper is as follows. We
% discussed the power estimates for $\mathcal{E}_{\lambda},
% \tau_{\lambda}, \Delta_{\lambda}$ and $\Psi_{\lambda}$ in
% \S~\ref{sec:power-estim-mathc}
%  through \S~\ref{sec:power-estim-psi_l}. We illustrate our analysis with some
% simulation experiments in \S~\ref{sec:experimental-results}  
\begin{figure}[htbp]
  \centering
  \includegraphics[width=8cm]{graphics/Fig4-SSP2011.pdf}
  \caption{Temporal standardization: when testing for change at time
    $t$, the recent past (graphs $G(t - l), \dots, G(t-1))$ is used to
    standardize the invariants}
  \label{fig:temporal}
\end{figure}
\section{Power estimates}
\subsection{Power estimates for $\mathcal{E}_\lambda$}
\label{sec:power-estim-mathc}
It was shown in \cite{lee11} that if $J_\lambda(t) =
\mathcal{E}_{\lambda}(G(t))$, then $T_{\lambda}^{l}(t)$ follows a
$t$-distribution with $l - 1$ degrees of freedom in the
limit. Specifically, we have the following results
\begin{theorem}
  \label{thm:9}
  Let $t \in \{l+1, \dots, t^{*}\}$ and $\lambda \in
  \mathbb{R}^{K}$. Define the vector $\zeta$ and the matrix $\xi$ by
  \begin{gather*}
    \zeta = \tbinom{m}{2}(\pi_{11} - \pi_{00}) + (n-m)m(\pi_{01} -
    \pi_{00}) \\
    \xi = \tfrac{l+1}{l}\tbinom{n}{2} \eta_{00} +
          \tbinom{m}{2}(\eta_{11} - \eta_{00}) + (n-m)m(\eta_{01} -
          \eta_{00}).
  \end{gather*}
  Define the random variable $\psi^{l}_{\lambda}(t)$ by
  \begin{equation}
    \label{eq:10}
    \psi^{l}_{\lambda}(t) = \begin{cases}
      \sqrt{\frac{l}{l + 1}} T_{\lambda}^{l}(t)& \text{if $t < t^{*}$}
      \\ \sqrt{\frac{\langle \lambda, \tbinom{n}{2} \eta_{00}
            \lambda \rangle}{\langle \lambda, \xi
            \lambda \rangle}} T_{\lambda}^{l}(t) & \text{if $t =
          t^{*}$}.
      \end{cases}
  \end{equation}
As $n \rightarrow \infty$,
  $\psi^{l}_{\lambda}(t)$ converges weakly to the Student
  $t$-distribution with $l-1$ degrees of freedom and non-centrality
  parameter $\mu_{\lambda}$, where
  \begin{equation}
    \label{eq:15}
    \mu_{\lambda} = \begin{cases}
      0 & \text{if $t < t^{*}$} \\
      \frac{\langle \lambda, \zeta \rangle}{\sqrt{\langle \lambda, \xi \lambda \rangle}} & \text{if $t = t^{*}$} 
    \end{cases}
  \end{equation}
\end{theorem}
We are interested in finding the $\lambda$ that will maximize the
power of the test. The power approximation is determined by various
factors but the dominating factor is $\mu_{\lambda}$ (for large $n$). 
The following corollary extends a result in
\cite{lee11} for $K = 2$ to the case
where $K \geq 2$. 
\begin{corollary}
  \label{cor:1}
  Let $\zeta$ and $\xi$ be as defined in Theorem~\ref{thm:9}. Suppose
  that $\xi$ is also positive definite. Let $\nu^{*}$ be the
  normalized eigenvector corresponding to the largest eigenvalue of $ \xi^{-1/2}
  \zeta \zeta^{T} \xi^{-1/2}$, i.e.,
 \begin{equation}
   \label{eq:9}
  \nu^{*} = \argmax_{\nu \colon \| \nu \| = 1}
  \nu^{T} \xi^{-1/2} \zeta \zeta^{T} \xi^{-1/2}
  \nu.
 \end{equation}
 Then $\lambda^{*} = \tfrac{\xi^{-1/2} \nu^{*}}{\|\xi^{-1/2} \nu^{*} \|}$ satisfies
 \begin{gather}
   \label{eq:8}
 \argmax_{ \|\lambda\| = 1}
 \mu_{\lambda} \, \cap \, \{\lambda^{*}, - \lambda^{*}\} \not = \emptyset \\
 \argmin_{ \| \lambda \| = 1} \mu_{\lambda} \, \cap \, \{\lambda^{*}, -
 \lambda^{*}\} \not = \emptyset.
 \end{gather}
\end{corollary}
\subsection{Power estimates for $\tau_{\lambda}$}
\label{sec:power-estim-tau_l}
The limiting distribution for the number of triangles in unattributed
random graphs was considered in
\cite{nowicki88:_subgr_u_statis_method} for the Erd\"{o}s-Renyi and in
\cite{rukhin09:_asymp_analy_various_statis_random_graph_infer} for the
kidney-egg model. We note here the small changes that allow us to
extend the results in
\cite{rukhin09:_asymp_analy_various_statis_random_graph_infer,%
nowicki88:_subgr_u_statis_method} to our attributed graphs model. Let
$Y_{uv} = \langle \lambda, \Gamma_{uv} \rangle$ for $\{u,v\} \in
\tbinom{V}{2}$. We can now write $\tau_{\lambda}(G(t))$ as
\begin{equation}
  \label{eq:45}
  \frac{\tau_{\lambda}(G(t))}{\tbinom{n}{3}} = \tbinom{n}{3}^{-1}
  \sum_{\mathclap{\{u,v,w\} \in \tbinom{V}{3}}} Y_{uv} Y_{uw} Y_{vw};
\end{equation}
$\tau_{\lambda}(G(t))/\tbinom{n}{3}$ is then an
$U$-statistic on $\{Y_{e}\}$, with kernel function $h(Y_{1}, Y_{2},
Y_{3}) = Y_{1} Y_{2} Y_{3}$. By using Hajek's projection method, we
can show that $\tau_{\lambda}(G(t))$ converges to a normal
distribution as $n \rightarrow \infty$. 
\begin{lemma}
  \label{lem:3}
  Let $\tau_{\lambda}^{*}$ be the Hajek's projection of $\tau_{\lambda}$, i.e.,
\begin{equation}
  \label{eq:46}
  \tau_{\lambda}^{*}(G) - \mathbb{E}[\tau_{\lambda}(G)] =
  \sum_{\mathclap{\{u,v\} \in \tbinom{V}{2}}} \, \Bigl(\mathbb{E}[
  \tau_{\lambda}(G) \, \lvert \, Y_{uv}] -
  \mathbb{E}[\tau_{\lambda}(G)]\Bigr).
\end{equation}
For $t < t^{*}$, we have
\begin{align*}
  \mathbb{E}[\tau_{\lambda}^{*}(G(t))] &= \tbinom{n}{3} \langle 
  \lambda, \pi_{00} \rangle^{3} \\
  \sqrt{\mathrm{Var}[\tau_{\lambda}^{*}(G(t))]} &= (n-2) \langle \lambda,
  \pi_{00} \rangle^{2} \sqrt{\tbinom{n}{2} \langle \lambda, \eta_{00}
  \lambda \rangle}.
\end{align*}
For $t = t^{*}$, we have
  \begin{gather*}
    \label{eq:42}
    \begin{split}
    \mathbb{E}[\tau_{\lambda}^{*}(G(t))] = \tbinom{m}{3}
    \langle \lambda, \pi_{11} \rangle^{3} &+
    \tbinom{m}{2}(n-m) \langle \lambda, \pi_{11} \rangle \langle
    \lambda, \pi_{01} \rangle^{2} \\ &+ m \tbinom{n-m}{2} \langle \lambda,
    \pi_{01} \rangle^{2} \langle \lambda, \pi_{00} \rangle \\ &+
    \tbinom{n-m}{3} \langle \lambda, \pi_{00} \rangle^{3}
    \end{split} \\
    \begin{split}
    \mathrm{Var}[\tau_{\lambda}^{*}(G(t))] = \tbinom{m}{2} \langle \lambda, \eta_{11} \lambda
    \rangle S_1^2  
    &+ m(n-m) \langle \lambda, \eta_{01} \lambda \rangle S_2^2  \\
    &+ \tbinom{n-m}{2} \langle \lambda, \eta_{00} \lambda \rangle
    S_3^2
    \end{split}
  \end{gather*}
  where $S_1$, $S_2$, and $S_3$ are given by
\begin{align*}
    S_1 &= (m-2) \langle \lambda, \pi_{11} \rangle^{2} + (n-m)
    \langle \lambda, \pi_{01} \rangle^{2} \\ 
    S_2 &= (m-1)
    \langle \lambda, \pi_{11} \rangle \langle \lambda, \pi_{01}
    \rangle + (n-m-1) \langle \lambda, \pi_{00} \rangle \langle
    \lambda, \pi_{01} \rangle \\
    S_3 &= m \langle \lambda, \pi_{01} \rangle^{2} + (n-m-2) \langle
    \lambda, \pi_{00} \rangle^{2}.
    \end{align*}
  As $n \rightarrow \infty$, we have
  \begin{equation}
    \label{eq:49}
    \frac{\tau_{\lambda}(t) -
      \mathbb{E}[\tau_{\lambda}^{*}(G(t))]}{\sqrt{\mathrm{Var}[\tau_{\lambda}^{*}(G(t))]}}
\overset{\mathrm{d}}{\longrightarrow}  \mathcal{N}(0,1).
  \end{equation}
\end{lemma}
From Lemma~\ref{lem:3}, we can show that the limiting distribution of
$T_{\lambda}^{l}(t)$ is once again a Student t-distribution with
$l-1$ degrees of freedom.
\begin{theorem}
  \label{thm:5}
  Let $t \in \{l+1, \dots, t^{*}\}$ and $\lambda \in
  \mathbb{R}^{K}$. Let $\mu_0 = \mathbb{E}[\tau_{\lambda}^{*}(G(t))]$
  and $\sigma_0^{2} = \mathrm{Var}[\tau_{\lambda}^{*}(G(t))]$ for $t <
  t^{*}$. Let $\mu_A = \mathbb{E}[\tau_{\lambda}^{*}(G(t^{*}))]$ and
  $\sigma_{A}^{2} = \mathrm{Var}[\tau_{\lambda}^{*}(G(t^{*}))]$. Define
  the random variable $\psi_{\lambda}^{l}(t)$ by
  \begin{equation}
    \label{eq:50}
    \psi_{\lambda}^{l}(t) = \begin{cases}
      \sqrt{\tfrac{l}{l+1}}\, T_{\lambda}^{l}(t) & \text{if $t <
        t^{*}$} \\
      \sqrt{\tfrac{l \sigma_0^{2}}{l \sigma_A^{2} + \sigma_0^{2}} }\,
      T_{\lambda}^{l}(t) & \text{if $t = t^{*}$}.
        \end{cases}
  \end{equation}
  As $n \rightarrow \infty$, $\psi_{\lambda}^{l}(t)$ converges to a Student $t$-distribution with $l - 1$ degrees of freedom and
  non-centrality parameter $\mu_{\lambda}$ where
  \begin{equation}
    \label{eq:51}
    \mu_{\lambda} = \begin{cases} 0 & \text{if $t < t^{*}$} \\
      \frac{\mu_A - \mu_0}{\sqrt{l \sigma_A^{2} + \sigma_0^{2}}} &
       \text{if $t = t^{*}$}.
      \end{cases}
  \end{equation}
\end{theorem}
The power approximation for $\tau_{\lambda}$ is determined by various
factors, but similar to the power approximation for
$\mathcal{E}_{\lambda}$, the dominating factor is $\mu_\lambda$ (for
large $n$). Thus, we are also interested in finding the $\lambda$
that will maximize $\mu_\lambda$. However, because of the high power
of $\lambda$ in the expression for $\mu_{\lambda}$, an exact 
solution to $\argmax_{\lambda}{\mu_{\lambda}}$ is more challenging for
$\tau$ than for $\mathcal{E}$. 
\subsection{Power estimates for $\Delta_{\lambda}(t)$}
\label{sec:power-estim-delt}
The limiting distribution for maximum degree in unattributed random
graphs was considered in \cite{bollobas85:_random_graph} for the
Erd\"{o}s-Renyi and in \cite{rukhin11} for the kidney-egg model. We
note here the necessary changes that allow us to to extend the results
in \cite{bollobas85:_random_graph,rukhin11} to our attributed graphs
model. We denote by $\mathcal{G}(\alpha,\beta)$ the Gumbel
distribution with location parameter $\alpha$ and scale parameter
$\beta$.
\begin{proposition}
  \label{prop:3}
  Let $a_n$ and $b_n$ be functions of $n$ given by
  \begin{align*}
    a_n &= (2 \log{n})^{1/2}\Bigl(1 - \frac{\log{\log{n}} + \log{4\pi}}{4 \log{n}} \Bigr) \\ 
    b_n &= (2 \log{n})^{-1/2}.
  \end{align*}
  Then as $n \rightarrow \infty$ and $m = \Omega( \sqrt{n \log{n}})$,
  we have
  \begin{gather}
    \label{eq:14}
    \frac{\Delta_{\lambda}(t) - a_{n} \sigma_0 - \mu_0}{b_n \sigma_0}
   \overset{\mathrm{d}}{\longrightarrow}  \mathcal{G}(0,1) \quad \text{for $t < t^{*}$ } \\
    \frac{\Delta_{\lambda}(t) - a_{m} \sigma_A - \mu_A}{b_m \sigma_A}
       \overset{\mathrm{d}}{\longrightarrow}\mathcal{G}(0,1) \quad \text{for $t = t^{*}$ }
  \end{gather}
  where
  \begin{align*}
    \mu_0 &= (n-1)\langle \lambda, \pi_{00} \rangle \\
    \sigma_A &= \sqrt{(n-1)\langle \lambda, \eta_{00} \lambda \rangle} \\
    \mu_0 &= \,\, (m - 1) \langle \lambda, \pi_{11} \rangle + (n-
    m)\langle \lambda, \pi_{01} \rangle
    \\ \sigma_A &= \sqrt{ (m - 1) \langle \lambda, \eta_{11} \lambda \rangle + (n -
      m) \langle \lambda, \eta_{01} \lambda \rangle}.
    \end{align*}
\end{proposition}
$T_{\lambda}^{l}(t)$ based on the Gumbel distributed
$\Delta_{\lambda}(t)$, in contrast to $T_{\lambda}^{l}(t)$ based on
the normally distributed $\mathcal{E}_{\lambda}$ and
$\tau_{\lambda}(t)$, does not have a simple distribution for small or
moderate values of $l$. The following result give the limiting
distribution for $T_{\lambda}^{l}(t)$ under the assumption that $l$ is
sufficiently large. 
\begin{theorem}
  \label{thm:8}
  Let $t \in \{l+1, \dots, t^{*}\}$ and $\lambda \in
  \mathbb{R}^{K}$. For sufficiently large $n$ and sufficiently large
  $l$, the variable $T_{\lambda}^{l}(t)$ has approximately a
  $\mathcal{G}(\rho_{\lambda}, \varsigma_{\lambda})$ distribution with
\begin{align}
  \label{eq:13}
  \rho_{\lambda} &= \begin{cases}
      - \frac{ \sqrt{\pi}}{6} \gamma & \text{if $t < t^{*}$} \\
    \frac{\sqrt{\pi}}{6} \frac{\mu_A - \mu_0 + a_m\sigma_A - (a_n + b_n
      \gamma)\sigma_0}{b_n \sigma_0} & \text{if $t =
        t^{*}$} 
  \end{cases}\\
  \varsigma_{\lambda} &= \begin{cases}
    \frac{\sqrt{\pi}}{6} & \text{if $t < t^{*}$} \\
    \frac{\sqrt{\pi} b_m \sigma_A}{6 b_n \sigma_0} & \text{if $t =
      t^{*}$} \\
    \end{cases}
\end{align}
where $\gamma \approx 0.57721$ is the Euler-Mascheroni constant.
\end{theorem}
We are interested in finding the $\lambda$ that will maximize the
power of the test using $\Delta_{\lambda}$. The power approximation is
determined by various factors, but the dominating factor is
$\tfrac{\rho_\lambda}{\varsigma_\lambda}$. For sufficiently large $n$
and $m$, we have for $t = t^{*}$,
\begin{equation*}
  \begin{split}
  \frac{\rho_\lambda}{\varsigma_\lambda} &= (1 + o(1)) \frac{\mu_A - \mu_0}{b_m
    \sigma_A} \\ &= (1 + o(1)) \frac{\langle \lambda, (m-1)\pi_{11} + (n-m)\pi_{01} -
    (n-1) \pi_{00} \rangle}{b_m \sqrt{\langle \lambda, ((m-1) \eta_{11} +
    (n-m)\eta_{01}) \lambda \rangle}}.
  \end{split}
\end{equation*}
We can thus find the maximum and minimum of
$\tfrac{\rho_{\lambda}}{\varsigma_{\lambda}}$ by solving an 
eigenvalue problem as in Corollary~\ref{cor:1}.
\subsection{Power estimates for $\Psi_{\lambda}(t)$}
\label{sec:power-estim-psi_l}
The limiting distribution for the scan statistics in unattributed
random graphs was considered in
\cite{rukhin:_limit_distr_graph_scan_statis}. We note here the changes
that allow us to extend the results in
\cite{rukhin:_limit_distr_graph_scan_statis} to our attributed graphs
model. Let $z_n$ be defined, for $n \in \mathbb{N}$, by
\begin{equation}
  \label{eq:20}
   z_n = \sqrt{2
      \log{n}}\Bigl(1 - \frac{\log{\log{n}} + \log{4 \pi}}{4
      \log{n}}\Bigr).
\end{equation}
Let us also define $E$ and $F$ by
\begin{gather*}
 %  B \sim \mathrm{Bin}(m, \langle 1, \pi_{00} \rangle); \quad C \sim \mathrm{Bin}(n-m-1,
 % \langle 1, \pi_{01} \rangle) \\
  E \sim \mathrm{Bin}(m-1, \langle 1, \pi_{11} \rangle) \\ F \sim \mathrm{Bin}(n-m,
  \langle 1, \pi_{01} \rangle).
\end{gather*}%
Let $E + F$ be the convolution of $E$ and $F$. Denote by $\mu_{E+F}$ and
$\sigma_{E+F}^{2}$ the mean and variance of $E + F$. Let
$N_{\kappa} = \mu_{E + F} + z_m \sigma_{E + F}$. We then have the
following results.
\begin{lemma}
  \label{lem:5}
  Let $\Psi_{\lambda}(t)$ be the scan statistic for $t < t^{*}$. Let
  $N_{0}$, $a_n$, and $b_n$ be defined by
\begin{align*}
   \mu_0 &= (n-1) \langle 1, \pi_{00} \rangle \\
   \sigma_0 &= \sqrt{(n-1) (\langle 1, \pi_{00} \rangle - \langle
     1, \pi_{00} \rangle^2)} \\ 
    N_{0} &= \mu_0 + z_n \sigma_0 \\
  %  a_n &= N_{0} \frac{\langle \lambda, \pi_{00} \rangle}{\langle
  %    1, \pi_{00} \rangle} + C_{00}p_{00}\tbinom{N_{0}}{2} \\ 
    a_n &=  \langle \lambda, \pi_{00} \rangle \tbinom{N_{0}}{2} \\ 
   b_n &= \langle \lambda, \pi_{00} \rangle N_{0} \frac{\sigma_0}{\sqrt{2 \log
     n}}.
%   b_n &= (1 - \tfrac{C_{00}p_{00}}{2} + C_{00}p_{00} N_{0}) \frac{\sigma_0}{\sqrt{2 \log
%     n}}
\end{align*}
Then we have
\begin{equation}
  \frac{\Psi_{\lambda}(t) - a_{n}}{b_n}
  \overset{\mathrm{d}}{\longrightarrow}  \mathcal{G}(0, 1).
\end{equation}
\end{lemma}%
\begin{lemma}
  \label{lem:6}
  Let $a_{n,m}$ and $b_{n,m}$ be given by
  \begin{gather}
    \label{eq:28}
    \begin{split}
    a_{n,m} =%  N_{\kappa} \frac{\langle x, \pi_{01} \rangle}{\langle
 %      1, \pi_{01} \rangle} \\
 % & + 
\langle \lambda, \pi_{00} \rangle \tbinom{N_\kappa}{2} &+
\langle \lambda, \pi_{11} - \pi_{00} \rangle
\tbinom{\mu_E}{2} \\ &+ \langle \lambda, \pi_{01} - \pi_{00} \rangle \mu_E \mu_F 
    \end{split} \\
    b_{n,m} = \langle \lambda, \pi_{00} \rangle N_\kappa \frac{\sigma_{E + F}}{\sqrt{2
        \log{m}}}.
    %b_{n,m} = (1 - \tfrac{C_{00} p_{00}}{2} + C_{00} p_{00}
    %N_\kappa)\frac{\sigma_{E + F}}{\sqrt{2 \log{n_1}}}
  \end{gather}
  If $m = \Omega(\sqrt{n \log n})$ and $m = O(n^{k/(k+1)})$ for some
  $k \in \mathbb{N}$ then
  \begin{equation}
    \label{eq:29}
    \frac{\Psi_{\lambda}(t^{*}) - a_{n,m}}{b_{n,m}}
    \overset{\mathrm{d}}{\longrightarrow} \mathcal{G}(0,1).
  \end{equation}
\end{lemma}
The next result is analogous to Theorem \ref{thm:8} and gives 
the limiting distribution for
$T_{\lambda}^{l}(t)$ based on $\Psi_{\lambda}$ for the case of large
$l$.
\begin{theorem}
  \label{thm:6}
  Let $t \in \{l+1, \dots, t^{*}\} $ and $\lambda \in
  \mathbb{R}^{K}$. For sufficiently large $n$ and sufficiently large
  $l$, $T_{\lambda}^{l}(t)$ has approximately a
  $\mathcal{G}(\rho_{\lambda}, \varsigma_{l})$ distribution with
  \begin{align}
    \label{eq:52}
    \rho_{\lambda} &= \begin{cases}
      - \tfrac{\sqrt{\pi}}{6} \gamma & \text{if $t < t^{*}$} \\
      \tfrac{\sqrt{\pi}}{6} \tfrac{a_{n,m} - a_n - b_n \gamma}{b_n} & \text{if
        $t = t^{*}$} 
    \end{cases} \\
      \varsigma_{\lambda} &= \begin{cases}
        \tfrac{\sqrt{\pi}}{6} & \text{if $t < t^{*}$} \\
        \tfrac{\sqrt{\pi} b_{n,m}}{6 b_n} & \text{if $t = t^{*}$}. \\
      \end{cases}
  \end{align}
\end{theorem}
The dominating factor in the power approximation for $\Psi_{\lambda}$
is $\tfrac{\rho_{\lambda}}{\varsigma_{\lambda}}$. For $t = t^{*}$ and
sufficiently large $n$ and $l$, we have
\begin{equation}
  \label{eq:53}
  \begin{split}
    \frac{\rho_{\lambda}}{\varsigma_{l}} &= \frac{a_{n,m} - a_n - b_n
      \gamma}{b_{n,m}} \\
    &= (1 + o(1)) \frac{ \langle \lambda, \xi \rangle}{\langle
      \lambda, N_{\kappa} \tfrac{\sigma_{E+F}}{\sqrt{2 \log{n}}}
      \pi_{00} \rangle}
  \end{split}
\end{equation}
where $\xi$ is given by
\begin{equation*}
 \xi =  \Bigl(\tbinom{N_{\kappa}}{2} -
    \tbinom{N_0}{2} - \tbinom{\mu_E}{2} - \mu_E \mu_F\Bigr)\pi_{00} +
    \tbinom{\mu_{E}}{2} \pi_{11} + \mu_E \mu_F \pi_{10}.
\end{equation*}
From Eq.~\eqref{eq:53}, we see that there exists (as $n \rightarrow
\infty$) a $\lambda$ that maximizes
$\tfrac{\rho_\lambda}{\varsigma_{\lambda}}$ and is on the boundary,
i.e., $\lambda_k \not = 0$ for exactly one $k$. Therefore the asymptotic theory
for scan statistics, in contrast with the other graph invariants that
were considered, indicates that there may be no benefits in fusing
attributes. However, as \figurename~\ref{fig5:subfig_scan} shows, this
phenomenon does not hold in general for moderate values of
$n$. Similar observations about potential inaccuracies in using
asymptotic results to predict finite but large samples behavior
in testing using graph invariants were discussed in
\cite{rukhin11,priebe10:_you_i}.   
%\section{Experimental Results}
%\label{sec:experimental-results}
\section{Inference Example}
We present here an inference example with the following
parameters
\begin{equation*}
  K = 2, n = 100, m = 9, l = 10,
\end{equation*}
transition matrices
\begin{equation*}
  \label{eq:55}
  \begin{matrix}
  \mathbf{Q}_0 = \begin{bmatrix}
    -\tfrac{2}{3} & \tfrac{1}{6} & \tfrac{1}{2} \\
    1 & -1 & 0 \\
    -1 & 0 & -1 
  \end{bmatrix},
  & \mathbf{Q}_1 = \begin{bmatrix}
    -\tfrac{13}{7} & \tfrac{5}{7} & \tfrac{8}{7} \\
    1 & -1 & 0 \\
    1 & 0 & -1
  \end{bmatrix}
  \end{matrix}
\end{equation*}
and stationary probability vectors
\begin{equation*}
  \pi_0 = (0.10,0.30,0.60)^{T}, \quad \pi_1 = (0.25,0.40,0.35)^{T}
\end{equation*}
Power estimates for our attribute fusion statistics for this example
are presented in \figurename~\ref{fig:power-scale} and
\figurename~\ref{fig:power-estimate}. We consider the asymptotic,
first approximation, second approximation, and exact model power
estimates. \figurename~\ref{fig:power-scale} shows power as a function
of the vertex process parameter $r$ at a specific point, namely at
$\lambda^{*} = \argmax_{\lambda} \beta(\lambda)$.
\figurename~\ref{fig:power-estimate} shows power as a function of
angle $\theta$, where $\lambda = (cos(\theta), sin(\theta))$.

The main implication that can
be inferred from \figurename~\ref{fig:power-scale} and
\figurename~\ref{fig:power-estimate}
is that inferential performance in the
mathematically tractable first-order and second-order approximation
models does provide guidance for methodological choices applicable to
the exact (realistic but intractable) model. Furthermore, to the
extent that the exact model is realistic, we may tentatively conclude
that approximation model investigations have some bearing on real
data applications.
\begin{figure*}[!t]
  \centering
  \subfloat[$\mathcal{E}_{\lambda}$]{
    \includegraphics[width=7cm]{graphics/pwr-scale-size.pdf}
  }
  \hfil
  \subfloat[$\Delta_{\lambda}$]{
    \includegraphics[width=7cm]{graphics/pwr-scale-maxd.pdf}
  }
  \hfil
  \subfloat[$\tau_{\lambda}$]{
    \includegraphics[width=7cm]{graphics/pwr-scale-scan.pdf}
  }
  \hfil
  \subfloat[$\Psi_{\lambda}$]{
    \includegraphics[width=7cm]{graphics/pwr-scale-ntri.pdf}
  }
  \caption{Power $\beta(\theta^{*}_r,r)$ at $\theta_{r}^{*} =
    \argmax_{\theta} \beta(\theta,r)$ as a function of $r$ for the
    four invariants at test size $\alpha = 0.05$. The horizontal lines
    represents first-order approximation $\pm$ three
    standard deviations, and the two curves represent the second
    approximation (green) and exact model (blue). The 10000 Monte
    Carlo replicates yields standard deviations not exceeding $0.005$ for the power
    estimates . The second approximation results match well with the exact model results, and
    both match well for large $r$ with the first-order approximation results.
  }
  \label{fig:power-scale}
\end{figure*}
\begin{figure*}[!t]
  \centering
  \subfloat[$\mathcal{E}_{\lambda}$]{
    \includegraphics[width=7cm]{graphics/pwr-size.pdf}
  }
  \hfil
  \subfloat[$\Delta_{\lambda}$]{
    \includegraphics[width=7cm]{graphics/pwr-maxd.pdf}
  }
  \hfil
  \subfloat[$\Psi_{\lambda}$]{
    \includegraphics[width=7cm]{graphics/pwr-scan.pdf}
    \label{fig5:subfig_scan}
  }
  \hfil
  \subfloat[$\tau_{\lambda}$]{
    \includegraphics[width=7cm]{graphics/pwr-ntri.pdf}
  }
  \caption{Power $\beta$ as a function of angle $\theta$ for $\lambda
    = (cos(\theta),sin(\theta))$. The plot shows analytic asymptotic
    result (black), and first approximation (red), second
    approximation (green), and exact model (blue) estimates for the
    various invariants via Monte Carlo. The 10000 Monte Carlo
    replicates yields standard deviations not exceeding $0.005$ for
    the power estimates. The four vertical lines in each plot
    correspond to $\theta \in \{0, \pi/2, \pi/4,
    \theta^{*}\}$. Because we are considering a one-sided test and
    because $\pi_{1,k} > \pi_{0,k}$ for both $k = 1$ and $k = 2$, the
    power is maximized in the first quadrant, i.e., $\theta \in (0,
    \pi/2)$. The second approximation results match well
    with the exact model results, and both match well for large $r$
    with the first order approximation and the asymptotic results. The
    optimal $\theta^{*}$ is apparently different for the four different graph
    invariants.  
  }
  \label{fig:power-estimate}
  \end{figure*}
\section{Conclusions}
We have presented an analysis of change-point detection for time
series of attributed graphs through the use of test statistics which
are based on linear attribute fusion of some graph invariants. 
We derived the limiting distribution of these test statistics under
the assumptions that $n$, the number of vertices, is sufficiently
large and $r$, the process parameter rate, is also sufficiently
large. The limiting distribution for these test statistics are then
used to derive estimates for the power of the tests. 

The experiment in \S \ref{eq:55} indicates that the power estimates
are accurate, even for the moderate value of $n = 100$. Furthermore,
it was also indicated that the optimal linear fusion parameter depends
on the graph invariant considered. In particular, the results depicted
in \figurename~\ref{fig:power-estimate} yield
$\theta^{*}_{\mathcal{E}_{\lambda}} \approx 0.24$,
$\theta^{*}_{\Delta_{\lambda}} \approx 0.32$,
$\theta^{*}_{\Psi_{\lambda}} \approx 0.14$ and
$\theta^{*}_{\tau_{\lambda}} \approx 0.38$. These optimal fusion
parameter differences are statistically significant, and combining
this result with the ``no uniformly most powerful invariant'' result
\cite{pao11:_statis_infer_random_graph}, we conclude that optimal
linear attribute fusion theory requires significant additional
development. Toward this end, the approximation models from
\cite{lee11} promise to be of assistance.

Hypothesis testing on time series of attributed graphs has
applications in diverse areas, e.g., social network analysis (wherein
vertices represent individual actors or organizations), connectome
inference (wherein vertices are neurons or brain regions) and text
processing (wherein vertices represent authors or documents). These
and many other applications may benefit from generalizations of our
results and the model in \cite{lee11} to directed, multi, and weighted
graphs, as well as the consideration of inference with errorful edge
attributes through an attribute confusion matrix.
% \appendix[Proofs of some stated results]
% \label{sec:proofs-some-stated}
% \begin{IEEEproof}[Corollary~\ref{cor:1}]
%   The maximizer of $\mu_\lambda$ also maximizes
%   \begin{equation*}
%     \mu_{\lambda}^{2} = \frac{\lambda^{T} \zeta \zeta^{T}
%       \lambda}{\lambda^{T} \xi \lambda}
%   \end{equation*}
%   Because $\xi$ is positive definite, there exists a positive definite matrix
%   $\xi^{1/2}$ such that $\xi^{1/2} \xi^{1/2} = \xi$. Letting $\nu = \xi^{1/2}
%   \lambda$, the above expression can be rewritten as
%   \begin{equation*}
%     \mu_{\lambda}^{2} = \frac{\nu^{T} \xi^{-1/2} \zeta \zeta^{T}
%       \xi^{-1/2} \nu}{ \nu^{T} \nu}
%   \end{equation*}
%   The claim then follows directly from the Rayleigh-Ritz theorem for
%   Hermitian matrices.
% \end{IEEEproof}
% \begin{IEEEproof}[Lemma~\ref{lem:3}]
%   $\tau_{\lambda}(G)$ is a U-statistics with kernel function
%   $h(Y_1, Y_2, Y_3) = Y_1 Y_2 Y_3$. By the theory of U-statistics, we
%   know that
%   \begin{equation}
%     \label{eq:48}
%     \frac{\tau_{\lambda}(G) -
%       \mathbb{E}[\tau_{\lambda}^{*}(G)]}{\sqrt{\mathrm{Var}[\tau_{\lambda}^{*}(G)]}}
%        \overset{\mathrm{d}}{\longrightarrow}  N(0,1)
%   \end{equation}
%   provided that $\mathrm{Var}[\tau_{\lambda}(G) -
%   \tau_{\lambda}^{*}(G)] = o(\mathrm{Var}[\tau_{\lambda}^{*}(G)])$.
  
%   By the independent edge assumption, we have
%   \begin{align}
%     \mathbb{E}[h(Y_i, Y_j, Y_k) &= \mathbb{E}[Y_i]
%   \mathbb{E}[Y_j] \mathbb{E}[Y_k] \\ 
%   \mathbb{E}[h(Y_i, Y_j, Y_k) |
%   Y_i] &= Y_i \mathbb{E}[Y_j] \mathbb{E}[Y_k].
%   \end{align}
%  Thus, for $t < t^{*}$, we have $\mathbb{E}[\tau_{\lambda}^{*}(G(t))] = \tbinom{n}{3} \langle
%   \lambda, \pi_{00} \rangle^{3}$ and
%   \begin{equation}
%     \begin{split}
%       \mathrm{Var}[\tau_{\lambda}^{*}(G(t))] &=
%       \mathrm{Var}\Bigl[\sum_{\{u,v,w\}} Y_{uv} \mathbb{E}[Y_{uw}]
%       \mathbb{E}[Y_{vw}]\Bigr] \\
%       &= (n-2)^{2} \langle \lambda, \pi_{00} \rangle^{4}
%       \mathrm{Var}\Bigl[\sum_{\{u,v\}} Y_{uv} \Bigr] \\
%       &= (n-2)^{2} \langle \lambda, \pi_{00} \rangle^{4} \tbinom{n}{2}
%       \langle \lambda, \eta_{00} \lambda \rangle.
%     \end{split}
%   \end{equation}
%   We now sketch the derivation of
%   $\mathrm{Var}[\tau_{\lambda}^{*}(G(t))]$ for $t = t^{*}$. We partition the set
%   $\{u,v\} \in \tbinom{V}{2}$ into the sets 
% \begin{gather*}
% \mathcal{S}_1 = \{ u,v \in [m]
%   \}, \\ \mathcal{S}_2 = \{ u \in [m], v \in [n] \setminus [m]\}, \\
%   \mathcal{S}_3 = \{ u, v \in [n] \setminus[m]\}.
% \end{gather*} 
% We can thus decompose $\mathrm{Var}[\tau_{\lambda}^{*}(G(t))]$ as 
% \begin{equation}
%   \begin{split}
%   \mathrm{Var}[\tau_{\lambda}^{*}(G(t))] &= S_1^{2} \mathrm{Var}[\sum_{\{u,v\} \in \mathcal{S}_1} Y_{uv}] +
%   S_2^{2}
%   \mathrm{Var}[\sum_{\{u,v\} \in \mathcal{S}_2} Y_{uv}] \\ &+
%   S_3^{2}
%   \mathrm{Var}[\sum_{\{u,v\} \in \mathcal{S}_3} Y_{uv}] 
%   \end{split}
% \end{equation}
% Now, for $\{u,v\} \in \mathcal{S}_1$, we have
% \begin{equation}
%   \label{eq:58}
%   \begin{split}
% S_1 Y_{uv} &=\sum_{w \not = u,v} \mathbb{E}[h(Y_{uv}, Y_{uw}, Y_{vw})
% \, | \, Y_{uv}] \\
%  &= ((m-2) \langle \lambda, \pi_{11} \rangle^{2} + (n-m)
%     \langle \lambda, \pi_{10} \rangle^{2})Y_{uv}.
%   \end{split}
% \end{equation}
% The above expression is reasoned as follows. If $w \in [m]$, then
% $\mathbb{E}[Y_{uw}] = \mathbb{E}[Y_{vw}] = \langle \lambda,
% \pi_{11} \rangle$ and there are $m-2$ possible choices for $w \in [m]$
% different from $u$ and $v$. If $w \in [n] \setminus [m]$, then
% $\mathbb{E}[Y_{vw}] = \mathbb{E}[Y_{uw}] = \langle \lambda, \pi_{10}
% \rangle$ and there are $n - m$ possible choices for $w$. Analogous
% reasoning gives the expressions for $S_2$ and $S_3$ in the statement
% of the lemma. 

% We also have
%  \begin{equation}
%    \label{eq:56}
%    \mathrm{Var}[Y_{uv}] = \begin{cases}
%      \langle \lambda, \eta_{00} \lambda \rangle & \text{if $\{u,v\} \in
%        \mathcal{S}_1$} \\
%      \langle \lambda, \eta_{01} \lambda \rangle & \text{if $\{u,v\} \in
%        \mathcal{S}_2$} \\
%      \langle \lambda, \eta_{11} \lambda \rangle & \text{if $\{u,v\} \in
%        \mathcal{S}_3$} \\
%      \end{cases}.
%  \end{equation}
% and thus
% \begin{equation*}
%   \begin{split}
%     \mathrm{Var}[\tau_{\lambda}^{*}(G(t))] &=
%     \tbinom{m}{2} \langle \lambda, \eta_{00} \lambda \rangle S_1^{2} +
%     m(n-m) \langle \lambda, \eta_{01} \lambda \rangle S_2^{2} \\ &+
%     \tbinom{n-m}{2} \langle \lambda, \eta_{00} \lambda \rangle S_3^{2}
%   \end{split}
% \end{equation*}
% as desired. To complete the proof one must show that
% $\mathrm{Var}[\tau_{\lambda}(G) - \tau_{\lambda}^{*}(G)] =
% o(\mathrm{Var}[\tau_{\lambda}^{*}(G)])$ and this follows directly from
% the argument in \cite{nowicki88:_subgr_u_statis_method}
% or \cite{rukhin09:_asymp_analy_various_statis_random_graph_infer}.
% \end{IEEEproof}
% \begin{IEEEproof}[Proposition~\ref{prop:3}]
%   Let $v \in V(t)$ and denote by $d_{\lambda}(v;t)$ the (fused) degree
%   of vertex $v$, i.e.,
%   \begin{equation*}
%     d_{\lambda}(v;t) = \sum_{w \in N(v)} \langle \lambda,
%     \Gamma_{vw} \rangle.
%   \end{equation*}
%   For $t < t^{*}$, each of the $\Gamma_{vw}$ is a multinomial
%   trial with probability vector $\pi_{00}$. The following statements are
%   made as $n \rightarrow \infty$ for fixed
%   $K$.  By the central limit theorem, we have
%   \begin{equation}
%     \label{eq:17}
%     \frac{d_{\lambda}(v;t) - (n-1) \langle \lambda, \pi_{00}
%       \rangle}{\sqrt{(n-1) \langle \lambda, \eta_{00} \lambda \rangle}}
%     \overset{\mathrm{d}}{\longrightarrow} \mathcal{N}(0, 1).
%   \end{equation}
%   We can thus consider the degree sequence of $G(t)$ for $t < t^{*}$
%   as a sequence of {\em dependent} normally distributed random
%   variables. By an argument analogous to the argument for
%   Erd\"{o}s-Renyi random graphs in \cite[\S
%   III.1]{bollobas85:_random_graph} we can show that the dependency
%   among the $\{d_{\lambda}(v;t)\}_{v \in V(t)}$ can be
%   ignored. Another way of doing this is to note that the covariance
%   between $X_u$ and $X_v$, where $X_u$ and $X_v$ are the ratio in
%   Eq.~\eqref{eq:17} for vertices $u$ and $v$, is given by
%   \begin{equation}
%     \label{eq:63}
%     r = \mathrm{Cov}(X_u,X_v) = 
%     \frac{3 \langle \lambda, \pi_{00} \rangle}{\sqrt{(n-1)
%         \langle \lambda, \eta_{00} \lambda \rangle}}.
%   \end{equation}
%   Because $r \log{n} \rightarrow 0$ as $n \rightarrow \infty$, the
%   sample maximum of the $X_u$ converges to the sample maximum of a
%   sequence of {\em independent} $\mathcal{N}(0,1)$ random variables.
%   $d_{\lambda}(v;t)$, can thus be considered as a sequence
%   of independent random variables from a normal distribution. It is
%   well known that the sample maximum of standard normal random
%   variables converges weakly to a Gumbel distribution \cite[\S
%   2.3]{galambos87:_asymp_theor_extrem_order_statis}. It is, however,
%   not clear whether the convergence of $\Delta_{\lambda}(t)$ to a
%   Gumbel distribution continues to hold under the composition of
%   weak convergence as outlined above. We avoid this problem by
%   showing directly that
%   \begin{equation}
%     \label{eq:22}
%     \mathbb{P}\Bigl(\tfrac{\Delta_{\lambda}(t) - (n-1)
%       \langle \lambda, \pi_{00} \rangle}{\sqrt{(n-1) \langle
%         \lambda, \eta_{00} \lambda \rangle}} \leq a_n + b_n x\Big)
%     \rightarrow 
%     e^{-e^{-x}}. 
%   \end{equation}
%   Let $\zeta_v = \tfrac{d_{\lambda}(v;t) - (n-1) \langle \lambda,
%     \pi_{00} \rangle}{\sqrt{(n-1) \langle \lambda, \eta_{00} \lambda
%       \rangle}}$ and $F_n(u) = \mathbb{P}(\zeta_v \leq u)$. If $n
%   \rightarrow \infty$ and $u = O(\sqrt{\log{n}})$, we have the
%   following moderate deviations result \cite[Theorem~2, \S
%   XVI.7]{feller71:_introd_probab_theor_its_applic,rubin65:_probab}.
%   \begin{equation}
%     \label{eq:23}
%     \frac{1 - F_{n}(u)}{1 - \Phi(u)} = \Bigl[1 + (C\tfrac{u^{3}}{\sqrt{n}}) + O(\tfrac{u^{6}}{n}) \Bigr]
%   \end{equation}
%   for some constant $C$. Letting $u_n = a_n + b_n x$ in
%   Eq.~\eqref{eq:23}, we have
%   \begin{equation*}
%     \begin{split}
%       F_n(u_n) &= 1 - (1 - \Phi(u_n))(1 + C \tfrac{u_n^{3}}{\sqrt{n}} +
%       O(\tfrac{u_n^{6}}{n})) \\
%       &= \Phi(u_n) + (1 - \Phi(u_n))(C \tfrac{u_n^{3}}{\sqrt{n}} +
%       O(\tfrac{u_n^6}{n})) \\
%       &= \Phi(u_n) + O(\tfrac{1}{u_n n^{1- \delta}})(C \tfrac{u_n^{3}}{\sqrt{n}} +
%       O(\tfrac{u_n^6}{n})) \\
%       &= \Phi(u_n) + O(\tfrac{u_n^{5}}{n^{3/2 - \delta}})
%     \end{split}
%   \end{equation*}
%   for some sufficiently small $\delta > 0$. We therefore have
%   \begin{equation}
%     \label{eq:25}
%     \begin{split}
%       \mathbb{P}(\max_{v \in [n]} \zeta(v) \leq u_n) &= (F_n(u_n))^{n} \\
%       &= \Bigl[\Phi(u_n) + O(\tfrac{u_n^{5}}{n^{3/2 - \delta}})\Bigr]^{n} \\
%       &= (\Phi(u_n))^{n} + O(\tfrac{u_n^{5}}{n^{1/2 - \delta}}) \\
%       & \rightarrow e^{-e^{-x}}. 
%     \end{split}
%   \end{equation}
%   Eq.~\eqref{eq:22} is established and we obtain the limiting Gumbel distribution for
%   $\Delta_{\lambda}(t)$ for $t < t^{*}$ in Eq.~\eqref{eq:14}.
  
%   The case when $t = t^{*}$ can be derive in a similar manner. We
%   first show that if $m = \Omega(\sqrt{n \log n})$ then
%   $\Delta_{\lambda}(v;t^{*}) \overset{\mathrm{d}}{\longrightarrow}
%   \max_{v \in [m]}{d_{\lambda}(v;t^{*})}$ \cite[Lemma
%   3.1]{rukhin:_limit_distr_graph_scan_statis}. We then show, again by
%   the central limit theorem, that for $v \in [m]$,
%   $\tfrac{d_{\lambda}(v;t^{*}) - \mu_2}{\sigma_2}
%   \overset{\mathrm{d}}{\longrightarrow} \mathcal{N}(0,1)$. It then
%   follows, similar to our previous reasoning for the case where $t <
%   t^{*}$, that $\max_{v \in [m]} \tfrac{d_{\lambda}(v;t^{*}) -
%     \mu_2}{\sigma_2}
%   \overset{\mathrm{d}}{\longrightarrow}\mathcal{G}(a_m,b_m)$ and we
%   obtain Eq.~\eqref{eq:14} for $t = t^{*}$.
% \end{IEEEproof}
% \begin{IEEEproof}[Theorem~\ref{thm:8}]
%   Let $ X \sim \mathcal{G}(\alpha, \beta)$. We consider the
%   normalization $\tfrac{X - \mu}{\sigma}$. We have
%   \begin{equation*}
%     \begin{split}
%     \mathbb{P}\Bigl[ \tfrac{X - \mu}{\sigma} \leq z\Bigr]  &= \mathbb{P}[X \leq z
%     \sigma + \mu] 
%     = e^{-e^{-(z \sigma + \mu - \alpha)/\beta}} \\
%       &= e^{- e^{-(z - (\alpha - \mu)/\sigma)/(\beta/\sigma)}}.
%     \end{split}
%   \end{equation*}
%   Thus, $\tfrac{X - \mu}{\sigma} \sim \mathcal{G}(\tfrac{\alpha -
%     \mu}{\sigma}, \tfrac{\beta}{\sigma})$. Because the sample
%   mean and the sample variance are consistent estimators, the claim
%   follows after an application of Slutsky's theorem.
% \end{IEEEproof}
% \begin{IEEEproof}[Lemma~\ref{lem:5}]
% Let $\phi_{\lambda}(v;t) = \psi_{\lambda}(v;t) - d_{\lambda}(v;t)$ be
% the (fused) locality statistics for
% vertex $v$ at time $t$ not including the (fused) degree of $v$, i.e.,
% \begin{equation}
%   \label{eq:11}
%   \phi_{\lambda}(v;t) = \sum_{\substack{uw
%       \in N(v) \\ u,w \not = v}} \langle \lambda,
%   \Gamma_{uw} \rangle.
% \end{equation}
% The following statements are conditional on $|N(v)| = l$. First
% of all, we have
% \begin{equation*}
%   \phi_{\lambda}(v;t) = \sum_{k=1}^{K}{\lambda_k z_k}
% \end{equation*}
% where the $(z_1, \dots, z_K)$ are distributed as
% \begin{gather*}
%   (z_1,z_2,\dots,z_K) \sim \textrm{multinomial}\Bigl(
%   \tbinom{l}{2}, \pi_{00}\Bigr). 
% \end{gather*}
% By the central limit theorem, we have
% \begin{equation*}
%   \frac{\phi_\lambda(v;t) - \tbinom{l}{2} \langle \lambda, \pi_{00}
%     \rangle}{\sqrt{\tbinom{l}{2} \langle \lambda, \eta_{00} \lambda \rangle}}
%   \overset{\mathrm{d}}{\longrightarrow} \mathcal{N}(0,1).
% \end{equation*}
% Let $\lambda^{(2)}$ be the element-wise square of $\lambda$. Define
% $C_{00}$ and $p_{00}$ to be 
%  \begin{gather}
%    \label{eq:26}
%    C_{00} = \tfrac{\langle \lambda^{(2)}, \pi_{00} \rangle}{\langle \lambda,
%      \pi_{00}\rangle}, \quad p_{00} = \tfrac{(\langle \lambda, \pi_{00}
%      \rangle)^{2}}{\langle \lambda^{(2)},\, \pi_{00} \rangle}.
%  \end{gather}
%  We note that $p_{00} \in [0,1]$. Now let $Y_{l} = C_{00}
%  \mathrm{Bin}(\tbinom{l}{2}, p_{00})$. Then $\mathbb{E}[Y_l] =
%  \tbinom{l}{2} \langle \lambda, \pi_{00} \rangle$ and
%  $\mathrm{Var}[Y_l] = \tbinom{l}{2} \langle \lambda, \eta_{00} \lambda
%  \rangle$ and again by the central limit theorem, we have
% \begin{equation}
%   \label{eq:12}
%   \frac{\psi_{\lambda}(v;t) - \tbinom{l}{2} \langle \lambda, \pi_{00}
%     \rangle}{\sqrt{\tbinom{l}{2} \langle \lambda, \eta_{00} \lambda
%       \rangle}}   \overset{\mathrm{d}}{\longrightarrow} \frac{Y_l -
%     \tbinom{l}{2} \langle \lambda, \pi_{00} \rangle}{\sqrt{\tbinom{l}{2}
%     \langle \lambda, \eta_{00} \lambda \rangle}}.
% \end{equation}
% Eq.~\eqref{eq:12} states that the locality statistics for our
% attributed random graphs model with $t < t^{*}$
% can be approximated by the locality statistics for an Erd\"{o}s-Renyi
% graph with edge probability $p_{00}$. The lemma then follows
% from Theorem~1.1 in \cite{rukhin:_limit_distr_graph_scan_statis}.
% \end{IEEEproof}
% \begin{IEEEproof}[Lemma~\ref{lem:6}]
%   For ease of exposition we drop the index $t^{*}$ from our discussion. Let
%   $\phi_{\lambda}(v) = \psi_{\lambda}(v) - d_{\lambda}(v)$. Let
%   $M(v)$ be the number of neighbors of $v$ that lies in $[m]$ and
%   $W(v)$ be the number of neighbors of $v$ that lies in $[n]
% \setminus [m]$. The following statements are conditional on
% $M(v) = l_{\zeta}$ and $W(v) = l_{\xi}$. We have
% \begin{equation}
%   \phi_{\lambda}(v) = \sum_{k=1}^{K} \lambda_k ( y^{(\zeta)}_k +
%   y^{(\xi)}_k + y^{(\omega)}_k)
% \end{equation}
% where $(y^{(\zeta)}_1, \dots, y^{(\zeta)}_K)$, $(y^{(\xi)}_1,\dots,
%  y^{(\xi)}_K)$,  $(y^{(\omega)}_1, \dots, y^{(\omega)}_K)$ are
%  distributed as
% \begin{gather*}
% (y^{(\zeta)}_1,\dots,y^{(\zeta)}_K) \sim \textrm{multinomial}\Bigl(
% \tbinom{l_\zeta}{2}, \pi_{11}\Bigr) \\ 
% (y^{(\xi)}_1,\dots,y^{(\xi)}_K) \sim \textrm{multinomial}\Bigl(
% \tbinom{l_\xi}{2}, \pi_{00}\Bigr) \\
% (y^{(\omega)}_1,\dots,y^{(\omega)}_m) \sim \textrm{multinomial}\Bigl(
% l_\zeta l_\xi, \pi_{10}\Bigr).
% \end{gather*}
% Let $\rho$ and $\varsigma$ be defined as
% \begin{gather*}
%   \rho = \langle \lambda, \tbinom{l_{\zeta}}{2} \pi_{11} +
%   \tbinom{l_{\xi}}{2} \pi_{00} + l_{\zeta} l_{\xi} \pi_{10} \rangle \\
%   \varsigma = \langle \lambda, \Bigl(\tbinom{l_{\zeta}}{2} \eta_{11} +
%   \tbinom{l_{\xi}}{2} \eta_{00} + l_{\zeta} l_{\xi} \eta_{10}\Bigr)
%   \lambda.
%   \rangle
% \end{gather*}
% By the central limit theorem, as $l_{\zeta} \rightarrow
% \infty$ and $l_{\xi} \rightarrow \infty$
% \begin{equation}
%   \label{eq:16}
%   \frac{\phi_{\lambda}(v) - \rho}{\varsigma}  \overset{\mathrm{d}}{\longrightarrow}  \mathcal{N}(0,1)
% \end{equation}
% Let $\lambda^{(2)}$ be the
% element-wise square of $\lambda$. Define $C_{00}$, $C_{01}$, $C_{11}$
% and $p_{00}$, $p_{01}$, $p_{11}$ to be
%  \begin{gather}
%    \label{eq:18}
%    C_{00} = \tfrac{\langle \lambda^{(2)}, \pi_{00} \rangle}{\langle \lambda,
%      \pi_{00}\rangle}; \quad p_{00} = \tfrac{(\langle \lambda, \pi_{00}
%      \rangle)^{2}}{\langle \lambda^{(2)},\, \pi_{00} \rangle} \\
%  C_{11} = \tfrac{\langle \lambda^{(2)}, \pi_{11} \rangle}{\langle \lambda,
%      \pi_{11}\rangle}; \quad p_{11} = \tfrac{(\langle \lambda, \pi_{11}
%      \rangle)^{2}}{\langle \lambda^{(2)},\, \pi_{11} \rangle} \\
%  C_{10} = \tfrac{\langle \lambda^{(2)}, \pi_{11} \rangle}{\langle \lambda,
%      \pi_{10}\rangle}; \quad p_{10} = \tfrac{(\langle \lambda, \pi_{10}
%      \rangle)^{2}}{\langle \lambda^{(2)},\, \pi_{10} \rangle} 
%  \end{gather}
%  We note that $p_{00}$, $p_{01}$, and $p_{11}$ are all elements of
%  $[0,1]$. Now let $Y_{\zeta} \sim C_{11}
%  \mathrm{Bin}\Bigl(\tbinom{l_{\zeta}}{2}, p_{11}\Bigr)$, $Y_{\xi} \sim
%  C_{00} \mathrm{Bin}\Bigl(\tbinom{l_\xi}{2}, p_{00}\Bigr)$ and
%  $Y_{\omega} \sim C_{10} \mathrm{Bin}\Bigl( l_{\zeta} l_{\xi}, p_{10}
%  \Bigr)$. We also set $Y = Y_{\zeta} + Y_{\xi} + Y_{\omega}$. By the
%  central limit theorem, we have
% \begin{equation}
%   \label{eq:21}
%  \frac{\phi_{\lambda}(v) - \rho}{\varsigma}
%  \overset{\mathrm{d}}{\longrightarrow}  \frac{Y - \rho}{\varsigma}.
% \end{equation}
% Eq.~\eqref{eq:21} states that the locality statistics
% $\phi_\lambda(v)$ for our attributed random graphs model at time $t =
% t^{*}$ can be approximated by the locality statistics $Y(v)$ for an
% unattributed kidney and egg model. The limiting distribution for the
% scan statistics in unattributed kidney-egg graphs had previously been
% considered in \cite{rukhin:_limit_distr_graph_scan_statis}. We
% provided a sketch of the arguments from
% \cite{rukhin:_limit_distr_graph_scan_statis} below, along with some
% minor changes to handle the case where the probability of
% kidney-kidney and
% kidney-egg connections are different. \\ \\
% \noindent
% Let $G$ be an instance of
% $\kappa(n,m,p_{11}, p_{10}, p_{00})$, an unattributed kidney-egg graph
% with the probability of egg-egg, egg-kidney, and kidney-kidney
% connections being $p_{11}$, $p_{10}$, and $p_{00}$,
% respectively. $D(v) = M(v) + W(v)$ is then the degree of $v$ in
% $G$. We now show two inequalities relating the tail distribution of
% $\Delta(G)$ and $\Upsilon(G) = \max_{v \in V(G)} Y(v)$.
% \begin{gather}
%   \label{eq:27}
%     \limsup\,\, \mathbb{P}( \Upsilon(G) \geq a_{n,m} ) \leq \lim
%    \mathbb{P}( \Delta(G) \geq N_\kappa), \\
%    \label{eq:30}
%   \liminf\,\, \mathbb{P}( \Upsilon(G) \geq a_{n,m} ) \geq \lim \mathbb{P}(
%   \Delta(G) \geq N_{\kappa}).
% \end{gather}
% \begin{IEEEproof}[Eq.~\eqref{eq:27}]
%  Let $C^{*} = \max\{C_{11},
% C_{10}, C_{00}\}$ and $d_{n,m} = \sqrt{2 a_{n,m}/C^{*}}$. We first
% note that
% \begin{equation*}
% \Upsilon(G) \geq a_{n,m} \Rightarrow C^{*} \tbinom{D(v)}{2} \geq
% a_{n,m} \Rightarrow D(v) \geq d_{n,m}.
% \end{equation*}
% Let us define $h(v) = \mathbb{E}[Y(v)]$, i.e., 
% \begin{equation*}
%   \begin{split}
%   h(v) = C_{00}p_{00} \tbinom{D(v)}{2} &+ (C_{11} p_{11} - C_{00}
%   p_{00}) \tbinom{M(v)}{2} \\ &+ (C_{10} p_{10} - C_{00} p_{00}) M(v)
%   W(v).
%   \end{split}
% \end{equation*}
% We then have
%   \begin{equation*}
%     \label{eq:31}
%     \begin{split}
%     \mathbb{P}(\Upsilon(G) \geq a_{n,m}) &= \mathbb{P}\Bigl( \bigcup_{v
%       \in V(G)} Y(v) \geq a_{n,m}\Bigr) \\
%     &= \mathbb{P}\Bigl( \bigcup_{v
%       \in V(G)} Y(v) \geq a_{n,m}, \, D(v) \geq d_{n,m} \Bigr) \\
%     &\leq P_1 + P_2
%     \end{split}
%   \end{equation*}
%   where 
%   \begin{align*}
%     \vartheta_n &= C_{00} \Bigl[ \tbinom{n}{2} p_{00} (1 -
%     p_{00})\Bigr]^{1/2} \log{n} \\
%     P_1 &= \mathbb{P}(\bigcup_{v \in V(G)} D(v) \geq d_{n,m}, h(v) \geq
%     a_{n,m} - \vartheta_n) \\
%     P_2 &= \mathbb{P}(\bigcup_{v \in V(G)} D(v) \geq d_{n,m},
%     Y(v) - h(v) \geq \vartheta_n).
%   \end{align*}
%   We now show that $P_2$ is negligible as $n \rightarrow \infty$. To
%   proceed, let $A$ be the event $\{M(v) = e, W(v) = f\}$ and let
%   $p_{e,f} = \mathbb{P}(A)$. $P_2$ can then be bounded as follows
% \begin{equation*}
%   \begin{split}
%     \frac{P_2}{n} & \leq \sum_{e + f \geq
%         d_{n,m}}{\mathbb{P}( Y(v) - h(v) \geq
%       \vartheta_{n} \, | \, A)} p_{e,f}  \\
%   &=  \sum_{e + f \geq d_{n,m}}{\mathbb{P}\Bigl(
%     \tfrac{Y(v) - h(v)}{\mathrm{Var}[Y(v)]^{1/2}} \geq
%     \tfrac{\vartheta_n}{\mathrm{Var}[Y(v)]^{1/2}} \, \Bigl| \, A\Bigr) p_{e,f}} \\
%     &\leq \sum_{e + f \geq d_{n,m}} (1 + o(1))
%     \mathbb{P}(Z \geq \Theta(\log{n})) p_{e,f} \\ &= o(n^{-1}).
%   \end{split}
% \end{equation*}
% We now consider $P_1$. We note that $P_1 \leq R_1 + R_2$
% where
% \begin{align*}
%   R_1 &= \mathbb{P}\Bigl(\bigcup_{v \in [m]} D(v) \geq
%   d_{n,m}, h(v) \geq a_{n,m} - \vartheta_n \Bigr), \\  
%   R_2 &= \mathbb{P}\Bigl(\bigcup_{v \in [n] \setminus [m]} D(v) \geq
%   d_{n,m}, h(v) \geq a_{n,m} - \vartheta_n \Bigr).
% \end{align*}
% Let us define $g(v) = h(v) - C_{00}p_{00} \tbinom{D(v)}{2}$.
% $R_1$ is then bounded as follows
% \begin{equation}
%   \begin{split}
%     \label{eq:36}
%     R_1 &\leq \mathbb{P}\Bigl( \bigcup_{v \in [m]} h(v) \geq
%     a_{n,m} - \vartheta_{n} \Bigr) \\
%     % &\leq \mathbb{P}\Bigl( \bigcup_{v \in [n_1]}  C_{00} p_{00}
%     % \tbinom{D(v)}{2} \geq a_{n,m} - \vartheta_{n} - g(v) \Bigr) \\
%     &\leq \mathbb{P}\Bigl(\bigcup_{v \in [m]} D(v) \geq
%       \sqrt{\tfrac{2 (a_{n,m} - \vartheta_n -
%         g(v))}{C_{00} p_{00}}} \Bigr).
%   \end{split}
% \end{equation}
% We now consider the term $a_{n,m} - g(v)$. We have
% \begin{equation*}
%   \begin{split}
%   a_{n,m} - g(v) &= C_{00} p_{00} \tbinom{N_\kappa}{2} \\ &+ (C_{11}p_{11} -
%   C_{00} p_{00})(\tbinom{\mu_E}{2} - \tbinom{M(v)}{2})
%   \\ &+ (C_{10} p_{10} - C_{00} p_{00})(\mu_E\mu_F - M(v)W(v)).
%   \end{split}
% \end{equation*}
% Let $\mathfrak{E}$ and $\mathfrak{F}$ be sets of vertices defined by
% \begin{align}
% \mathfrak{E} &=
% \{v \colon |M(v) - \mu_E| \leq \sigma_E \log{m}\} \\ \mathfrak{F} &=
% \{v \colon |W(v) - \mu_F| \leq \sigma_F \log{(n-m)}\}.
% \end{align}
% Then we have, for $v \in \mathfrak{E} \cap \mathfrak{F}$
% \begin{equation}
%   \label{eq:34}
%   \begin{split}
%   a_{n,m} - g(v) =  C_{00} p_{00} \tbinom{N_\kappa}{2} &+
%   \Theta(m^{3/2} \log{m}) \\ &+ 
%   \Theta(m \sqrt{n - m}) \\
%   \end{split}
% \end{equation}
% When $m = \Omega(\sqrt{n \log n})$, Eq.~\eqref{eq:34} gives
% \begin{equation}
%   \label{eq:35}
%   a_{n,m} - g(v) = N_{\kappa}^{2}\Bigl(\tfrac{C_{00}p_{00}}{2} + O(n^{-1/2 - a}
%   \log{n})\Bigr).
% \end{equation}
% for some $a > 0$. The set $\{v \in [m]\}$ can be partition into
% $\{v \in [m] \cap (\mathfrak{E} \cap\mathfrak{F})\}$ and $\{v \in [m]
% \setminus (\mathfrak{E} \cap \mathfrak{F})\}$. We can show that
% $\mathbb{P}\{v \in [m] \setminus (\mathfrak{E} \cap \mathfrak{F})\} =
% o(1)$ by using a concentration inequality, e.g., Hoeffding's
% bound. We thus have
% \begin{equation}
%   \label{eq:37}
%   \begin{split}
%     R_1 &\leq \mathbb{P}\biggl( \bigcup_{\substack{v \in [m] \\ v
%         \in \mathfrak{E} \cap \mathfrak{F}}} D(v) \geq
%     N_{\kappa} \sqrt{1 + O(\tfrac{\log{n}}{n^{1/2+a}})} \Bigr) + o(n^{-1}) \\
%     & = \mathbb{P}\Bigl(\bigcup_{v \in [m]} D(v) \geq N_\kappa +
%     O(n^{1/2 - a} \log{n})\Bigr) + o(n^{-1}) \\
%     &= \mathbb{P}\Bigl(\Delta \geq \mu_{E+F} +
%     \sigma_{E+F}(z_m + O(\tfrac{\log{n}}{n^{a}}))\Bigr) + o(n^{-1}) \\ 
%     & \rightarrow \mathbb{P}(\Delta \geq N_{\kappa}).
%     \end{split}
% \end{equation}
% The same argument can be applied to $R_2$ to show that
% \begin{equation}
%   \label{eq:38}
%   R_2 \leq \mathbb{P}\Bigl(\bigcup_{v \in [n] \setminus [m]} D(v) \geq
%   N_{\kappa}(1 + o(1))\Bigr) = o(1).
% \end{equation}
% Eq.~\eqref{eq:27} is therefore established.
% \end{IEEEproof}
% \begin{IEEEproof}[Eq.~\eqref{eq:30}]
%   We start by noting that
%   \begin{equation*}
%     \begin{split}
%       \mathbb{P}(\Upsilon(G) \geq a_{n,m}) &=
%       \mathbb{P}\Bigl(\bigcup_{v \in [n]}Y(v) \geq
%       a_{n,m}\Bigr) \\
%       & \geq \mathbb{P}\Bigl(\bigcup_{v \in [m]}Y(v) \geq
%       a_{n,m}, D(v) \geq N_\kappa \Bigr) \\
%       & \geq \mathbb{P}\Bigl(\bigcup_{v \in [m]} D(v) \geq
%       N_\kappa\Bigr) \\ &- \mathbb{P}\Bigl( \bigcup_{v
%         \in [m]} Y(v) < a_{n,m}, D(v) \geq N_\kappa\Bigr).
%     \end{split}
%   \end{equation*}
%   We now show that $\mathbb{P}( \cup_{v
%         \in [m]} Y(v) < a_{n,m}, D(v) \geq N_\kappa) \rightarrow 0$ as
%       $n \rightarrow \infty$. Let $v \in [m]$ be arbitrary. It is then
%       sufficient to show that $m\mathbb{P}(Y(v) < a_{n,m}, D(v) \geq
%       N_{\kappa}) = o(1)$. We note that $\mathbb{P}(Y(v) < a_{n,m}, D(v) \geq
%       N_{\kappa})$ can be rewritten as
%       \begin{equation}
%         \label{eq:24}
%         \sum_{e + f \geq N_{\kappa}}{\mathbb{P}(Y(v) \leq a_{n,m} \, |
%           \, M(v) = e, W(v) =
%           f)}p_{e,f}.
%       \end{equation}
% We now split the indices set $e + f \geq N_{\kappa}$ in
% Eq.~\eqref{eq:24} into three parts $S_1$, $S_2$ and $S_3$, namely
% \begin{gather}
%   \label{eq:32}
%     S_1 = \{  e \geq
%         \mu_E + \sigma_E \log{m}\} \\
%     S_2 = \{  e \leq
%         \mu_E + \sigma_E \log{m},e + f \leq
%         N_\kappa + \varphi(n)\} \\
%     S_3 = \{ e \leq \mu_E + \sigma_E \log{m}, e + f \geq
%         N_\kappa + \varphi(n)\}
% \end{gather}
% where $\varphi(n) = \Theta(n^{1/2 - a})$ for some $a > 0$. We can then
% show that $m\mathbb{P}(M(v) = e,
% W(v) =f, \{e,f\} \in S_1) = o(1)$ by applying a concentration
% inequality. Similarly, $e + f \geq N_{\kappa}$ and $e \leq \mu_{E}
% + \sigma_{E} \log{m}$ implies that
% \begin{equation}
%   \label{eq:33}
%   f \geq \mu_{F} + (z_{m} - o(1)) \sigma_F 
% \end{equation}
% and once again, by a concentration inequality, we can show that $m
% \mathbb{P}(M(v) = e, W(v) = f, \{e,f\} \in S_2) = o(1)$. As for
% $S_3$, from the fact that $e + f \geq N_{\kappa} + \varphi(n)$, we have the bound
%   \begin{equation}
%     \begin{split}
%     a_{n,m} - h(v) &\leq (C_{11} p_{11} - C_{10} p_{10})[m \sigma_E
%     \log{m} + \tbinom{\log{m}}{2}] \\ &- C_{00}p_{00} N_\kappa
%     \varphi(n).
%     \end{split}
%   \end{equation}
%   As $\mathrm{Var}[Y(v)] = \Theta(N_\kappa)$ for $\{M(v), W(v)\} \in
%   S_3$, we have
%   \begin{equation}
%     \label{eq:39}
%     \begin{split}
%       p_{S_3} &= \sum_{\{e,f\} \in S_3} \mathbb{P}( Y(v) < a_{n,m}) p_{e,f} 
%       \\ &\leq \sum_{\{e,f\} \in S_3} \mathbb{P}\Bigl(\tfrac{Y(v) - h(v))}{\mathrm{Var}[Y(v)]^{1/2}} \leq \tfrac{a_{n,m} -
%       h(v)}{\mathrm{Var}[Y(v)]^{1/2}}\Bigr)p_{e,f} \\
%     &\leq \sum_{\{e,f\} \in S_3} \mathbb{P}\Bigl[Z \leq
%     O\bigl(\tfrac{m^{3/2} \log m}{N_{\kappa}} - \varphi(n)\bigr)\Bigr]
%     p_{e,f}.
%     \end{split}
%   \end{equation}
%   We now set $a = \tfrac{1}{2(k+1)}$. Then for $m =
%   O(n^{k/(k+1)})$ and $\varphi(n) = O(n^{1/2 - a})$ we have
%   \begin{equation}
%     \label{eq:41}
%     \tfrac{m^{3/2} \log m}{N_\kappa} - \varphi(n) =
%     -O(n^{k/2(k+1))})
%   \end{equation}
%   which then implies
%   \begin{equation}
%     \label{eq:44}
%     m p_{S_3} \leq m \sum_{\{e,f\} \in S_3} \mathbb{P} \Bigl[Z \leq
%     - O(n^{k/2(k+1)}\Bigr] p_{e,f} = o(1).
%   \end{equation}
%   Thus $\mathbb{P}(Y(v) < a_{n,m}, D(v) \geq
%       N_{\kappa}) \rightarrow 0$ as desired.
% \end{IEEEproof}
% From Eq.~\eqref{eq:27} and Eq.~\eqref{eq:30}, we have
% \begin{equation}
%   \label{eq:40}
%  \lim
% \mathbb{P}(\Upsilon(G) \geq a_{n,m}) = \lim \mathbb{P}(\Delta(G) \geq
% N_{\kappa}).
%  \end{equation}
% Let $N_{\kappa,y} = N_\kappa + y \tfrac{\sigma_{E+F}}{\sqrt{2
%     \log{m}}}$. We now define $a_{n,m,y}$ as 
% \begin{equation*}
%   \langle \lambda, \pi_{00} \rangle
%     \tbinom{N_{\kappa,y}}{2} +
%   \langle \lambda, \pi_{11} - \pi_{00} \rangle \tbinom{\mu_E}{2}  +
%   \langle \lambda, \pi_{10} - \pi_{00} \rangle \mu_E \mu_F.
% \end{equation*}
% The above expression is equal to 
% \begin{equation}
%   \label{eq:57}
%  a_{n,m} + \langle \lambda, \pi_{00} \rangle y
%   \frac{\sigma_{E+F}}{\sqrt{2 \log m}} \Bigl(N_{\kappa} + y
%   \frac{\sigma_{E+F}^{2}}{2 \sqrt{2 \log{m}}} + O(1) \Bigr).
% \end{equation}
% We thus have
% \begin{equation*}
%    a_{n,m,y} = a_{n,m} + (y + o(1)) b_{n,m}.
% \end{equation*}
% We therefore have
% \begin{equation*}
%   \begin{split}
%   \lim \mathbb{P}(\Upsilon(G) \geq a_{n,m,y}) &= 
%   \lim \mathbb{P}\Bigl(\frac{\Upsilon(G) - a_{n,m}}{b_{n,m}} \geq y \Bigr) \\
%   &= \lim \mathbb{P}(\Delta(G) \geq N_{\kappa,y}) \\
%   &= \lim \mathbb{P}\Bigl(\frac{\Delta(G) -
%     N_{\kappa}}{\sigma_{E+F}} \geq \frac{y}{\sqrt{2 \log{m}}} \Bigr). \\
%   \end{split}
% \end{equation*}
% Because $\Delta(G)$ converges weakly to a Gumbel
% distribution in the limit
% (\cite{bollobas85:_random_graph,rukhin:_limit_distr_graph_scan_statis}),
% we have
% \begin{equation}
%   \label{eq:47}
%   \mathbb{P}\Big(\frac{\Upsilon(G) - a_{n,m}}{b_{n,m}} \leq
%     y\Bigr) \rightarrow e^{- e^{-y}} 
% \end{equation}
% and Eq.~\eqref{eq:29} follows.
% \end{IEEEproof}
\bibliography{ssp2011}
\end{document}
%%% Local Variables: 
%%% mode: latex
%%% TeX-master: t
%%% End: 
