\documentclass{article}
\pdfminorversion=4
\usepackage{parskip}
\usepackage{fontenc}
\begin{document}
\title{Response to reviewers}
\maketitle
\date{} Thank you for the careful re-reading of our manuscript and the
referee report and we deeply appreciate your comments and critiques on
the manuscript. The specific comments and suggestions for review have
been followed. The
comments on the lack of p-values are particularly relevant and we thank
the referees for pointing it out once again. We went back to the 
experiments on the Enron email corpus to obtain p-values for using
$T_{\lambda}^{l}$ as a statistic for hypothesis testing. The results
are detailed in Section 5 (on pages 9 \& 10 of the double column
version of the manuscript). Below we
give specific responses to the comments of the reviewers (the
reviewers comments are in italic). 
\section{Comments from Reviewer 1}
Recommendation: RQ - Review Again After Major Changes

Comments: 
{\em (1) The revision is much improved. The authors have included
better motivation of their approach, related the approach to signal
processing applications, included results on convergence analysis, and
reported diagnostic goodness of fit tests validating some of the
theoretical predictions.

This reviewer especially appreciates the inclusion of a review of the
weak convergence results that can be applied to specify asymptotic
distributions of the proposed test statistics.  These results are used
to compute power approximations that are in turn used to select
optimal parameters, e.g. lambda.  However, especially for a paper
emphasizing testing and anomaly detection, the principal utility of
these results are to specification of approximate p-values. Yet, the
authors do not take advantage of the stated theory on asymptotic
distributions to provide such p-values to quantify Type I error
control on statements that are made about possible anomalies- e.g. in
Figs. 7 and 8. Given the asymptotic results provided, this does not
seem difficult to do and the reviewer is somewhat puzzled that the
authors did not do it. Without such illustration of the theory, the
utility of the convergence analysis will likely be lost on the reader
and the author's final sentence "...these statistics can serve as the
basis for simple and robust inference procedures on time-series of
(attributed) graphs." suffers loss in credibility. Thus the authors
need to address this point in a next round of review.
}

*** Response: The p-values for the Enron email corpus were something we
should have done in earlier versions of the manuscript and we agree
with the referee that the principal utility of the theoretical results
in Section IV is to the specification of these approximate
p-values. We believe that the discussion on p-values in this new
version is a succinct example of the utility of the theoretical results. 

{\em (2): This reviewer still found some typographical and stylistic errors in
the revision. I suggest the last author give the paper a thorough
proofread next time to improve the english syntax.  Here are some that
I found:

Page 2 of manuscript: 
Line 10: "...can also be use to identify..." ->
"...can also be used to identify..."  
Line 46: "...(emphasis from
[4])..." -> "...(emphasis in [4])..."

Page 5 of manuscript: 
Line 13: "...an edge in G with attribute k with
probability..." -> "...an edge in G having attribute k with
probability..."

Page 9 of manuscript:
Line 7: "...done via spectral embedding technique..." ->  
"...done via spectral embedding techniques..."
Line 12: "....the issue of parameter estimation is fundamental but
feasible..." -> "....parameter estimation is  feasible..."

Caption of Fig. 8: "each graph invariants" -> "each graph invariant"
}

*** Response*** 
We have gone through the manuscript to check for errors in grammatical usage.

\section{Comments from Reviewer 2}
Recommendation: A - Publish Unaltered

Comments: 
{\em The authors improve their previous version of the manuscript
by introducing incremental changes including the following:

Very minor typos: i) the size of a n-hop neighborhood -> the size of
an n-hop neighborhood (page 2, line 48) ii) all the n-hop neighborhood
-> all n-hop neighborhoods (page 2, line 52)}

*** Response ***
We have fixed these and other typos noted by the first referee in the text.
\end{document}


%%% Local Variables: 
%%% mode: latex
%%% End: 
