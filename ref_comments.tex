Paper:T-SP-14347-2012, "Attribute fusion in a latent process model for
time series of graphs"

Dear Prof.  Priebe,

I am writing to you concerning the above referenced manuscript, which
you submitted to the IEEE Transactions on Signal Processing.

Based on the enclosed set of reviews (**See note below about
attachments), I have decided that the manuscript be REVISED AND
RESUBMITTED (RQ). Judging by the recommendations (RQ and A), one may
have impression that the reviewers differ in their opinions
substantially. As a matter of fact they don't. They both are very much
concerned by the fact that your paper still lacks the focus on
applicability and practicality (Reviewer #2 points this out very
strongly in the confidential part of his review). Reviewer #1 proposed
changes that could make your paper more practically oriented, and
hence more interesting for the signal processing community. I strongly
recommend that you follow his advice. This could result in a good
paper and would prevent us from wasting your efforts (which are very
much appreciated) and the efforts of the reviewers.

Your revised manuscript must be submitted back to Manuscript Central
http://mc.manuscriptcentral.com/tsp-ieee no later than 6 weeks from
the date of this letter together with a point-by-point reply that
explains how you addressed the reviewers' comments.

Please note that the reviewers will review your revised article only
one more time. The only decisions available to an AE after a second
round of reviews are A(Accepted), AQ(Accepted with Mandatory
Revisions), and R(Reject).

If we do not receive your revised manuscript within 6 weeks from the
date of this letter, your manuscript will be considered withdrawn.

To revise your manuscript, log into
http://mc.manuscriptcentral.com/tsp-ieee and enter your Author Center,
where you will find your manuscript title listed under "Manuscripts
with Decisions."  Under "Actions," click on "Create a Revision."  Your
manuscript number has been appended to denote a revision.

You will be unable to make your revisions on the originally submitted
version of the manuscript.  Instead, revise your manuscript using a
word processing program and save it on your computer.

Once the revised manuscript is prepared, you can upload it and submit
it through your Author Center.

When submitting your revised manuscript, you will be able to respond
to the comments made by the reviewer(s) in the space provided.  You
can use this space to document any changes you make to the original
manuscript.  In order to expedite the processing of the revised
manuscript, please be as specific as possible in your response to the
reviewer(s).

IMPORTANT: Your original files are available to you when you upload
your revised manuscript.  Please delete any redundant files before
completing the submission.

If you have any questions regarding the reviews, please contact me.
Any other inquiries should be directed to Rebecca Wollman.
Additionally, please provide a detailed explanation how the revised
manuscript addresses each of the comments from the reviewers.

Best regards,

Prof. Maciej Niedzwiecki
Associate Editor
IEEE Transactions on Signal Processing
maciekn@eti.pg.gda.pl, maciekn@pg.gda.pl

Rebecca Wollman
Coordinator Society Publications
IEEE Signal Processing Society
r.wollman@ieee.org

** {This applies to SUBMITTING AUTHOR Accounts ONLY: You can find any
  possible ATTACHMENTS FROM THE REVIEWERS by going to the "Manuscript
  with Decisions" status link in your Author Center and clicking on
  "view decision letter". They are located at the bottom of decision
  letter under "Files attached" heading}

Reviewer Comments:

Reviewer: 1

Recommendation: RQ - Review Again After Major Changes

Comments: The revision is much improved. The authors have included
better motivation of their approach, related the approach to signal
processing applications, included results on convergence analysis, and
reported diagnostic goodness of fit tests validating some of the
theoretical predictions.

This reviewer especially appreciates the inclusion of a review of the
weak convergence results that can be applied to specify asymptotic
distributions of the proposed test statistics.  These results are used
to compute power approximations that are in turn used to select
optimal parameters, e.g. lambda.  However, especially for a paper
emphasizing testing and anomaly detection, the principal utility of
these results are to specification of approximate p-values. Yet, the
authors do not take advantage of the stated theory on asymptotic
distributions to provide such p-values to quantify Type I error
control on statements that are made about possible anomalies- e.g. in
Figs. 7 and 8. Given the asymptotic results provided, this does not
seem difficult to do and the reviewer is somewhat puzzled that the
authors did not do it. Without such illustration of the theory, the
utility of the convergence analysis will likely be lost on the reader
and the author's final sentence "...these statistics can serve as the
basis for simple and robust inference procedures on time-series of
(attributed) graphs." suffers loss in credibility. Thus the authors
need to address this point in a next round of review.

This reviewer still found some typographical and stylistic errors in
the revision. I suggest the last author give the paper a thorough
proofread next time to improve the english syntax.  Here are some that
I found:

Page 2 of manuscript: 
Line 10: "...can also be use to identify..." ->
"...can also be used to identify..."  
Line 46: "...(emphasis from
[4])..." -> "...(emphasis in [4])..."

Page 5 of manuscript: 
Line 13: "...an edge in G with attribute k with
probability..." -> "...an edge in G having attribute k with
probability..."

Page 9 of manuscript:
Line 7: "...done via spectral embedding technique..." ->  
"...done via spectral embedding techniques..."
Line 12: "....the issue of parameter estimation is fundamental but
feasible..." -> "....parameter estimation is  feasible..."

Caption of Fig. 8: "each graph invariants" -> "each graph invariant"

Additional Questions:
1. Is the topic appropriate for publication in these transactions?: Yes

2. Is the topic important to colleagues working in the field?: Yes

Explain:

1. Is the paper technically sound?: Yes

why not?:

2. Is the coverage of the topic sufficiently comprehensive and balanced?: Yes

3. How would you describe technical depth of paper?: Appropriate for
the Generally Knowledgeable Individual Working in the Field or a
Related Field

4. How would you rate the technical novelty of the paper?: Somewhat Novel

1. How would you rate the overall organization of the paper?: Satisfactory

2. Are the title and abstract satisfactory?: Yes

Explain:

3. Is the length of the paper appropriate? If not, recommend how the
length of the paper should be amended, including a possible target
length for the final manuscript.: Yes

4. Are symbols, terms, and concepts adequately defined?: Yes

5. How do you rate the English usage? : Needs improvement

6. Rate the Bibliography: Satisfactory

null:

1. How would you rate the technical contents of the paper?: Good

2. How would you rate the novelty of the paper?: Sufficiently Novel

3. How would you rate the "literary" presentation of the paper?: Mostly Accessible

4. How would you rate the appropriateness of this paper for
publication in this IEEE Transactions?: Good Match


Reviewer: 2

Recommendation: A - Publish Unaltered

Comments: 
The authors improve their previous version of the manuscript
by introducing incremental changes including the following:

1. An expansion of the literature on corresponding wavelet
representations and the connection of this manuscript to previous
works (citations [1] and [7]) in Section 1.

2. A comment on the choice of the value of l and a minor section
restructuring (at the boundaries of Section 3 and 4).

3. Q-Q plots and a small discussion on relating the theoretical
results to their first simulation example (mainly second paragraph of
Section 5).

4. Potential benefits from generalizations of their results (the last
two paragraphs of Section 6).

Very minor typos: i) the size of a n-hop neighborhood -> the size of
an n-hop neighborhood (page 2, line 48) ii) all the n-hop neighborhood
-> all n-hop neighborhoods (page 2, line 52)

Additional Questions:
1. Is the topic appropriate for publication in these transactions?: Perhaps

2. Is the topic important to colleagues working in the field?: Moderately So

Explain:

1. Is the paper technically sound?: Yes

why not?:

2. Is the coverage of the topic sufficiently comprehensive and balanced?: Yes

3. How would you describe technical depth of paper?: Appropriate for
the Generally Knowledgeable Individual Working in the Field or a
Related Field

4. How would you rate the technical novelty of the paper?: Not Novel

1. How would you rate the overall organization of the paper?: Satisfactory

2. Are the title and abstract satisfactory?: Yes

Explain:

3. Is the length of the paper appropriate? If not, recommend how the
length of the paper should be amended, including a possible target
length for the final manuscript.: Yes

4. Are symbols, terms, and concepts adequately defined?: Yes

5. How do you rate the English usage? : Satisfactory

6. Rate the Bibliography: Satisfactory

null:

1. How would you rate the technical contents of the paper?: Good

2. How would you rate the novelty of the paper?: Not Novel

3. How would you rate the "literary" presentation of the paper?: Mostly Accessible

4. How would you rate the appropriateness of this paper for
publication in this IEEE Transactions?: Weak Match
%%% Local Variables: 
%%% mode: latex
%%% TeX-master: t
%%% End: 
